% =============================================================================
% Structure Tower 発展演習 Level 4
% LuaLaTeX + ltjsarticle
% =============================================================================
\documentclass[a4paper,11pt]{ltjsarticle}

% --- Core packages ---
\usepackage{luatexja-fontspec}
\usepackage{amsmath,amssymb,amsthm}
\usepackage{mathtools}
\usepackage{tikz-cd}
\usepackage{tikz}
\usetikzlibrary{arrows.meta,positioning,calc,shapes.geometric}
\usepackage{enumitem}
\usepackage{hyperref}
\usepackage{tcolorbox}
\tcbuselibrary{skins,breakable}
\usepackage{listings}
\usepackage{xcolor}
\usepackage{geometry}
\usepackage{fancyhdr}
\usepackage{float}
\usepackage{booktabs}

% --- Page geometry ---
\geometry{margin=2.5cm, top=3cm, bottom=3cm}

% --- Fonts ---
\setmainjfont{Noto Serif JP}
\setsansjfont{Noto Sans JP}

% --- Colors ---
\definecolor{leanblue}{HTML}{2563EB}
\definecolor{leanbg}{HTML}{F8FAFC}
\definecolor{leancomment}{HTML}{6B7280}
\definecolor{leankeyword}{HTML}{7C3AED}
\definecolor{leanstring}{HTML}{059669}
\definecolor{accentcolor}{HTML}{1E40AF}
\definecolor{theoremcolor}{HTML}{EFF6FF}
\definecolor{definitioncolor}{HTML}{F0FDF4}
\definecolor{remarkcolor}{HTML}{FFF7ED}
\definecolor{warningcolor}{HTML}{FEF3C7}

% --- Hyperref ---
\hypersetup{
  colorlinks=true,
  linkcolor=accentcolor,
  urlcolor=leanblue,
  citecolor=accentcolor,
  pdftitle={Structure Tower 発展演習 Level 4},
  pdfauthor={su},
}

% --- Theorem environments ---
\theoremstyle{definition}
\newtheorem{definition}{定義}[section]
\newtheorem{example}[definition]{例}

\theoremstyle{plain}
\newtheorem{theorem}[definition]{定理}
\newtheorem{lemma}[definition]{補題}
\newtheorem{proposition}[definition]{命題}
\newtheorem{corollary}[definition]{系}

\theoremstyle{remark}
\newtheorem{remark}[definition]{注意}

% --- Lean code listing environment ---
\lstdefinelanguage{lean4}{
  morekeywords={def,theorem,lemma,example,instance,class,structure,
    where,let,in,do,if,then,else,match,with,fun,return,
    import,open,namespace,section,end,variable,
    inductive,abbrev,noncomputable,private,protected,
    sorry,have,show,suffices,calc,by,exact,
    apply,intro,intros,constructor,cases,induction,
    simp,rw,rfl,ext,ring,linarith,omega,
    decide,trivial,assumption,contradiction,
    Type,Prop,Sort,Set,true,false,
    attribute,deriving,refine,change,congr,
    funext,positivity,norm_num},
  sensitive=true,
  morecomment=[l]{--},
  morecomment=[n]{/-}{-/},
  morestring=[b]",
  literate=
    {α}{{\(\alpha\)}}1
    {β}{{\(\beta\)}}1
    {γ}{{\(\gamma\)}}1
    {ι}{{\(\iota\)}}1
    {→}{{\(\to\)}}1
    {←}{{\(\leftarrow\)}}1
    {↔}{{\(\leftrightarrow\)}}1
    {≤}{{\(\leq\)}}1
    {≥}{{\(\geq\)}}1
    {≠}{{\(\neq\)}}1
    {∀}{{\(\forall\)}}1
    {∃}{{\(\exists\)}}1
    {∧}{{\(\wedge\)}}1
    {∨}{{\(\vee\)}}1
    {¬}{{\(\neg\)}}1
    {∈}{{\(\in\)}}1
    {∉}{{\(\notin\)}}1
    {⊆}{{\(\subseteq\)}}1
    {⊂}{{\(\subset\)}}1
    {∩}{{\(\cap\)}}1
    {∅}{{\(\emptyset\)}}1
    {ℕ}{{\(\mathbb{N}\)}}1
    {ℤ}{{\(\mathbb{Z}\)}}1
    {:=}{{:=}}2,
}

\lstnewenvironment{leancode}[1][]{%
  \lstset{
    language=lean4,
    basicstyle=\ttfamily\small,
    keywordstyle=\color{leankeyword}\bfseries,
    commentstyle=\color{leancomment}\itshape,
    stringstyle=\color{leanstring},
    backgroundcolor=\color{leanbg},
    frame=single,
    rulecolor=\color{leanblue!30},
    framesep=8pt,
    xleftmargin=12pt,
    xrightmargin=12pt,
    breaklines=true,
    breakatwhitespace=true,
    showstringspaces=false,
    tabsize=2,
    captionpos=b,
    numbers=left,
    numberstyle=\tiny\color{leancomment},
    numbersep=8pt,
    aboveskip=1em,
    belowskip=1em,
    #1
  }%
}{}

% --- Tcolorbox styles ---
\newtcolorbox{proofstrategy}{
  colback=remarkcolor,
  colframe=orange!60!black,
  title={\textbf{証明戦略}},
  fonttitle=\sffamily,
  boxrule=0.5pt,
  arc=3pt,
}

\newtcolorbox{mathinsight}{
  colback=theoremcolor,
  colframe=accentcolor,
  title={\textbf{数学的洞察}},
  fonttitle=\sffamily,
  boxrule=0.5pt,
  arc=3pt,
}

% --- Header / Footer ---
\pagestyle{fancy}
\fancyhf{}
\renewcommand{\headrulewidth}{0.4pt}
\fancyhead[L]{\small\sffamily\nouppercase{\leftmark}}
\fancyhead[R]{\small\sffamily Structure Tower Level 4}
\fancyfoot[C]{\thepage}

% =============================================================================
\begin{document}
% =============================================================================

% --- Title ---
\title{%
  {\LARGE\sffamily\bfseries Structure Tower 発展演習 Level 4}\\[0.5em]
  {\large\sffamily 閉包演算子の比較・σ-代数・rank 一意性・反射的部分圏}\\[0.3em]
  {\normalsize\sffamily Lean 4 / Mathlib4 による形式化とその数学的解説}%
}
\author{su}
\date{\today}
\maketitle

% --- AI Disclosure ---
\begin{center}
\small
\textit{AI assistance disclosure:}\\
Lean ソースコードは Claude (Anthropic) で骨格を生成し、
Codex (OpenAI) で修正した。\\
\TeX{} 文書は Claude Code (Anthropic) で生成した。\\
著者による加筆・修正は行っていない。\\
内容の正確性は保証されず、誤りがあれば著者の責任である。
\end{center}

\vspace{1em}

% --- Abstract ---
\begin{abstract}
本稿は Lean 4 / Mathlib4 による Structure Tower 発展演習(Level 4)の形式化を数学的に解説する。
Level 3 で構築した閉包モナドの理論を基盤として、以下の4方向に発展させる:
(1)閉包演算子の順序 $\mathrm{cl}_1 \leq \mathrm{cl}_2$($\forall S,\ \mathrm{cl}_1(S) \subseteq \mathrm{cl}_2(S)$)が
\texttt{liftCl} 間の自然な射を誘導すること;
(2)可測空間($\sigma$-代数)を第3の接地例として \texttt{ClosedTower} の枠組みで捉えること;
(3)網羅的塔(\texttt{ExhaustiveTower})における rank 関数の一意性定理(Theorem B)の証明;
(4)\texttt{ClosedTower} が \texttt{StructureTower} の反射的部分圏をなすこと(unit の普遍性)。
これにより、位相・群論・測度論の3分野を統一するインターフェースとしての
Structure Tower の完成形を示す。
\end{abstract}

\tableofcontents
\newpage

% =============================================================================
\section{はじめに}
% =============================================================================

Structure Tower プロジェクトは、集合 $\alpha$ 上の添字付き集合族
$\{T_i\}_{i \in \iota}$($i \leq j \Rightarrow T_i \subseteq T_j$)を「塔」と呼び、
異なる数学分野に共通する「閉包」「単調性」「普遍性」の構造を統一的に扱う枠組みである。

Level 1--3 では以下を構築した:
\begin{itemize}[leftmargin=2em]
  \item \textbf{L1}: \texttt{StructureTower} の公理・射・圏論的な基礎
  \item \textbf{L2}: \texttt{union}, \texttt{global} など基本演算の性質
  \item \textbf{L3}: 閉包モナド $(\mathrm{liftCl},\ \mathrm{unit},\ \mathrm{join})$、
        \texttt{ClosedTower}(不動点の塔)の理論;
        位相的閉包と部分群生成による接地
\end{itemize}

Level 4 では4方向に理論を拡張する:

\begin{description}
  \item[\S\ref{sec:clparametric}\ \textbf{cl-parametric 比較}]
    異なる閉包演算子の比較が liftCl 間の射を誘導する。

  \item[\S\ref{sec:sigma}\ \textbf{$\sigma$-代数の接地}]
    \texttt{MeasurableTower} として可測空間を統合し、3分野統合を完成させる。

  \item[\S\ref{sec:rank}\ \textbf{Rank uniqueness}]
    網羅的塔における rank 関数の一意性定理(Theorem B)を証明する。

  \item[\S\ref{sec:cat}\ \textbf{ClosedTower の圏}]
    liftCl が reflector となり、ClosedTower は反射的部分圏をなす。
\end{description}

% =============================================================================
\section{準備:基本定義の再掲}
\label{sec:prelim}
% =============================================================================

本節では Level 1--3 で構築した定義を簡潔に再掲する。

\begin{definition}[\texttt{StructureTower}]
前順序集合 $(\iota, \leq)$ と集合 $\alpha$ に対し、\textbf{構造塔}(\texttt{StructureTower})とは
単調な集合族 $T = (T_i)_{i \in \iota}$ である:
\[
  i \leq j \implies T_i \subseteq T_j.
\]

\begin{leancode}
structure StructureTower (i a : Type*) [Preorder i] : Type _ where
  level : i -> Set a
  monotone_level : forall (i j : i), i <= j -> level i <= level j
\end{leancode}
\end{definition}

\begin{definition}[\texttt{Hom}]
$T_1,\ T_2 : \mathrm{StructureTower}\ \iota\ \alpha$ に対し、\textbf{射}(\texttt{Hom})とは
各レベルを保つ関数 $f : \alpha \to \alpha$:
$\forall i,\ f(T_1^i) \subseteq T_2^i$。
恒等射 $\mathrm{id}$ と合成 $g \circ f$ は通常の関数の合成。
\end{definition}

\begin{definition}[\texttt{liftCl}, \texttt{unit}, \texttt{join}]
閉包演算子 $\mathrm{cl} : \mathcal{P}(\alpha) \to \mathcal{P}(\alpha)$ に対し、
\[
  (\mathrm{liftCl}\ \mathrm{cl}\ T)_i := \mathrm{cl}(T_i).
\]
自然変換として
$\mathrm{unit}_T : T \to \mathrm{liftCl}\ \mathrm{cl}\ T$(恒等写像)、
$\mathrm{join}_T : \mathrm{liftCl}\ \mathrm{cl}\ (\mathrm{liftCl}\ \mathrm{cl}\ T)
               \to \mathrm{liftCl}\ \mathrm{cl}\ T$(恒等写像)
を定義する。これらは閉包演算子上の(冪等)モナドの単位と乗法に対応する。
\end{definition}

\begin{definition}[\texttt{ClosedTower}]
$T$ が $\mathrm{cl}$-\textbf{閉塔}(\texttt{ClosedTower})であるとは,各レベルが $\mathrm{cl}$ の不動点:
\[
  \forall i,\quad \mathrm{cl}(T_i) = T_i.
\]
\end{definition}

\begin{figure}[H]
\centering
\begin{tikzcd}[column sep=3cm, row sep=2cm]
  T
    \arrow[r, "\mathrm{unit}_T"]
  & \mathrm{liftCl}\ \mathrm{cl}\ T
    \arrow[r, "\mathrm{unit}_{\mathrm{liftCl}\ T}"]
  & \mathrm{liftCl}^2\ \mathrm{cl}\ T
    \arrow[l, bend right=20, "\mathrm{join}_T"']
\end{tikzcd}
\caption{モナドの単位律:$\mathrm{join} \circ \mathrm{unit}_{\mathrm{liftCl}} = \mathrm{id}$(冪等性の表れ)。}
\label{fig:monad}
\end{figure}

% =============================================================================
\section{cl-parametric な構造比較}
\label{sec:clparametric}
% =============================================================================

\subsection{動機}

Level 3 では「固定した閉包演算子 $\mathrm{cl}$」について塔を研究した。
Level 4 では視点を変え、「異なる閉包演算子 $\mathrm{cl}_1,\ \mathrm{cl}_2$ を
同一の塔 $T$ に適用する」比較に着目する。

直感的には、$\mathrm{cl}_1 \leq \mathrm{cl}_2$(弱い順から強い順)のとき
$\mathrm{cl}_1(T_i) \subseteq \mathrm{cl}_2(T_i)$ が各レベルで成立し、
自然な射 $\mathrm{liftCl}\ \mathrm{cl}_1\ T \to \mathrm{liftCl}\ \mathrm{cl}_2\ T$
が得られる。

\subsection{定義と主要定理}

\begin{definition}[\texttt{ClLeq}]
\[
  \mathrm{cl}_1 \leq \mathrm{cl}_2 \quad:\Longleftrightarrow\quad
  \forall S \subseteq \alpha,\quad \mathrm{cl}_1(S) \subseteq \mathrm{cl}_2(S).
\]
\end{definition}

\begin{proposition}[比較射の存在, L4-1a]
$\mathrm{cl}_1 \leq \mathrm{cl}_2$ かつ $T : \mathrm{StructureTower}\ \iota\ \alpha$ のとき、
\[
  \mathrm{liftCl\_comparison} :
  \mathrm{Hom}(\mathrm{liftCl}\ \mathrm{cl}_1\ T,\ \mathrm{liftCl}\ \mathrm{cl}_2\ T)
\]
が存在する($\mathrm{toFun} = \mathrm{id}$)。
\end{proposition}

\begin{proofstrategy}
$\mathrm{preserves}$:$x \in \mathrm{cl}_1(T_i)$ を仮定し、
仮定 $\mathrm{cl}_1 \leq \mathrm{cl}_2$ を集合 $S = T_i$ に適用して
$x \in \mathrm{cl}_2(T_i)$ を得る。
\begin{leancode}
def liftCl_comparison {cl1 cl2 : ClosureOperator (Set a)}
    (hle : ClLeq cl1 cl2) (T : StructureTower i a) :
    Hom (liftCl cl1 T) (liftCl cl2 T) where
  toFun := id
  preserves := fun i x hx => hle (T.level i) hx
\end{leancode}
\end{proofstrategy}

\begin{proposition}[比較射の推移性, L4-1b]
\label{prop:trans}
$\mathrm{cl}_1 \leq \mathrm{cl}_2 \leq \mathrm{cl}_3$ のとき、
\[
  \mathrm{comparison}_{23} \circ \mathrm{comparison}_{12}
  = \mathrm{comparison}_{13}.
\]
\end{proposition}

\begin{proof}
両辺の $\mathrm{toFun}$ がいずれも $\mathrm{id} \circ \mathrm{id} = \mathrm{id}$ なので
$\mathrm{Hom.ext\ rfl}$ により等しい。
\end{proof}

\begin{figure}[H]
\centering
\begin{tikzcd}[column sep=3cm, row sep=2.5cm]
  \mathrm{liftCl}\ \mathrm{cl}_1\ T
    \arrow[r, "\mathrm{comparison}_{12}"]
    \arrow[dr, "\mathrm{comparison}_{13}"', bend right=10]
  & \mathrm{liftCl}\ \mathrm{cl}_2\ T
    \arrow[d, "\mathrm{comparison}_{23}"] \\
  & \mathrm{liftCl}\ \mathrm{cl}_3\ T
\end{tikzcd}
\caption{$\mathrm{cl}_1 \leq \mathrm{cl}_2 \leq \mathrm{cl}_3$ のとき、
比較射の合成は推移的(命題~\ref{prop:trans})。}
\label{fig:comparison}
\end{figure}

\begin{proposition}[弱化(Weakening), L4-1c]
\label{prop:weaken}
$\mathrm{cl}_1 \leq \mathrm{cl}_2$ のとき、
\[
  T \in \mathrm{ClosedTower}_{\mathrm{cl}_2} \implies T \in \mathrm{ClosedTower}_{\mathrm{cl}_1}.
\]
\emph{すなわち「強い閉包の不動点は弱い閉包の不動点でもある」。}
\end{proposition}

\begin{proof}
各 $i$ で $\mathrm{cl}_1(T_i) = T_i$ を示す。
\begin{align*}
  \mathrm{cl}_1(T_i) &\subseteq \mathrm{cl}_2(T_i) = T_i
    && \text{($\mathrm{cl}_1 \leq \mathrm{cl}_2$ と $T$ の閉性)} \\
  T_i &\subseteq \mathrm{cl}_1(T_i)
    && \text{(閉包の拡大性)}
\end{align*}
よって反対称性から $\mathrm{cl}_1(T_i) = T_i$。
Lean では \texttt{Set.Subset.antisymm} と \texttt{cl.le\_closure} を組み合わせる。
\end{proof}

\begin{figure}[H]
\centering
\begin{tikzcd}[column sep=3.5cm, row sep=1.5cm]
  \mathrm{ClosedTower}_{\mathrm{cl}_2}
    \arrow[r, hook, "\text{弱化(命題~\ref{prop:weaken})}"]
  & \mathrm{ClosedTower}_{\mathrm{cl}_1}
    \arrow[r, hook]
  & \mathrm{StructureTower}
\end{tikzcd}
\caption{$\mathrm{cl}_1 \leq \mathrm{cl}_2$ のとき、
$\mathrm{cl}_2$-ClosedTower $\subseteq$ $\mathrm{cl}_1$-ClosedTower(強い閉包ほど不動点が少ない)。}
\label{fig:weaken-incl}
\end{figure}

\begin{proposition}[吸収条件, L4-1d/e]
\label{prop:absorb}
\begin{enumerate}[label=(\roman*)]
  \item $\forall S,\ \mathrm{cl}_2(\mathrm{cl}_1(S)) = \mathrm{cl}_2(S)$ のとき、
        \[
          \mathrm{liftCl}\ \mathrm{cl}_2\ (\mathrm{liftCl}\ \mathrm{cl}_1\ T)
          = \mathrm{liftCl}\ \mathrm{cl}_2\ T.
        \]
  \item 特に $\mathrm{cl}_1 = \mathrm{cl}_2 = \mathrm{cl}$(冪等性 $\mathrm{cl}(\mathrm{cl}(S)) = \mathrm{cl}(S)$)では
        \[
          \mathrm{liftCl}\ \mathrm{cl}\ (\mathrm{liftCl}\ \mathrm{cl}\ T) = \mathrm{liftCl}\ \mathrm{cl}\ T.
        \]
\end{enumerate}
\end{proposition}

\begin{mathinsight}
命題~\ref{prop:absorb}(ii) は Level 3 で構築した \texttt{join} が同型射であることの別表現である。
集合のモナドとしての冪等性 $\mu \circ T\mu = \mu \circ \mu T = \mathrm{id}$ に相当し、
liftCl が「べき等モナド」として振る舞うことを示している。
\end{mathinsight}

% =============================================================================
\section{$\sigma$-代数への第3の接地}
\label{sec:sigma}
% =============================================================================

\subsection{設計方針}

Level 3 では閉包演算子 $\mathrm{cl}$ を介した2つの接地を行った:

\begin{center}
\begin{tabular}{lll}
  \toprule
  分野 & 閉包演算子 & 不動点条件 \\
  \midrule
  位相空間論 & \texttt{topClosure} $S = \overline{S}$ & $S$ が閉集合 \\
  群論       & \texttt{subgroupClosure} $S = \langle S\rangle$ & $S$ が部分群の台 \\
  \bottomrule
\end{tabular}
\end{center}

$\sigma$-代数の場合、Mathlib の \texttt{MeasurableSet} は
$\texttt{MeasurableSet}\ S : \mathtt{Prop}$ であり、
$\mathtt{Set}\ \alpha \to \mathtt{Set}\ \alpha$ 型の閉包演算子として直接表現しにくい
(生成 $\sigma$-代数は存在するが、Lean 上で \texttt{ClosureOperator} として扱うための
型クラスの調整が煩雑)。
そのため、\texttt{ClosureOperator} を経由せず、各レベルへの可測性条件を
直接付与する構造体 \texttt{MeasurableTower} を定義する。

\begin{figure}[H]
\centering
\begin{tikzpicture}[
  box/.style={draw, rounded corners=5pt, fill=theoremcolor,
              minimum width=3.8cm, minimum height=1.4cm,
              align=center, font=\small\sffamily},
  cbox/.style={draw, rounded corners=5pt, fill=definitioncolor,
               minimum width=3.8cm, minimum height=1.4cm,
               align=center, font=\small\sffamily},
  arr/.style={-{Stealth[length=7pt]}, thick, accentcolor},
  label/.style={font=\footnotesize\itshape, fill=white, inner sep=2pt}
]
  \node[box] (ST) at (0,3.5) {StructureTower\\$T_i \subseteq T_{i+1}$};
  \node[cbox] (top) at (-4.5,0) {ClosedTower\\[2pt] $\overline{T_i} = T_i$\\{\tiny (topClosure)}};
  \node[cbox] (grp) at (0,0) {ClosedTower\\[2pt] $\langle T_i\rangle = T_i$\\{\tiny (subgroupClosure)}};
  \node[cbox] (meas) at (4.5,0) {MeasurableTower\\[2pt] $T_i \in \mathcal{M}$\\{\tiny (直接定義)}};

  \draw[arr] (ST) -- node[label, left]{位相空間論} (top);
  \draw[arr] (ST) -- node[label, right]{群論} (grp);
  \draw[arr] (ST) -- node[label, right]{測度論} (meas);
\end{tikzpicture}
\caption{StructureTower を中心とした3分野の接地。
各分野で「各レベルが性質 $P$ を持つ塔の global も $P$ を持つ」が成立する(定理~\ref{thm:three-fields})。}
\label{fig:three-groundings}
\end{figure}

\subsection{MeasurableTower の定義と性質}

\begin{definition}[\texttt{MeasurableTower}]
可測空間 $(\alpha, \mathcal{M})$ と前順序集合 $(\iota, \leq)$ に対し、
\textbf{可測塔}(\texttt{MeasurableTower})とは StructureTower $T$ であって
各レベルが可測集合であるもの:
\[
  \forall i \in \iota,\quad T_i \in \mathcal{M}.
\]

\begin{leancode}
structure MeasurableTower (i : Type*) [Preorder i]
    extends StructureTower i a where
  level_measurable : forall i, MeasurableSet (level i)
\end{leancode}
\end{definition}

\begin{proposition}[基本構成, L4-2a/b]
\begin{enumerate}[label=(\roman*)]
  \item 任意の可測集合 $S \in \mathcal{M}$ に対し、定数塔 $T_i := S$ は \texttt{MeasurableTower}。
  \item $T_i := \alpha$(\texttt{univTower})と $T_i := \emptyset$(\texttt{emptyTower})はいずれも
        \texttt{MeasurableTower}。
        Lean では \texttt{MeasurableSet.univ}, \texttt{MeasurableSet.empty} を使う。
\end{enumerate}
\end{proposition}

\begin{proposition}[演算の閉性, L4-2c/d/e]
\label{prop:meas-ops}
\begin{enumerate}[label=(\roman*)]
  \item \textbf{交叉}(L4-2c): $T_1,\ T_2$ が \texttt{MeasurableTower} ならば
        $(T_1 \cap T_2)_i := T_1^i \cap T_2^i$ も \texttt{MeasurableTower}。
  \item \textbf{補集合の可測性}(L4-2d): $T$ が \texttt{MeasurableTower} ならば
        $\forall i,\ (T_i)^c \in \mathcal{M}$。
        ただし $(T_i)^c$ は単調減少なので \texttt{StructureTower} にはならない。
  \item \textbf{global の可測性}(L4-2e): $\iota$ が可算(\texttt{Countable})なら
        \[
          T.\mathrm{global} := \bigcap_{i \in \iota} T_i \in \mathcal{M}.
        \]
\end{enumerate}
\end{proposition}

\begin{proof}[(iii) の証明]
$T.\mathrm{global} = \bigcap_{i \in \iota} T_i$ であり、
$\iota$ が可算のとき Mathlib の \texttt{MeasurableSet.iInter} を用いる:

\begin{leancode}
theorem global_measurable [Countable i] (T : MeasurableTower i) :
    MeasurableSet T.global := by
  change MeasurableSet (Set.iInter T.level)
  exact MeasurableSet.iInter (fun i => T.level_measurable i)
\end{leancode}

可算性条件は実質的であり、非可算添字集合 $\iota$ に対しては一般に成立しない
(Vitali 集合など)。
\end{proof}

\begin{theorem}[3分野統合定理, L4-2f]
\label{thm:three-fields}
以下の3分野で「各レベルが性質 $P$ を持つ塔の global も $P$ を持つ」が成立する:

\begin{center}
\begin{tabular}{lcl}
  \toprule
  分野 & 性質 $P$ & global の閉性 \\
  \midrule
  位相空間論 & $T_i$ が閉集合 & $\bigcap_i T_i$ も閉集合 \\
  群論       & $T_i$ が部分群の台 & $\bigcap_i T_i$ も部分群の台 \\
  測度論     & $T_i \in \mathcal{M}$ & $\bigcap_i T_i \in \mathcal{M}$($\iota$ 可算) \\
  \bottomrule
\end{tabular}
\end{center}
\end{theorem}

\begin{mathinsight}
$\sigma$-代数は「可算交叉に閉じた集合族」という公理から出発しており、
定理~\ref{thm:three-fields}(測度論) はまさにその公理の反映である。
位相的閉集合の可算交叉が閉じることも同様の直感を持つ。
Structure Tower の枠組みはこの共通構造を抽象的に捉えている。
\end{mathinsight}

% =============================================================================
\section{Rank uniqueness(Theorem B)}
\label{sec:rank}
% =============================================================================

\subsection{設定}

\begin{definition}[\texttt{ExhaustiveTower}]
$T : \mathrm{ExhaustiveTower}\ \mathbb{N}\ \alpha$ とは、
各 $x \in \alpha$ が何らかのレベルに属する StructureTower:
\[
  \forall x \in \alpha,\quad \exists i \in \mathbb{N},\quad x \in T_i.
\]
このとき、$x$ の \textbf{rank} を
\[
  \mathrm{rank}(x) := \min\bigl\{i \in \mathbb{N} \mid x \in T_i\bigr\}
\]
と定義する(Lean では \texttt{Nat.find} を用いる)。
\end{definition}

\begin{definition}[\texttt{HasCharRank}]
\label{def:charrank}
関数 $r : \alpha \to \mathbb{N}$ が $T$ の \textbf{特徴付け rank}(\texttt{HasCharRank})であるとは、
\[
  \forall x \in \alpha,\ \forall i \in \mathbb{N},\quad
  x \in T_i \iff r(x) \leq i.
\]
\end{definition}

\begin{figure}[H]
\centering
\begin{tikzpicture}[scale=0.9]
  % Level boxes
  \foreach \i in {0,1,2,3,4} {
    \draw[fill=theoremcolor!60, draw=accentcolor!70, rounded corners=2pt]
      (\i*2.2, 0) rectangle (\i*2.2+1.9, 0.9);
    \node[font=\small] at (\i*2.2+0.95, 0.45) {$T_{\i}$};
  }

  % Element x
  \node[circle, fill=leanblue!80, draw=leanblue, thick,
        inner sep=3pt, label=above:{$x$}] (x) at (4.95+0.95, 2.2) {};

  % Dashed lines showing membership
  \foreach \i in {2,3,4} {
    \draw[dashed, leanblue!60, thin] (x) -- (\i*2.2+0.95, 0.9);
  }
  % Cross marks for non-membership
  \foreach \i in {0,1} {
    \node[red!70, font=\Large] at (\i*2.2+0.95, -0.4) {$\times$};
  }
  % Check marks for membership
  \foreach \i in {2,3,4} {
    \node[leanblue!80, font=\normalsize] at (\i*2.2+0.95, -0.4) {$\checkmark$};
  }

  % rank arrow
  \draw[{Stealth[length=5pt]}-{Stealth[length=5pt]}, orange!80!black, thick]
    (0, -0.85) -- (4.4, -0.85)
    node[midway, below, font=\small\sffamily] {$r(x) = 2$};

  % Annotation
  \node[font=\footnotesize, align=left] at (10.5, 1.1)
    {$x \in T_i \iff r(x) \leq i$\\[2pt]
     HasCharRank の条件};
\end{tikzpicture}
\caption{$r(x) = 2$ のとき、$x \in T_i$ $\iff$ $2 \leq i$。
HasCharRank は「塔のレベルが $\mathrm{Iic}(r(x))$ の逆像」であることを意味する。}
\label{fig:rank}
\end{figure}

\subsection{補題と主定理}

\begin{lemma}[rank と HasCharRank の対応, L4-3a/b/c]
\label{lem:rank-char}
任意の $T : \mathrm{ExhaustiveTower}\ \mathbb{N}\ \alpha$ に対し、
$\mathrm{rank}$ は HasCharRank を満たす:
\[
  \forall x,\ \forall i,\quad x \in T_i \iff \mathrm{rank}(x) \leq i.
\]
\end{lemma}

\begin{proof}
\begin{itemize}
  \item[$(\Rightarrow)$] $x \in T_i$ ならば rank の最小性(\texttt{rank\_le})より
        $\mathrm{rank}(x) \leq i$。
  \item[$(\Leftarrow)$] $\mathrm{rank}(x) \leq i$ ならば $x \in T_{\mathrm{rank}(x)}$(\texttt{rank\_spec})
        と単調性より $x \in T_i$。
\end{itemize}
\end{proof}

\begin{theorem}[Rank uniqueness(Theorem B), L4-3d]
\label{thm:rank-unique}
$T : \mathrm{ExhaustiveTower}\ \mathbb{N}\ \alpha$ に対し、
HasCharRank $T\ r$ を満たす $r : \alpha \to \mathbb{N}$ は一意である:
\[
  \mathrm{HasCharRank}\ T\ r \implies r = T.\mathrm{rank}.
\]
\end{theorem}

\begin{proof}
任意の $x$ に対して $r(x) = \mathrm{rank}(x)$ を、$\mathbb{N}$ の反対称性を使って示す。

\medskip
\noindent
\textbf{$r(x) \leq \mathrm{rank}(x)$}:
\texttt{rank\_spec} より $x \in T_{\mathrm{rank}(x)}$ であり、
HasCharRank の $\Rightarrow$ 方向 $(\mathrm{hchar}\ x\ (\mathrm{rank}\ x)).1$ を適用。

\medskip
\noindent
\textbf{$\mathrm{rank}(x) \leq r(x)$}:
HasCharRank の $\Leftarrow$ 方向で $r(x) \leq r(x)$($\mathrm{le\_refl}$)から
$x \in T_{r(x)}$ を得て、rank の最小性 $\mathrm{rank\_le}$ を適用。

\begin{leancode}
theorem rank_unique (T : ExhaustiveTower N a)
    (r : a -> N) (hchar : HasCharRank T r) : r = T.rank := by
  funext x
  apply Nat.le_antisymm
  · exact (hchar x (T.rank x)).1 (T.rank_spec x)
  · exact T.rank_le x (r x) ((hchar x (r x)).2 le_rfl)
\end{leancode}
\end{proof}

\begin{corollary}[Iic-表示, L4-3e]
HasCharRank $T\ r$ のとき、
\[
  T_i = \{x \in \alpha \mid r(x) \leq i\}
  \quad \bigl(= r^{-1}(\mathrm{Iic}(i))\bigr).
\]
\end{corollary}

\begin{remark}[前順序では一意性が崩れる, L4-3f]
添字集合 $\iota$ が前順序(PartialOrder でない)の場合、
rank の一意性は一般に成立しない。

Lean では $\iota = \mathrm{Bool}$(全ての元が $\leq$-同値な「離散前順序」
\texttt{indiscreteBoolPreorder})上の定数塔
$T_b := \alpha$ を反例として構成した:
$r_1 \equiv \mathrm{false}$ と $r_2 \equiv \mathrm{true}$ の両方が HasCharRank を満たすが、
$r_1 \neq r_2$ である。

直感的には、前順序では「$r(x) \leq b$」が全ての $b \in \mathrm{Bool}$ で
恒に真(または偽)になりうるため、$r(x)$ の値が区別できなくなる。
PartialOrder の反対称性が一意性に本質的に必要である。
\end{remark}

% =============================================================================
\section{ClosedTower の圏と反射的部分圏}
\label{sec:cat}
% =============================================================================

\subsection{充満部分圏としての ClosedTower}

$\mathrm{ClosedTower}_{\mathrm{cl}}$ は $\mathrm{StructureTower}$ の\textbf{充満部分圏}をなす:
\begin{itemize}
  \item \textbf{対象}:$\mathrm{ClosedTower}\ \mathrm{cl}\ \iota$
  \item \textbf{射}:$\mathrm{StructureTower.Hom}$ をそのまま制限(追加条件なし)
\end{itemize}

「充満」とは、$T_1,\ T_2 \in \mathrm{ClosedTower}$ のとき
$\mathrm{Hom}(T_1, T_2)$ として \texttt{StructureTower.Hom} の全体をそのまま使える、
ということであり、Lean では恒等関数 $\mathrm{homRestrict}(f) := f$(L4-4b)で確認できる。

\subsection{liftCl は ClosedTower を生成する}

\begin{proposition}[\texttt{liftCl\_closedTower}, L4-4a]
\label{prop:liftcl-closed}
任意の $T : \mathrm{StructureTower}\ \iota\ \alpha$ に対し、
$\mathrm{liftCl}\ \mathrm{cl}\ T$ は \texttt{ClosedTower} である。
\end{proposition}

\begin{proof}
各 $i$ で
$\mathrm{cl}((\mathrm{liftCl}\ \mathrm{cl}\ T)_i)
= \mathrm{cl}(\mathrm{cl}(T_i))
= \mathrm{cl}(T_i)$(冪等性 \texttt{cl.idempotent})より。

\begin{leancode}
def liftCl_closedTower (T : StructureTower i a) :
    ClosedTower cl i where
  toStructureTower := liftCl cl T
  level_closed := fun i => cl.idempotent (T.level i)
\end{leancode}
\end{proof}

\subsection{unit の普遍性(Reflector)}

\begin{theorem}[unit の普遍性(存在), L4-4c]
\label{thm:unit-universal}
$T : \mathrm{StructureTower}\ \iota\ \alpha$、
$S : \mathrm{ClosedTower}\ \mathrm{cl}\ \iota$、
$f : \mathrm{Hom}(T, S)$($\mathrm{toFun} = \mathrm{id}$ と仮定)とする。
このとき、射 $\bar{f} : \mathrm{Hom}(\mathrm{liftCl}\ \mathrm{cl}\ T,\ S)$ であって
\[
  \bar{f} \circ \mathrm{unit}_T = f, \quad \bar{f}.\mathrm{toFun} = \mathrm{id}
\]
を満たすものが存在する。
\end{theorem}

\begin{proof}
$\bar{f}.\mathrm{toFun} := \mathrm{id}$ とする。
$x \in \mathrm{cl}(T_i)$ のとき:
\begin{align*}
  \mathrm{cl}(T_i)
  &\subseteq \mathrm{cl}(S_i)
    && \text{($f$ の preserves と $\mathrm{cl}$ の単調性)} \\
  \mathrm{cl}(S_i)
  &= S_i
    && \text{($S$ が ClosedTower: \texttt{S.level\_closed i})}
\end{align*}
より $x \in S_i$。

合成条件 $\bar{f} \circ \mathrm{unit}_T = f$ は
$\mathrm{toFun}$ がいずれも $\mathrm{id}$ であることから
$\mathrm{Hom.ext}$ で確認できる。
\end{proof}

\begin{theorem}[unit の普遍性(一意性), L4-4d]
定理~\ref{thm:unit-universal} における $\bar{f}$ は一意である。
\end{theorem}

\begin{proof}
$\bar{f}_1,\ \bar{f}_2$ がいずれも $\mathrm{toFun} = \mathrm{id}$ を満たすならば、
$\mathrm{Hom.ext\ (by\ rw\ [hf_1,\ hf_2])}$ で $\bar{f}_1 = \bar{f}_2$。
\end{proof}

\begin{figure}[H]
\centering
\begin{tikzcd}[column sep=4cm, row sep=3cm]
  T
    \arrow[r, "\mathrm{unit}_T"]
    \arrow[dr, "f"', bend right=20]
  & \mathrm{liftCl}\ \mathrm{cl}\ T
    \arrow[d, "\bar{f}", dashed, "{\exists !}"']
  \\
  & S \quad (S \in \mathrm{ClosedTower}_{\mathrm{cl}})
\end{tikzcd}
\caption{unit の普遍性(reflector の図式, 定理~\ref{thm:unit-universal})。
$\mathrm{ClosedTower}$ への任意の射 $f$ は、
$\bar{f} \circ \mathrm{unit}_T = f$ を満たす一意な射 $\bar{f}$ に持ち上がる。
これは $\mathrm{ClosedTower}$ が $\mathrm{StructureTower}$ の反射的部分圏をなすことを意味する。}
\label{fig:reflector}
\end{figure}

\begin{mathinsight}
この普遍性は圏論的な \textbf{反射的部分圏}(reflective subcategory)の定義そのものである。
$\mathrm{liftCl}\ \mathrm{cl}$ は包含関手 $\iota : \mathrm{ClosedTower} \hookrightarrow \mathrm{StructureTower}$
の左随伴として振る舞い、unit はその単位(unit of adjunction)に対応する:
\[
  \mathrm{liftCl}\ \mathrm{cl} \dashv \iota.
\]
ただし本証明では $\mathrm{toFun} = \mathrm{id}$ の制限下のみを扱った。
$\mathrm{toFun} \neq \mathrm{id}$ の一般的な Kleisli 合成(自然性条件の公理化)は
Level 5 以降の課題である。
\end{mathinsight}

% =============================================================================
\section{全体のまとめ}
\label{sec:summary}
% =============================================================================

\begin{figure}[H]
\centering
\begin{tikzpicture}[
  sbox/.style={draw, rounded corners=5pt, fill=definitioncolor,
               minimum width=4.2cm, minimum height=1.3cm,
               align=center, font=\small\sffamily},
  cbox/.style={draw, rounded corners=5pt, fill=theoremcolor,
               minimum width=4.2cm, minimum height=1.3cm,
               align=center, font=\small\sffamily},
  rbox/.style={draw, rounded corners=5pt, fill=remarkcolor,
               minimum width=4.2cm, minimum height=1.3cm,
               align=center, font=\small\sffamily},
  arr/.style={-{Stealth[length=7pt]}, thick, accentcolor},
  darr/.style={-{Stealth[length=7pt]}, thick, accentcolor, dashed},
  label/.style={font=\footnotesize\sffamily, fill=white, inner sep=2pt},
]
  \node[sbox] (ST) at (0,6)
    {StructureTower\\$T = (T_i)_{i \in \iota}$};

  \node[cbox] (CT) at (-3,3)
    {ClosedTower cl\\$\mathrm{cl}(T_i) = T_i$};

  \node[rbox] (liftT) at (3,3)
    {liftCl cl T\\$(\mathrm{cl}(T_i))_{i \in \iota}$};

  \node[cbox] (MT) at (-3,0)
    {MeasurableTower\\$T_i \in \mathcal{M}$};

  \draw[arr] (ST) -- node[label, left]{$\mathrm{unit}_T$} (CT);
  \draw[arr] (ST) -- node[label, right]{liftCl cl} (liftT);
  \draw[arr] (liftT) -- node[label, right]{命題~\ref{prop:liftcl-closed}} (CT);
  \draw[darr] (ST) -- node[label, left]{測度論的接地} (MT);

  \node[font=\footnotesize\sffamily, align=center, text=accentcolor] at (0,1.5)
    {普遍性(定理~\ref{thm:unit-universal}):\\
     $f : T \to S\ (S \in \mathrm{ClosedTower})$\\
     $\Longrightarrow \exists!\,\bar{f} : \mathrm{liftCl}\ T \to S$};
\end{tikzpicture}
\caption{Level 4 の全体像。liftCl が ClosedTower を生成し(冪等性)、
unit が反射的部分圏の普遍性を与える。MeasurableTower は ClosureOperator を経由せずに
第3の接地として追加される。}
\label{fig:summary}
\end{figure}

Level 4 で達成した内容を整理する:

\begin{description}
  \item[\S\ref{sec:clparametric}\ \textbf{cl-parametric 比較}]
    閉包演算子の順序 $\mathrm{cl}_1 \leq \mathrm{cl}_2$ が
    liftCl 間の自然な比較射を誘導することを確認した(L4-1a)。
    推移性(L4-1b)、弱化(L4-1c)、吸収条件(L4-1d)を証明し、
    冪等性を吸収の特殊ケースとして再発見した(L4-1e)。

  \item[\S\ref{sec:sigma}\ \textbf{$\sigma$-代数の接地}]
    \texttt{MeasurableTower} を \texttt{ClosureOperator} を経由せずに定義し、
    交叉・補集合の可測性(L4-2c/d)と global の可測性(L4-2e)を確認した。
    位相・群論・測度論の3分野で「global の閉性定理」が成立することを示した
    (定理~\ref{thm:three-fields})。

  \item[\S\ref{sec:rank}\ \textbf{Rank uniqueness}]
    HasCharRank $T\ r \Rightarrow r = T.\mathrm{rank}$(定理~\ref{thm:rank-unique})を証明した。
    rank は常に HasCharRank を満たし(補題~\ref{lem:rank-char})、かつ一意である。
    前順序では一意性が崩れることを Bool 上の反例で確認した(L4-3f)。

  \item[\S\ref{sec:cat}\ \textbf{ClosedTower の圏}]
    liftCl cl T は冪等性により ClosedTower であり(命題~\ref{prop:liftcl-closed})、
    unit : $T \to \mathrm{liftCl}\ \mathrm{cl}\ T$ が reflector の普遍性を持つことを証明した
    (定理~\ref{thm:unit-universal})。
    これにより $\mathrm{ClosedTower}$ が $\mathrm{StructureTower}$ の
    反射的部分圏をなすことが示された。
\end{description}

\subsection*{達成状況}

\begin{description}[leftmargin=10em, style=nextline]
  \item[非自明な主定理 3本以上]
    \texttt{rank\_unique}(L4-3d),
    \texttt{unit\_universal\_id}(L4-4c),
    3分野統合(L4-2f),
    EM代数 $\iff$ ClosedTower(L3 M6e)
  \item[ケーススタディ 3分野以上]
    位相空間論・群論・測度論・順序論
  \item[再利用可能なライブラリ]
    cl-parametric 比較,ClosedTower の圏構造
\end{description}

\subsection*{次のステップ(Level 5 以降)}
\begin{itemize}
  \item $\mathrm{toFun} \neq \mathrm{id}$ の一般 Kleisli 合成(naturality 条件の公理化)
  \item Mathlib \texttt{CategoryTheory.Monad} との正式接続
  \item I-adic filtration: \texttt{FilteredRing} + \texttt{ClosedTower} の統合
  \item Enriched hom から 2-圏的構造へ
\end{itemize}

% =============================================================================
\end{document}
