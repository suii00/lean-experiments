\documentclass[a4paper,11pt]{ltjsarticle}
\usepackage{amsmath,amssymb,amsthm}
\usepackage{longtable,booktabs,array}
\usepackage[hidelinks]{hyperref}

\title{Lean 4 $\leftrightarrow$ \TeX\ 対応表\\
{\normalsize OrderNotes プロジェクト}}
\author{su}
\date{}

\begin{document}

\maketitle

\begin{center}
\small
\textit{AI assistance disclosure:}
Lean ソースコードは Claude (Anthropic) で骨格を生成し、
Codex (OpenAI) で修正した。
\TeX 文書は Claude Code (Anthropic) で生成した。
著者による加筆・修正は行っていない。
内容の正確性は保証されず、誤りがあれば著者の責任である。
\end{center}

\begin{abstract}
本表は \texttt{src/MyProjects/OrderNotes/} 以下の Lean 4 ソースファイルと、
\texttt{TeX/OrderNotes/} 以下の \TeX ファイルの対応関係を一覧にしたものである。
Lean の定義・定理名から数学的記述、および対応する \TeX 文書の箇所を速引きするために用いる。
\end{abstract}

\tableofcontents

%% =====================================================================
\section{ファイル対応}
%% =====================================================================

\begin{longtable}{lll}
\toprule
\textbf{Lean ソース} & \textbf{\TeX 文書} & \textbf{スタイル} \\
\midrule
\endfirsthead
\toprule
\textbf{Lean ソース} & \textbf{\TeX 文書} & \textbf{スタイル} \\
\midrule
\endhead
\texttt{ZornsLemma.lean}      & \texttt{ZornsLemma.tex}      & Bourbaki 独自定義 \\
\texttt{ZornsLemmaMathlib.lean}& \texttt{ZornsLemmaMathlib.tex}& Mathlib 直乗り   \\
\bottomrule
\end{longtable}

%% =====================================================================
\section{ZornsLemma.lean(Bourbaki 版)}
%% =====================================================================

\subsection{クラス・型}

\begin{longtable}{>{\ttfamily}l l p{6cm}}
\toprule
\textrm{\textbf{Lean 名}} & \textbf{数学的概念} & \textbf{\TeX 記述} \\
\midrule
\endfirsthead
\toprule
\textrm{\textbf{Lean 名}} & \textbf{数学的概念} & \textbf{\TeX 記述} \\
\midrule
\endhead
Ordre           & 半順序      & 型 $\alpha$ 上の $\le$ が反射・推移・反対称を満たす \\
OrdreTotal      & 全順序      & \texttt{Ordre} $+$ $\forall a,b,\; a \le b \lor b \le a$ \\
\bottomrule
\end{longtable}

\subsection{定義}

\begin{longtable}{>{\ttfamily}l l p{6cm}}
\toprule
\textrm{\textbf{Lean 名}} & \textbf{数学的概念} & \textbf{\TeX 記述} \\
\midrule
\endfirsthead
\toprule
\textrm{\textbf{Lean 名}} & \textbf{数学的概念} & \textbf{\TeX 記述} \\
\midrule
\endhead
EstChaine       & 鎖(チェーン) &
  $C$ が鎖 $\overset{\mathrm{def}}{\iff}$
  $\forall a,b \in C,\; a \le b \lor b \le a$ \\
EstMajorant     & 上界       &
  $x$ が $S$ の上界 $\overset{\mathrm{def}}{\iff}$
  $\forall a \in S,\; a \le x$ \\
EstMaximal      & 極大元      &
  $x$ が $S$ の極大元 $\overset{\mathrm{def}}{\iff}$
  $x \in S \land (\forall y \in S,\; x \le y \implies x = y)$ \\
EstInductif     & 帰納的集合   &
  $S$ が帰納的 $\overset{\mathrm{def}}{\iff}$
  $\forall C \subseteq S$(鎖),$\exists b \in S$($C$ の上界) \\
AxiomeDeChoix   & 選択公理    &
  $\forall (\iota:\text{Type})\;(A:\iota \to \text{Type}),\;
  (\forall i,\;\text{Nonempty}(A_i))
  \implies \text{Nonempty}(\prod_i A_i)$ \\
\bottomrule
\end{longtable}

\subsection{インスタンス・補題}

\begin{longtable}{>{\ttfamily}l l p{6cm}}
\toprule
\textrm{\textbf{Lean 名}} & \textbf{種別} & \textbf{\TeX 記述} \\
\midrule
\endfirsthead
\toprule
\textrm{\textbf{Lean 名}} & \textbf{種別} & \textbf{\TeX 記述} \\
\midrule
\endhead
ordreToPartialOrder &
  instance &
  \texttt{Ordre} $\to$ Mathlib の \texttt{PartialOrder} への変換 \\
estChaine\_iff\_isChain &
  lemma &
  $\texttt{EstChaine}\;C \iff \texttt{IsChain}\;(\le)\;C$ \\
\bottomrule
\end{longtable}

\subsection{定理}

\begin{longtable}{>{\ttfamily}l l p{6cm}}
\toprule
\textrm{\textbf{Lean 名}} & \textbf{数学的概念} & \textbf{\TeX 記述} \\
\midrule
\endfirsthead
\toprule
\textrm{\textbf{Lean 名}} & \textbf{数学的概念} & \textbf{\TeX 記述} \\
\midrule
\endhead
zorn &
  Zorn の補題(局所版) &
  $S$ が帰納的 $\implies \exists m,\;\texttt{EstMaximal}\;S\;m$ \\
zorn\_global &
  Zorn の補題(大域版) &
  すべての鎖が上界を持つ $\implies \exists m,\;\forall x,\; m \le x \implies x \le m$ \\
axiomeDeChoix\_classical &
  AC の証明 &
  \texttt{Classical.choice} から \texttt{AxiomeDeChoix} を導出 \\
choix\_implique\_zorn &
  AC $\implies$ Zorn &
  選択公理からツォルンの補題を導く \\
zorn\_implique\_choix &
  Zorn $\implies$ AC &
  ツォルンの補題から選択公理を導く \\
equivalence\_choix\_zorn &
  AC $\iff$ Zorn &
  選択公理とツォルンの補題の同値性 \\
\bottomrule
\end{longtable}

%% =====================================================================
\section{ZornsLemmaMathlib.lean(Mathlib 版)}
%% =====================================================================

\subsection{§1 ツォルンの補題のバリエーション}

\begin{longtable}{>{\ttfamily}l l p{6cm}}
\toprule
\textrm{\textbf{Lean 名}} & \textbf{数学的概念} & \textbf{\TeX 記述} \\
\midrule
\endfirsthead
\toprule
\textrm{\textbf{Lean 名}} & \textbf{数学的概念} & \textbf{\TeX 記述} \\
\midrule
\endhead
zorn\_subset\_version &
  Zorn(部分集合版) &
  $\forall C \subseteq S$(鎖)に上界 $\implies
  \exists m \in S,\;\forall x \in S,\; m \le x \implies m = x$ \\
zorn\_total\_version &
  Zorn(全体版) &
  $\forall C$(鎖)が \texttt{BddAbove} $\implies
  \exists m,\;\forall x,\; m \le x \implies x \le m$ \\
zorn\_nonempty &
  Zorn(空チェーン対応版) &
  $S \neq \emptyset$ かつ空でない鎖に上界 $\implies$
  極大元が存在 \\
\bottomrule
\end{longtable}

\subsection{§2 選択公理との同値性}

\begin{longtable}{>{\ttfamily}l l p{6cm}}
\toprule
\textrm{\textbf{Lean 名}} & \textbf{数学的概念} & \textbf{\TeX 記述} \\
\midrule
\endfirsthead
\toprule
\textrm{\textbf{Lean 名}} & \textbf{数学的概念} & \textbf{\TeX 記述} \\
\midrule
\endhead
AC &
  選択公理(型理論版) &
  $\forall i,\;\text{Nonempty}(A_i) \implies
  \text{Nonempty}(\prod_i A_i)$ \\
AC\_Set &
  選択公理(集合論版) &
  $\forall i,\;S_i \neq \emptyset \implies
  \exists f,\;\forall i,\; f(i) \in S_i$ \\
ac\_from\_classical &
  AC の証明 &
  \texttt{Classical.choice} $\implies$ AC \\
ac\_implies\_zorn &
  AC $\implies$ Zorn &
  AC から \texttt{zorn\_subset\_version} を導出 \\
zorn\_implies\_ac\_set &
  Zorn $\implies$ AC(集合論版) &
  Zorn 仮定下で \texttt{AC\_Set} を証明 \\
zorn\_implies\_ac &
  Zorn $\implies$ AC(型理論版) &
  Zorn 仮定下で \texttt{AC} を証明 \\
ac\_iff\_zorn &
  AC $\iff$ Zorn(主定理) &
  $\texttt{AC} \iff \texttt{zorn\_subset\_version}$ \\
\bottomrule
\end{longtable}

\subsection{§3 整列定理との関係}

\begin{longtable}{>{\ttfamily}l l p{6cm}}
\toprule
\textrm{\textbf{Lean 名}} & \textbf{数学的概念} & \textbf{\TeX 記述} \\
\midrule
\endfirsthead
\toprule
\textrm{\textbf{Lean 名}} & \textbf{数学的概念} & \textbf{\TeX 記述} \\
\midrule
\endhead
WellOrderOn &
  整列順序 &
  $\{r : \alpha \to \alpha \to \text{Prop} \mid \text{IsWellOrder}\;\alpha\;r\}$ \\
WellOrderingTheorem &
  整列定理 &
  $\forall \alpha,\;\text{Nonempty}(\texttt{WellOrderOn}\;\alpha)$ \\
ac\_implies\_well\_ordering &
  AC $\implies$ 整列定理 &
  \texttt{WellOrderingRel} による構成 \\
well\_ordering\_implies\_ac &
  整列定理 $\implies$ AC &
  \texttt{ac\_from\_classical} に帰着 \\
\bottomrule
\end{longtable}

\subsection{§4 典型的応用}

\begin{longtable}{>{\ttfamily}l l p{6cm}}
\toprule
\textrm{\textbf{Lean 名}} & \textbf{数学的概念} & \textbf{\TeX 記述} \\
\midrule
\endfirsthead
\toprule
\textrm{\textbf{Lean 名}} & \textbf{数学的概念} & \textbf{\TeX 記述} \\
\midrule
\endhead
Inductive &
  帰納的集合 &
  $\forall C \subseteq S$(鎖),$\exists b \in S,\;\forall a \in C,\;a \le b$ \\
inductive\_has\_maximal &
  帰納的 $\implies$ 極大元 &
  \texttt{Inductive}\;$S \implies \exists m \in S$(極大元) \\
maximal\_in\_family &
  集合族の極大元 &
  包含帰納的 $+$ 鎖の和集合閉 $\implies$ $\exists M \in \mathcal{S}$($\subseteq$-極大) \\
maximal\_partial\_map\_extension &
  写像の極大拡張 &
  性質 $P$ を保つ鎖条件 $\implies$ $\exists (T, f)$(極大部分写像) \\
ultrafilter\_extension\_pattern &
  超フィルター拡張 &
  真フィルター $F$ $\implies$ $\exists U \supseteq F$(極大フィルター) \\
\bottomrule
\end{longtable}

%% =====================================================================
\section{Mathlib API 対応}
%% =====================================================================

本プロジェクトで用いている主要な Mathlib の型・定理との対応を示す。

\begin{longtable}{>{\ttfamily}l >{\ttfamily}l l}
\toprule
\textrm{\textbf{独自定義 (Bourbaki)}} & \textrm{\textbf{Mathlib}} & \textbf{関係} \\
\midrule
\endfirsthead
\toprule
\textrm{\textbf{独自定義 (Bourbaki)}} & \textrm{\textbf{Mathlib}} & \textbf{関係} \\
\midrule
\endhead
Ordre             & PartialOrder     & \texttt{ordreToPartialOrder} で変換 \\
OrdreTotal        & LinearOrder      & 概念的に対応 \\
EstChaine         & IsChain          & \texttt{estChaine\_iff\_isChain} で同値 \\
EstMajorant       & BddAbove (上界)   & 概念的に対応 \\
EstMaximal        & (zorn\_le₀ の結論) & 反対称律で一致 \\
EstInductif       & (zorn\_le₀ の仮定) & 鎖条件として同等 \\
AxiomeDeChoix     & Classical.choice & 体系に組込み済 \\
\textrm{---}      & zorn\_le₀        & 部分集合版 Zorn \\
\textrm{---}      & zorn\_le         & 全体版 Zorn \\
\textrm{---}      & WellOrderingRel  & 整列関係の構成 \\
\bottomrule
\end{longtable}

%% =====================================================================
\section{記法対応(Lean 構文 $\leftrightarrow$ 数学記法)}
%% =====================================================================

\begin{longtable}{>{\ttfamily}l l}
\toprule
\textrm{\textbf{Lean 構文}} & \textbf{数学記法(\TeX)} \\
\midrule
\endfirsthead
\toprule
\textrm{\textbf{Lean 構文}} & \textbf{数学記法(\TeX)} \\
\midrule
\endhead
a ≤ b                    & $a \le b$ \\
a ∈ C                    & $a \in C$ \\
C ⊆ S                    & $C \subseteq S$ \\
∀ a, P a                 & $\forall a,\; P(a)$ \\
∃ a, P a                 & $\exists a,\; P(a)$ \\
a ∨ b                    & $a \lor b$ \\
a ∧ b                    & $a \land b$ \\
a → b                    & $a \implies b$ \\
a ↔ b                    & $a \iff b$ \\
Set α                    & $\mathcal{P}(\alpha)$ \\
Set.univ                 & 全体集合 \\
⋃₀ C                     & $\bigcup \mathcal{C}$($= \bigcup_{A \in \mathcal{C}} A$) \\
A ∩ B                    & $A \cap B$ \\
∅                        & $\emptyset$ \\
Nonempty α               & $\alpha \neq \emptyset$ \\
Set.Nonempty S           & $S \neq \emptyset$ \\
BddAbove C               & $C$ は上に有界 \\
Subtype \{x | P x\}      & $\{x \mid P(x)\}$ \\
fun i => A i             & $i \mapsto A_i$ \\
(∀ i, A i)               & $\prod_i A_i$ \\
\bottomrule
\end{longtable}

%% =====================================================================
\section{セクション名対応}
%% =====================================================================

Lean ソースのセクション名と \TeX 文書の節タイトルの対応を示す。

\subsection{ZornsLemma.lean $\leftrightarrow$ ZornsLemma.tex}

\begin{longtable}{>{\ttfamily}l l}
\toprule
\textrm{\textbf{Lean セクション}} & \textbf{\TeX 節} \\
\midrule
\endfirsthead
\toprule
\textrm{\textbf{Lean セクション}} & \textbf{\TeX 節} \\
\midrule
\endhead
Definitions & §2 順序に関する基本的定義 \\
Bridge      & §3 Mathlibとの連携 \\
Zorn        & §4 ツォルンの補題 \\
Choice      & §5 選択公理との同値性 \\
\bottomrule
\end{longtable}

\subsection{ZornsLemmaMathlib.lean $\leftrightarrow$ ZornsLemmaMathlib.tex}

\begin{longtable}{>{\ttfamily}l l}
\toprule
\textrm{\textbf{Lean セクション}} & \textbf{\TeX 節} \\
\midrule
\endfirsthead
\toprule
\textrm{\textbf{Lean セクション}} & \textbf{\TeX 節} \\
\midrule
\endhead
ZornVariants    & §2 ツォルンの補題のバリエーション \\
ACEquivalence   & §3 選択公理との同値性 \\
WellOrdering    & §4 整列定理との関係 \\
Applications    & §5 ツォルンの補題の典型的応用 \\
\bottomrule
\end{longtable}

\end{document}
