% =============================================================================
%  StructureTower の主定理と接続定理
%  LuaLaTeX 文書
% =============================================================================
\documentclass[a4paper,11pt]{ltjsarticle}

%% ─── パッケージ ─────────────────────────────────────────────────────────────
\usepackage{luatexja-fontspec}
\usepackage{amsmath,amssymb,amsthm,mathtools}
\usepackage{tikz-cd}
\usepackage{tikz}
\usetikzlibrary{arrows.meta,positioning,calc,decorations.pathmorphing}
\usepackage{enumitem}
\usepackage{hyperref}
\usepackage{tcolorbox}
\tcbuselibrary{skins,breakable}
\usepackage{listings}
\usepackage{xcolor}
\usepackage{geometry}
\usepackage{fancyhdr}
\usepackage{float}
\usepackage{bm}

%% ─── ページ設定 ──────────────────────────────────────────────────────────────
\geometry{margin=2.5cm, top=3cm, bottom=3cm}

%% ─── フォント ────────────────────────────────────────────────────────────────
% TeXLive 付属の原ノ味フォント(Harano Aji)を使用
\setmainjfont{Harano Aji Mincho}[
  BoldFont = Harano Aji Mincho,
  BoldFeatures = {FakeBold=2},
]
\setsansjfont{Harano Aji Gothic}
% モノスペースフォント(DejaVu Sans Mono は広い Unicode 範囲をカバー)
\setmonofont[Scale=0.85]{DejaVu Sans Mono}

%% ─── 色 ──────────────────────────────────────────────────────────────────────
\definecolor{leanblue}{HTML}{2563EB}
\definecolor{leanbg}{HTML}{F8FAFC}
\definecolor{leancomment}{HTML}{6B7280}
\definecolor{leankeyword}{HTML}{7C3AED}
\definecolor{leanstring}{HTML}{059669}
\definecolor{accentcolor}{HTML}{1E40AF}
\definecolor{theoremcolor}{HTML}{EFF6FF}
\definecolor{definitioncolor}{HTML}{F0FDF4}
\definecolor{remarkcolor}{HTML}{FFF7ED}

%% ─── hyperref ────────────────────────────────────────────────────────────────
\hypersetup{
  colorlinks=true,
  linkcolor=accentcolor,
  urlcolor=leanblue,
  citecolor=accentcolor,
  pdftitle={StructureTower の主定理と接続定理},
  pdfauthor={su},
}

%% ─── 定理環境 ────────────────────────────────────────────────────────────────
\theoremstyle{definition}
\newtheorem{definition}{定義}[section]
\newtheorem{example}[definition]{例}

\theoremstyle{plain}
\newtheorem{theorem}[definition]{定理}
\newtheorem{lemma}[definition]{補題}
\newtheorem{proposition}[definition]{命題}
\newtheorem{corollary}[definition]{系}

\theoremstyle{remark}
\newtheorem{remark}[definition]{注意}

%% ─── tcolorbox スタイル ───────────────────────────────────────────────────────
\newtcolorbox{proofstrategy}{
  colback=remarkcolor,
  colframe=orange!60!black,
  title={\textbf{証明戦略}},
  fonttitle=\sffamily,
  boxrule=0.5pt,
  arc=3pt,
}

\newtcolorbox{mathinsight}{
  colback=theoremcolor,
  colframe=accentcolor,
  title={\textbf{数学的洞察}},
  fonttitle=\sffamily,
  boxrule=0.5pt,
  arc=3pt,
}

%% ─── Lean コードリスト ────────────────────────────────────────────────────────
\lstdefinelanguage{lean4}{
  morekeywords={def,theorem,lemma,example,instance,class,structure,
    where,let,in,do,if,then,else,match,with,fun,return,
    import,open,namespace,section,end,variable,
    inductive,abbrev,noncomputable,private,protected,
    sorry,admit,have,show,suffices,calc,by,exact,
    apply,intro,intros,constructor,cases,induction,
    simp,rw,rfl,ext,ring,linarith,omega,norm_num,
    decide,trivial,assumption,contradiction,classical,
    refine,rcases,obtain,use,push_neg,funext,
    Type,Prop,Sort,Set,true,false,
    attribute,deriving},
  sensitive=true,
  morecomment=[l]{--},
  morecomment=[n]{/-}{-/},
  morestring=[b]",
  literate=
    % 長いマッチを先に(×ˢ は × の前に置く)
    {×ˢ}{{\(\times\)}}2
    % ギリシャ文字
    {α}{{\(\alpha\)}}1
    {β}{{\(\beta\)}}1
    {γ}{{\(\gamma\)}}1
    {ι}{{\(\iota\)}}1
    {κ}{{\(\kappa\)}}1
    {μ}{{\(\mu\)}}1
    {σ}{{\(\sigma\)}}1
    {φ}{{\(\varphi\)}}1
    % 矢印
    {→}{{\(\to\)}}1
    {←}{{\(\leftarrow\)}}1
    {↔}{{\(\leftrightarrow\)}}1
    % 括弧
    {⟨}{{\(\langle\)}}1
    {⟩}{{\(\rangle\)}}1
    {⦃}{{\(\{\)}}1
    {⦄}{{\(\}\)}}1
    % 比較・論理
    {≤}{{\(\leq\)}}1
    {≥}{{\(\geq\)}}1
    {≠}{{\(\neq\)}}1
    {∀}{{\(\forall\)}}1
    {∃}{{\(\exists\)}}1
    {∧}{{\(\wedge\)}}1
    {∨}{{\(\vee\)}}1
    {¬}{{\(\neg\)}}1
    {⊢}{{\(\vdash\)}}1
    % 集合
    {∈}{{\(\in\)}}1
    {∉}{{\(\notin\)}}1
    {⊆}{{\(\subseteq\)}}1
    {⊂}{{\(\subset\)}}1
    {∪}{{\(\cup\)}}1
    {∩}{{\(\cap\)}}1
    {⋂}{{\(\bigcap\)}}1
    {∅}{{\(\emptyset\)}}1
    % 積・合成
    {×}{{\(\times\)}}1
    {∘}{{\(\circ\)}}1
    % 数域
    {ℕ}{{\(\mathbb{N}\)}}1
    {ℤ}{{\(\mathbb{Z}\)}}1
    {ℝ}{{\(\mathbb{R}\)}}1
    {ℚ}{{\(\mathbb{Q}\)}}1
    % 記号
    {:=}{{:=}}2,
}

\lstnewenvironment{leancode}[1][]{%
  \lstset{
    language=lean4,
    basicstyle=\ttfamily\small,
    keywordstyle=\color{leankeyword}\bfseries,
    commentstyle=\color{leancomment}\itshape,
    stringstyle=\color{leanstring},
    backgroundcolor=\color{leanbg},
    frame=single,
    rulecolor=\color{leanblue!30},
    framesep=8pt,
    xleftmargin=12pt,
    xrightmargin=12pt,
    breaklines=true,
    breakatwhitespace=true,
    showstringspaces=false,
    tabsize=2,
    captionpos=b,
    numbers=left,
    numberstyle=\tiny\color{leancomment},
    numbersep=8pt,
    aboveskip=1em,
    belowskip=1em,
    #1
  }%
}{}

%% ─── ヘッダー・フッター ───────────────────────────────────────────────────────
\pagestyle{fancy}
\fancyhf{}
\renewcommand{\headrulewidth}{0.4pt}
\fancyhead[L]{\small\sffamily\nouppercase{\leftmark}}
\fancyhead[R]{\small\sffamily StructureTower 形式化}
\fancyfoot[C]{\thepage}

%% ─── 便利マクロ ───────────────────────────────────────────────────────────────
\newcommand{\Tower}[2]{\mathbf{Tower}(#1,\,#2)}
\newcommand{\NatIncl}{\mathrm{NatIncl}}
\newcommand{\reindex}{\mathrm{reindex}}
\newcommand{\rank}{\mathrm{rank}}
\newcommand{\Fix}{\mathrm{Fix}}
\newcommand{\EMAlg}{\mathrm{EMAlg}}
\newcommand{\lev}{\mathrm{level}}
\newcommand{\iInf}{\mathsf{iInf}}

%% =============================================================================
\begin{document}

%% ─── タイトル ─────────────────────────────────────────────────────────────────
\title{%
  {\LARGE\sffamily\bfseries StructureTower の主定理と接続定理}\\[0.5em]
  {\large\sffamily Lean\,4 による形式化とその数学的解説}%
}
\author{su}
\date{\today}
\maketitle

%% ─── AI 開示 ──────────────────────────────────────────────────────────────────
\begin{center}
\small
\textit{AI assistance disclosure:}\\
Lean ソースコードは Claude (Anthropic) で骨格を生成し、
Codex (OpenAI) で修正した。\\
\TeX{}文書は Claude Code (Anthropic) で生成した。\\
著者による加筆・修正は行っていない。\\
内容の正確性は保証されず、誤りがあれば著者の責任である。
\end{center}

\vspace{1em}

%% ─── アブストラクト ────────────────────────────────────────────────────────────
\begin{abstract}
本稿では,前順序集合 $\iota$ 上の単調集合族
$(\lev(i))_{i \in \iota}$ を「構造塔(StructureTower)」として
Lean\,4 / Mathlib4 で形式化した3本の主定理と接続定理を数学的に解説する.

\textbf{定理A}(reindex の関手性)では,
単調写像 $f \colon \iota \to \kappa$ による添字引き戻し操作 $\reindex_f$ が,
NatInclusion(逐次包含)を射とする「構造塔の圏」の関手をなすことを証明する.

\textbf{定理B}(rank の一意性)では,
全充足塔(ExhaustiveTower)において
$\rank(x) = \min\{i \mid x \in \lev(i)\}$ が一意に定まり,
「強い単射公理」$x \in \lev(i) \Leftrightarrow r(x) \leq i$
を満たす関数 $r$ が $\rank$ に等しいことを示す.

\textbf{定理C}(EMAlgebras の完備束構造)では,
閉包作用素 $c$ の固定点集合(EM代数)が任意交叉で閉じており,
完備束の下半束をなすことを確立する.

さらに \textbf{接続定理}(Bridge)として,
次数付き環(FilteredRing)の乗法互換条件(mul\_mem 公理)が,
二項タワー(BinaryTower)から reindex タワーへの Hom と同値であり,
mapOnHom の $h_g$ 条件の代数的具現化であることを示す.
\end{abstract}

\tableofcontents
\newpage

%% =============================================================================
\section{導入}
%% =============================================================================

数学における「塔構造」とは,前順序集合 $\iota$ によって添字付けられた集合族
$(A_i)_{i \in \iota}$ であって,$i \leq j \Rightarrow A_i \subseteq A_j$
という単調性を満たすものである.
フィルトレーション・グレーデッドモジュール・閉包列など,
数学の各分野でこの概念が現れる.

本稿が解説するのは,次の二つの Lean\,4 ファイルである:
\begin{itemize}
  \item \texttt{StructureTower\_MainTheorems.lean}:3本の主定理(A, B, C)
  \item \texttt{StructureTower\_Bridge.lean}:代数的フィルトレーションとの接続
\end{itemize}

\begin{mathinsight}
なぜ単なる \texttt{OrderHom\;ι\;(Set\;α)}(前順序集合から冪集合への単調写像)ではなく,
\texttt{StructureTower} という構造体が必要か?
\begin{itemize}
  \item \textbf{定理A}:reindex の関手性は「自明な前合成」に見えるが,
        NatInclusion の圏での関手的性質は定理として明示的に述べる必要がある.
  \item \textbf{定理B}:rank(最小レベル)は OrderHom の言語では隠れた情報であり,
        StructureTower の「内部構造」として初めて定式化される.
  \item \textbf{定理C}:EM代数と iInf の相互作用は,集合の包含だけでは見えない
        塔固有の代数的構造を反映している.
\end{itemize}
\end{mathinsight}

%% =============================================================================
\section{基本定義}
%% =============================================================================

\subsection{構造塔(StructureTower)}

\begin{definition}[構造塔]
\label{def:structure-tower}
型 $\iota$(前順序 $\leq$ 付き)と型 $\alpha$ に対して,
\textbf{構造塔} $\mathsf{StructureTower}(\iota, \alpha)$ は以下のデータからなる:
\begin{itemize}
  \item 関数 $\lev \colon \iota \to \mathcal{P}(\alpha)$
  \item 単調性:$i \leq j \Rightarrow \lev(i) \subseteq \lev(j)$
\end{itemize}
\end{definition}

\begin{leancode}
structure StructureTower (ι α : Type*) [Preorder ι] : Type _ where
  level          : ι → Set α
  monotone_level : ∀ ⦃i j : ι⦄, i ≤ j → level i ⊆ level j
\end{leancode}

\subsection{NatInclusion と Hom}

\begin{definition}[NatInclusion]
\label{def:nat-inclusion}
同じ型 $\alpha$ 上の二つの構造塔
$T_1, T_2 \colon \mathsf{StructureTower}(\iota, \alpha)$ に対して,
\[
  \NatIncl(T_1, T_2)
  \;\stackrel{\mathrm{def}}{=}\;
  \forall i,\quad T_1.\lev(i) \subseteq T_2.\lev(i)
\]
を \textbf{NatInclusion}(自然変換的包含)という.
\end{definition}

\begin{definition}[Hom]
\label{def:hom}
$T_1 \colon \mathsf{StructureTower}(\iota, \alpha)$ と
$T_2 \colon \mathsf{StructureTower}(\iota, \beta)$ に対して,
\textbf{Hom} $T_1 \to T_2$ は以下のデータからなる:
\begin{itemize}
  \item 関数 $\phi \colon \alpha \to \beta$
  \item レベル保存:$\forall i,\; \phi(T_1.\lev(i)) \subseteq T_2.\lev(i)$
\end{itemize}
\end{definition}

\begin{remark}
NatInclusion は $\alpha = \beta$ かつ $\phi = \mathrm{id}$ のときの Hom の特殊例だが,
「塔間の順序」としての性格を際立たせるため独立に定義する.
\end{remark}

\subsection{reindex と iInf}

\begin{definition}[reindex]
\label{def:reindex}
単調写像 $f \colon \iota \to \kappa$ と塔
$T \colon \mathsf{StructureTower}(\kappa, \alpha)$ に対して,
\[
  (\reindex_f T).\lev(i)
  \;\stackrel{\mathrm{def}}{=}\;
  T.\lev(f(i))
\]
で $\reindex_f T \colon \mathsf{StructureTower}(\iota, \alpha)$ を定める.
単調性は $f$ の単調性と $T$ の単調性の合成から従う.
\end{definition}

\begin{definition}[iInf]
\label{def:iinf}
族 $\{T_s\}_{s \in \sigma}$(各 $T_s \colon \mathsf{StructureTower}(\iota, \alpha)$)に対して,
\[
  \Bigl(\bigwedge_{s \in \sigma} T_s\Bigr).\lev(i)
  \;\stackrel{\mathrm{def}}{=}\;
  \bigcap_{s \in \sigma} T_s.\lev(i)
\]
で下限塔 $\iInf\{T_s\}$ を定める.
\end{definition}

\begin{leancode}
def reindex {κ : Type*} [Preorder κ] (f : ι → κ) (hf : Monotone f)
    (T : StructureTower κ α) : StructureTower ι α where
  level i        := T.level (f i)
  monotone_level := fun _i _j hij => T.monotone_level (hf hij)

def iInf {σ : Type*} (T : σ → StructureTower ι α) : StructureTower ι α where
  level i        := ⋂ s, (T s).level i
  monotone_level := fun _i _j hij _x hx =>
    Set.mem_iInter.mpr
      (fun s => (T s).monotone_level hij (Set.mem_iInter.mp hx s))
\end{leancode}

\subsection{閉包作用素との接続}

\begin{definition}[towerOfClosure と EMAlgebras]
\label{def:tower-of-closure}
半順序集合 $\alpha$ 上の閉包作用素
$c \colon \alpha \to \alpha$(単調・拡大・冪等)に対して,
\begin{align*}
  \mathsf{towerOfClosure}(c).\lev(x)
    &\;\stackrel{\mathrm{def}}{=}\; \{y \in \alpha \mid y \leq c(x)\}
    = {\downarrow}\,c(x), \\
  \EMAlg(c)
    &\;\stackrel{\mathrm{def}}{=}\; \{x \in \alpha \mid c(x) \leq x\}
    = \Fix(c).
\end{align*}
$\EMAlg(c)$ を \textbf{EM代数}(閉元全体・固定点集合)という.
\end{definition}

\begin{remark}
$c$ が拡大的($x \leq c(x)$)であるから,
$c(x) \leq x$ は $c(x) = x$(固定点条件)と同値である.
\end{remark}

%% =============================================================================
\section{定理A: reindex の関手性}
%% =============================================================================

\subsection{圏論的設定}

型 $\alpha$ を固定する.
NatInclusion を射とする「構造塔の圏」$\Tower{\iota}{\alpha}$ を
\begin{itemize}
  \item \textbf{対象}:$\mathsf{StructureTower}(\iota, \alpha)$ の全元
  \item \textbf{射}:$\NatIncl(T_1, T_2)$(前順序圏,thin category)
\end{itemize}
と定める.
NatInclusion は反射性・推移性を持つので,これは適切に定義された圏である.

単調写像 $f \colon \iota \to \kappa$ が与えられると,
$\reindex_f \colon \Tower{\kappa}{\alpha} \to \Tower{\iota}{\alpha}$
という対象間の写像が定まる.定理Aは,これが圏の関手をなすことを主張する.

\subsection{主定理}

\begin{theorem}[定理A:reindex の関手性]
\label{thm:reindex-functor}
単調写像 $f \colon \iota \to \kappa$ に対して,
$\reindex_f \colon \Tower{\kappa}{\alpha} \to \Tower{\iota}{\alpha}$
は圏の関手をなす.すなわち:
\begin{enumerate}[label=\textup{(\roman*)}]
  \item \textbf{射の保存}:
        $\NatIncl(T_1, T_2) \Rightarrow \NatIncl(\reindex_f T_1, \reindex_f T_2)$
  \item \textbf{恒等の保存}:
        $\NatIncl(\reindex_f T, \reindex_f T)$(自明)
  \item \textbf{合成の保存}:
        $\NatIncl(T_1,T_2)$ かつ $\NatIncl(T_2,T_3)$
        $\Rightarrow \NatIncl(\reindex_f T_1, \reindex_f T_3)$
  \item \textbf{合成の可換性}:
        $\reindex_f \circ \reindex_g \simeq \reindex_{g \circ f}$
        (NatInclusion として等価)
\end{enumerate}
\end{theorem}

\begin{figure}[H]
  \centering
  \begin{tikzcd}[column sep=5em, row sep=3.5em]
    T_1 \arrow[r, "h"{name=top}, "\NatIncl"'] \arrow[d, "\reindex_f"'] &
    T_2 \arrow[d, "\reindex_f"] \\
    \reindex_f(T_1) \arrow[r, "\reindex_f(h)"', "\NatIncl"] &
    \reindex_f(T_2)
  \end{tikzcd}
  \caption{%
    定理A (i):$\reindex_f$ は NatInclusion を保存する(関手の射への作用).
    $h(f(i))$ という一行の証明に凝縮される.
  }
  \label{fig:reindex-functor}
\end{figure}

\begin{figure}[H]
  \centering
  \begin{tikzcd}[column sep=4em, row sep=3em]
    & \Tower{\kappa}{\alpha} \arrow[dr, "\reindex_f"] & \\
    \Tower{\mu}{\alpha}
      \arrow[ur, "\reindex_g"]
      \arrow[rr, "\reindex_{g \circ f}"', bend right=15, "\simeq" above=3pt]
    & & \Tower{\iota}{\alpha}
  \end{tikzcd}
  \caption{%
    定理A (iv):合成則 $\reindex_f \circ \reindex_g \simeq \reindex_{g \circ f}$.
    両方向の NatInclusion は $\Subset.\mathsf{refl}$ で得られる.
  }
  \label{fig:reindex-comp}
\end{figure}

\begin{proofstrategy}
\textbf{(i) の証明}:
$h \colon \NatIncl(T_1, T_2)$ は
$\forall i,\; T_1.\lev(i) \subseteq T_2.\lev(i)$ を意味する.
$(\reindex_f T_k).\lev(i) = T_k.\lev(f(i))$ であるから,
$\NatIncl(\reindex_f T_1, \reindex_f T_2)$ は
$\forall i,\; T_1.\lev(f(i)) \subseteq T_2.\lev(f(i))$
に帰着し,これは $h$ を $f(i)$ に適用すれば得られる:
\[
  \texttt{fun i \(\Rightarrow\) h (f i)}
\]

\textbf{(iv) の証明}:
$(\reindex_f \circ \reindex_g)(T).\lev(i)
  = T.\lev(g(f(i)))
  = \reindex_{g \circ f}(T).\lev(i)$
となり,両 NatInclusion は集合の等しさから自明に従う.
Lean では \texttt{simp [reindex]} で結論する.
\end{proofstrategy}

\begin{leancode}
-- A-1: reindex は NatInclusion を保存する
theorem reindex_preserves_natInclusion {κ : Type*} [Preorder κ]
    {T₁ T₂ : StructureTower κ α}
    (f : ι → κ) (hf : Monotone f)
    (h : NatInclusion T₁ T₂) :
    NatInclusion (reindex f hf T₁) (reindex f hf T₂) :=
  fun i => h (f i)   -- 証明の全体はこの一行

-- A-5: 合成則(両方向 NatInclusion)
theorem reindex_comp_natInclusion {κ μ : Type*} [Preorder κ] [Preorder μ]
    (f : ι → κ) (hf : Monotone f) (g : κ → μ) (hg : Monotone g)
    (T : StructureTower μ α) :
    NatInclusion (reindex f hf (reindex g hg T))
                 (reindex (g ∘ f) (hg.comp hf) T) ∧
    NatInclusion (reindex (g ∘ f) (hg.comp hf) T)
                 (reindex f hf (reindex g hg T)) := by
  constructor <;> (intro i; simp [reindex])
\end{leancode}

%% =============================================================================
\section{定理B: ExhaustiveTower における rank の一意性}
%% =============================================================================

\subsection{全充足塔(ExhaustiveTower)}

\begin{definition}[ExhaustiveTower]
\label{def:exhaustive-tower}
$\mathbb{N}$-添字付き構造塔 $T \colon \mathsf{StructureTower}(\mathbb{N}, \alpha)$ が
\textbf{全充足塔}(ExhaustiveTower)であるとは,
\[
  \forall x \in \alpha,\quad \exists i \in \mathbb{N},\quad x \in T.\lev(i)
\]
が成立することをいう.すなわち,すべての元はある有限レベルに収まる.
\end{definition}

\begin{leancode}
structure ExhaustiveTower (α : Type*) extends StructureTower ℕ α where
  exhaustive : ∀ x : α, ∃ i : ℕ, x ∈ level i
\end{leancode}

\subsection{rank の定義と基本性質}

\begin{definition}[rank]
\label{def:rank}
全充足塔 $T$ と元 $x \in \alpha$ に対して,
\[
  \rank_T(x)
  \;\stackrel{\mathrm{def}}{=}\;
  \min\{i \in \mathbb{N} \mid x \in T.\lev(i)\}
\]
を $x$ の \textbf{rank}(最小レベル)という.
Lean では \texttt{Nat.find}($\omega$-の最小元)で定義する.
\end{definition}

\begin{leancode}
noncomputable def ExhaustiveTower.rank
    {α : Type*} (T : ExhaustiveTower α) (x : α) : ℕ :=
  Nat.find (T.exhaustive x)   -- 最小の i を選択
\end{leancode}

rank は次の基本性質を持つ:

\begin{lemma}[rank\_spec]
\label{lem:rank-spec}
$x \in T.\lev(\rank_T(x))$
\quad(rank で達成される)
\end{lemma}

\begin{lemma}[rank\_le]
\label{lem:rank-le}
$x \in T.\lev(n) \Rightarrow \rank_T(x) \leq n$
\quad(rank は最小である)
\end{lemma}

\begin{lemma}[rank\_antitone]
\label{lem:rank-antitone}
$T_1 \leq T_2$(逐次包含)$\Rightarrow \rank_{T_2}(x) \leq \rank_{T_1}(x)$
\quad(細かい塔ほど rank が大きくなる)
\end{lemma}

\begin{proofstrategy}
補題3(rank の反単調性)の証明:
$T_1 \leq T_2$ より $T_1.\lev(\rank_{T_1}(x)) \subseteq T_2.\lev(\rank_{T_1}(x))$.
補題1(rank\_spec)より $x \in T_1.\lev(\rank_{T_1}(x))$,
したがって $x \in T_2.\lev(\rank_{T_1}(x))$.
補題2(rank\_le)より $\rank_{T_2}(x) \leq \rank_{T_1}(x)$.
\end{proofstrategy}

\subsection{一意性定理}

\begin{theorem}[定理B:rank の一意性]
\label{thm:rank-unique}
全充足塔 $T$ に対して,以下の二条件は同値である:
\begin{enumerate}[label=\textup{(\roman*)}]
  \item 関数 $r \colon \alpha \to \mathbb{N}$ が
        \textbf{強い単射公理}
        $\forall x,\, \forall i,\quad x \in T.\lev(i) \Leftrightarrow r(x) \leq i$
        を満たす
  \item $r = \rank_T$
\end{enumerate}
特に,強い単射公理を満たす関数は一意であり,$\rank_T$ に等しい.
\end{theorem}

\begin{figure}[H]
  \centering
  \begin{tikzcd}[column sep=3.5em, row sep=3em]
    x \arrow[r, phantom, "{\in}"] & T.\lev(i)
    \arrow[d, Leftrightarrow, "\text{強い単射公理}"] \\
    r(x) \arrow[r, phantom, "{\leq}"] & i
  \end{tikzcd}
  \hspace{4em}
  \begin{tikzcd}[column sep=3em, row sep=3em]
    \alpha \arrow[r, "\rank_T"] \arrow[rd, "r"', bend right=15] & \mathbb{N} \\
    & \mathbb{N} \arrow[u, "\text{id}"', equal]
  \end{tikzcd}
  \caption{%
    左:強い単射公理(Iic-塔の条件).
    右:rank の一意性——強い単射公理を満たす $r$ は必ず $\rank_T$ に等しい.
  }
  \label{fig:rank-unique}
\end{figure}

\begin{proofstrategy}
$r = \rank_T$ を $\mathtt{Nat.le\_antisymm}$ で示す:

\medskip
\noindent\textbf{$r(x) \leq \rank_T(x)$}:
rank\_spec より $x \in T.\lev(\rank_T(x))$.
強い単射公理の $(\Rightarrow)$ 方向より $r(x) \leq \rank_T(x)$.

\medskip
\noindent\textbf{$\rank_T(x) \leq r(x)$}:
強い単射公理の $(\Leftarrow)$ 方向に $r(x) \leq r(x)$(自明)を代入すると
$x \in T.\lev(r(x))$ が成立する.
rank\_le より $\rank_T(x) \leq r(x)$.
\end{proofstrategy}

\begin{leancode}
theorem ExhaustiveTower.rank_unique_iff {α : Type*} (T : ExhaustiveTower α)
    (r : α → ℕ) :
    (∀ x i, x ∈ T.level i ↔ r x ≤ i) → r = T.rank := by
  intro hchar
  funext x
  apply Nat.le_antisymm
  · -- r x ≤ rank(x): rank_spec から hchar の ⇒ 方向で移す
    exact (hchar x (T.rank x)).1 (T.rank_spec x)
  · -- rank(x) ≤ r x: hchar の ⇐ 方向(r x ≤ r x)から rank_le を使う
    exact T.rank_le x (r x) ((hchar x (r x)).2 (le_refl _))
\end{leancode}

\begin{theorem}[Iic-塔の存在]
\label{thm:iic-tower}
任意の関数 $r \colon \alpha \to \mathbb{N}$ に対して,
強い単射公理 $x \in T.\lev(i) \Leftrightarrow r(x) \leq i$
を満たす全充足塔 $T_r$ が存在する.
\[
  T_r.\lev(i) = \{x \in \alpha \mid r(x) \leq i\} = r^{-1}([0, i])
\]
この $T_r$ を \textbf{Iic-塔}という.
\end{theorem}

%% =============================================================================
\section{定理C: EMAlgebras の完備束構造}
%% =============================================================================

\subsection{EM代数の基本性質}

\begin{proposition}[EM代数の特徴づけ]
\label{prop:em-algebra-char}
閉包作用素 $c$ に対して,
$x \in \EMAlg(c) \Leftrightarrow c(x) = x$.
\end{proposition}

\begin{proof}
拡大性 $x \leq c(x)$ と仮定 $c(x) \leq x$ を合わせると $c(x) = x$.
逆は定義から自明.
\end{proof}

\begin{proposition}
\label{prop:closure-mem-em}
任意の $x \in \alpha$ に対して $c(x) \in \EMAlg(c)$.
\end{proposition}

\begin{proof}
冪等性 $c(c(x)) = c(x)$ より.
\end{proof}

\subsection{主定理}

\begin{theorem}[定理C:EMAlgebras の完備束構造]
\label{thm:em-iinf}
族 $(x_s)_{s \in \sigma} \subseteq \EMAlg(c)$ に対して,
任意の $y \in \alpha$ について
$\forall s,\; y \leq x_s$($y$ は全 $x_s$ の下界)ならば
$\forall s,\; c(y) \leq x_s$($c(y)$ も全 $x_s$ の下界)が成立する.

すなわち,$\EMAlg(c)$ は任意交叉(無限下限)で閉じており,
完備束の下半束(meet-semilattice)をなす.
\end{theorem}

\begin{figure}[H]
  \centering
  \begin{tikzcd}[column sep=4em, row sep=3em]
    \alpha \arrow[r, "c", bend left=10]
    \arrow[r, "{c \circ c = c}"', phantom, shift right=4pt]
    & \alpha \\
    \Fix(c) \arrow[u, hook, "{\iota}"] \arrow[ur, "c \circ \iota = \iota"', bend right=10] &
  \end{tikzcd}
  \hspace{4em}
  \begin{tikzcd}[column sep=3em, row sep=2.5em]
    & y \arrow[dl, "{\leq x_s}"' near start] \arrow[d, "{c}" description] \\
    x_s \in \EMAlg(c) & c(y) \arrow[l, "{\leq x_s}"]
  \end{tikzcd}
  \caption{%
    左:閉包作用素の冪等性($\Fix(c)$ への収縮).
    右:下界 $y$ を $c$ で写しても下界のまま——定理Cの証明アイデア.
  }
  \label{fig:em-diagram}
\end{figure}

\begin{proofstrategy}
$c(y) \leq x_s$ を示す:
\[
  c(y) \;\leq\; c(x_s) \;=\; x_s.
\]
第一の不等式は $y \leq x_s$ と $c$ の単調性から.
等式は $x_s \in \EMAlg(c)$,すなわち $c(x_s) = x_s$ から.

Lean では \texttt{calc} を用いて:
\begin{center}
  \texttt{calc c y ≤ c x := c.monotone hyx\quad  \_   = x   := (emAlgebra\_iff\_fixed ...).mp hx}
\end{center}
\end{proofstrategy}

\begin{leancode}
-- C-3: EM代数の上界性(中心補題)
theorem emAlgebra_upper_bound (c : ClosureOperator α) {x y : α}
    (hx : x ∈ EMAlgebras c) (hyx : y ≤ x) : c y ≤ x := by
  calc c y ≤ c x := c.monotone hyx
    _      = x   := (emAlgebra_iff_fixed c x).mp hx

-- C-6 (主定理C): EM代数は iInf に閉じている
theorem emAlgebras_iInf_closed (c : ClosureOperator α)
    {σ : Type*} (xs : σ → α)
    (hxs : ∀ i, xs i ∈ EMAlgebras c) :
    ∀ y, (∀ i, y ≤ xs i) → (∀ i, c y ≤ xs i) :=
  fun _y hy i => emAlgebra_upper_bound c (hxs i) (hy i)
\end{leancode}

\subsection{towerOfClosure との接続}

\begin{proposition}[EM代数と安定層]
\label{prop:em-stable-level}
$x \in \EMAlg(c)
  \;\Leftrightarrow\;
  (\mathsf{towerOfClosure}(c)).\lev(x) = {\downarrow}\,x$.

すなわち,EM代数の元 $x$ は「自己安定層」を持つ:
$c(x) = x$ ゆえ ${\downarrow}\,c(x) = {\downarrow}\,x$.
\end{proposition}

\begin{figure}[H]
  \centering
  \begin{tikzcd}[column sep=5em, row sep=3em]
    (\alpha,\leq)
      \arrow[r, "\mathsf{towerOfClosure}(c)"]
    & \mathsf{StructureTower}(\alpha,\alpha) \\
    \Fix(c)
      \arrow[u, hook]
      \arrow[r, "\text{安定層}"]
    & \{T \mid T.\lev(x) = {\downarrow}\,x\}
      \arrow[u, hook]
  \end{tikzcd}
  \caption{%
    EM代数(固定点集合)と塔の安定層の対応.
    $\Fix(c)$ の元は $\mathsf{towerOfClosure}(c)$ において
    自己安定なレベルを持つ.
  }
  \label{fig:em-tower}
\end{figure}

%% =============================================================================
\section{接続定理(Bridge): 代数的フィルトレーション}
%% =============================================================================

本節では \texttt{StructureTower\_Bridge.lean} の内容を解説する.
目標は,FilteredRing(次数付き環のフィルトレーション)における
乗法の条件が StructureTower の Hom として圏論的に定式化できることを示すことである.

\subsection{二項タワー(BinaryTower)}

乗法 $\ast \colon R \times R \to R$ を Tower の射として捉えるには,
$R \times R$ 上のタワー構造が必要である.

\begin{definition}[BinaryTower]
\label{def:binary-tower}
$T \colon \mathsf{StructureTower}(\iota, R)$ に対して,
\[
  (\mathsf{BinaryTower}\,T).\lev(i, j)
  \;\stackrel{\mathrm{def}}{=}\;
  T.\lev(i) \times T.\lev(j)
\]
で $\mathsf{StructureTower}(\iota \times \iota, R \times R)$ を定める.
$\iota \times \iota$ の前順序は積順序
$(i,j) \leq (i',j') \Leftrightarrow i \leq i' \wedge j \leq j'$.
\end{definition}

\begin{leancode}
def binaryTower (T : StructureTower ι α) : StructureTower (ι × ι) (α × α) where
  level ij := T.level ij.1 ×ˢ T.level ij.2   -- Set.prod
  monotone_level := by
    intro ⟨i₁, i₂⟩ ⟨j₁, j₂⟩ ⟨h₁, h₂⟩ ⟨x, y⟩ ⟨hx, hy⟩
    exact ⟨T.monotone_level h₁ hx, T.monotone_level h₂ hy⟩
\end{leancode}

\subsection{FilteredRing とその乗法 Hom}

\begin{definition}[FilteredRing]
\label{def:filtered-ring}
$\iota$(可換加法モノイド + 前順序,加法が単調)と環 $R$ に対して,
\textbf{次数付き環のフィルトレーション} $F$ は
構造塔 $F \colon \mathsf{StructureTower}(\iota, R)$ であって次を満たすもの:
\begin{enumerate}[label=\textup{(\roman*)}]
  \item $0 \in F.\lev(i)$ (零元を含む)
  \item $x, y \in F.\lev(i) \Rightarrow x + y \in F.\lev(i)$ (加法閉)
  \item $x \in F.\lev(i) \Rightarrow -x \in F.\lev(i)$ (逆元閉)
  \item $1 \in F.\lev(0)$ (単位元は最小層)
  \item $x \in F.\lev(i),\; y \in F.\lev(j) \Rightarrow x \cdot y \in F.\lev(i+j)$
        (\textbf{乗法の次数加算},mul\_mem 公理)
\end{enumerate}
\end{definition}

条件 (v) が StructureTower の Hom として解釈できる:

\begin{theorem}[乗法と Hom の同値]
\label{thm:mul-hom-equiv}
FilteredRing $F$ に対して,乗法 $\ast \colon R \times R \to R$ は
\[
  \mathsf{BinaryTower}(F)
  \xrightarrow{\ (x,y)\,\mapsto\, x \cdot y\ }
  \reindex_{+}(F)
\]
の Hom になる.ただし $\reindex_{+}(F)$ は
$(i,j) \mapsto i+j$ による引き戻し.さらに:
\[
  \text{mul\_mem 公理}
  \;\Longleftrightarrow\;
  \text{乗法が }\mathsf{BinaryTower}(F) \to \reindex_{(i,j)\mapsto i+j}(F)\text{ の Hom}
\]
\end{theorem}

\begin{figure}[H]
  \centering
  \begin{tikzcd}[column sep=5em, row sep=3.5em]
    \mathsf{BinaryTower}(F)
    \arrow[r, "{(x,y)\mapsto x\cdot y}", "\mathrm{Hom}"']
    & \reindex_{+}(F) \\
    F.\lev(i) \times F.\lev(j)
    \arrow[u, phantom, "\ni"{sloped}]
    \arrow[r, "\ni\; x\cdot y"']
    & F.\lev(i+j)
    \arrow[u, phantom, "\ni"{sloped}]
  \end{tikzcd}
  \caption{%
    定理(乗法 Hom 同値):FilteredRing の mul\_mem 公理は,
    乗法が BinaryTower から reindex タワーへの Hom であることと同値.
  }
  \label{fig:mul-hom}
\end{figure}

\begin{leancode}
-- 方向1: mul_mem → Hom の構成
def FilteredRing.toMulHom (F : FilteredRing ι R) :
    Hom (binaryTower F.toStructureTower)
        (reindex (fun ij : ι × ι => ij.1 + ij.2) fst_add_snd_monotone
          F.toStructureTower) where
  toFun     := fun ⟨x, y⟩ => x * y
  preserves := by
    intro ⟨i, j⟩ ⟨x, y⟩ ⟨hx, hy⟩
    simp [reindex]
    exact F.mul_mem i j hx hy  -- mul_mem 公理を直接適用
\end{leancode}

\subsection{FilteredRingHom と Tower の Hom}

\begin{definition}[FilteredRingHom]
\label{def:filtered-ring-hom}
FilteredRing $F \colon \mathsf{FilteredRing}(\iota, R)$,
$G \colon \mathsf{FilteredRing}(\iota, S)$ に対して,
\textbf{FilteredRingHom} $\varphi \colon F \to G$ は:
\begin{itemize}
  \item 環準同型 $\varphi \colon R \to^{+*} S$
  \item レベル保存:$\forall i,\; \varphi(F.\lev(i)) \subseteq G.\lev(i)$
\end{itemize}
\end{definition}

\begin{proposition}
\label{prop:filtered-hom-is-tower-hom}
FilteredRingHom $\varphi \colon F \to G$ は,
下部の StructureTower の Hom になる.
\end{proposition}

\subsection{mapOnHom と $h_g$ 条件}

\texttt{StructureTower\_Bridge.lean} の核心は,
\texttt{mapOnHom} の $h_g$ 条件(可換性条件)が
FilteredRingHom の乗法互換性に対応するという観察である.

\begin{definition}[mapOnHom]
\label{def:map-on-hom}
$f \colon \alpha \to \beta$,
$g \colon \mathsf{Hom}(T_1, T_2)$,
$g_\beta \colon \beta \to \beta$,
および \textbf{可換性条件}($h_g$ 条件)
\[
  \forall x,\quad g_\beta(f(x)) = f(g(x))
\]
が与えられたとき,\textbf{mapOnHom} は
$\mathsf{Hom}(\mathsf{map}_f(T_1),\, \mathsf{map}_f(T_2))$
を構成する(\texttt{toFun} $= g_\beta$).
\end{definition}

\begin{figure}[H]
  \centering
  \begin{tikzcd}[column sep=4.5em, row sep=3.5em]
    \alpha \arrow[r, "f"] \arrow[d, "g"'] & \beta \arrow[d, "g_\beta"] \\
    \alpha \arrow[r, "f"'] & \beta
  \end{tikzcd}
  \caption{%
    $h_g$ 条件:$g_\beta \circ f = f \circ g$.
    これが mapOnHom の存在条件であり,
    同時に FilteredRingHom の乗法互換性の圏論的本質でもある.
  }
  \label{fig:hg-cond}
\end{figure}

\subsection{統合定理:可換図式}

FilteredRingHom $\varphi \colon F \to G$ があるとき,
次の可換図式が成立する:

\begin{figure}[H]
  \centering
  \begin{tikzcd}[column sep=5em, row sep=4em]
    \mathsf{BinaryTower}(F)
    \arrow[r, "{\varphi \times \varphi}"]
    \arrow[d, "\ast_F"']
    & \mathsf{BinaryTower}(G)
    \arrow[d, "\ast_G"] \\
    \reindex_{+}(F)
    \arrow[r, "\varphi"']
    & \reindex_{+}(G)
  \end{tikzcd}
  \caption{%
    FilteredRingHom の可換図式
    (定理 \texttt{mul\_hom\_square\_commutes}).
    上の経路:$(x,y) \mapsto (\varphi x, \varphi y) \mapsto \varphi(x) \cdot \varphi(y)$.
    下の経路:$(x,y) \mapsto x \cdot y \mapsto \varphi(x \cdot y)$.
    両者は $\varphi(x \cdot y) = \varphi(x) \cdot \varphi(y)$(環準同型)により一致.
  }
  \label{fig:filtered-hom-square}
\end{figure}

この可換性の証明は,環準同型の乗法保存という事実から直接得られる:

\begin{leancode}
-- mul_hom_square_commutes: 可換図式の成立
theorem mul_hom_square_commutes
    (F : FilteredRing ι R) (G : FilteredRing ι S)
    (φ : FilteredRingHom F G)
    (ij : ι × ι) (p : R × R) (...) :
    -- 両経路が一致することの等式
    φ.toFun (p.1 * p.2) = φ.toFun p.1 * φ.toFun p.2 :=
  map_mul φ.toFun p.1 p.2   -- RingHom の乗法保存

-- 系: φ が Hom かつ mul_mem が結合して φ(x*y) ∈ G.level(i+j) を導く
theorem filteredRing_category_coherence
    (F : FilteredRing ι R) (G : FilteredRing ι S)
    (φ : FilteredRingHom F G) (i j : ι) {x y : R}
    (hx : x ∈ F.level i) (hy : y ∈ F.level j) :
    φ.toFun (x * y) ∈ G.level (i + j) := by
  rw [map_mul]
  exact G.mul_mem i j (φ.preserves i hx) (φ.preserves j hy)
\end{leancode}

%% =============================================================================
\section{\texorpdfstring{$h_g$}{hg} 条件の3つの顔}
%% =============================================================================

接続定理の哲学的核心は,一見異なる3つの条件が同一の数学的構造の「3つの顔」であることだ.

\begin{figure}[H]
  \centering
  \begin{tikzcd}[column sep=0em, row sep=4em]
    & h_g\text{ 条件(mapOnHom)}
    \arrow[dl, Leftrightarrow, "\text{§2 Bridge}"']
    \arrow[dr, Leftrightarrow, "\text{§4 Bridge}"]
    & \\
    \text{mul\_mem 公理}
    \arrow[rr, Leftrightarrow, "\text{同値の連鎖}"']
    & & \text{FilteredRingHom の乗法互換}
  \end{tikzcd}
  \caption{$h_g$ 条件の同値の連鎖}
  \label{fig:hg-three-faces}
\end{figure}

\begin{enumerate}[leftmargin=2em, label=\textbf{顔\arabic*:}]
  \item \textbf{CategoricalView(mapOnHom の設計条件)}\\
        $g_\beta \circ f = f \circ g$\\
        「$\beta$-側の操作 $g_\beta$ が $f$ を通じて $\alpha$-側の操作 $g$ と可換」

  \item \textbf{EscapeExercises(FilteredRing.mul\_mem)}\\
        $x \in F(i),\; y \in F(j) \Rightarrow x \cdot y \in F(i+j)$\\
        「乗法が次数加算と可換」

  \item \textbf{Bridge(FilteredRingHom の乗法互換性)}\\
        $\varphi(x \cdot y) = \varphi(x) \cdot \varphi(y)$
        かつ $\varphi(F(i)) \subseteq G(i)$\\
        「環準同型が次数を保つ」
\end{enumerate}

\medskip

\textbf{同値の連鎖}(ファイルコメントより):
\begin{align*}
  &\text{FilteredRing.mul\_mem} \\
  \Longleftrightarrow\; &\text{(§2)乗法が }
    \mathsf{BinaryTower}(F) \to \reindex_{+}(F)\text{ の Hom} \\
  \Longleftrightarrow\; &\text{(§4)mapOnHom の }h_g\text{ 条件が成立} \\
  \Longleftrightarrow\; &\text{(§3)FilteredRingHom のレベル保存 + 乗法可換性}
\end{align*}

\begin{mathinsight}
この同値の連鎖が「EscapeExercises(代数的フィルトレーション)と
CategoricalView(圏論的視点)の橋渡し」(Bridge ファイルの目的)である.

特に,FilteredRing の圏における射の条件
(レベル保存 + 環準同型)が,
StructureTower の圏における Hom の条件
(mapOnHom の $h_g$ 型)と完全に一致することは,
この形式化プロジェクトの中心的な知見の一つである.
\end{mathinsight}

%% =============================================================================
\section{まとめ}
%% =============================================================================

本稿では,StructureTower の形式化における3本の主定理と接続定理を解説した.

\begin{itemize}
  \item \textbf{定理A}(\ref{thm:reindex-functor}):
        $\reindex_f$ は NatInclusion の圏の関手である.
        証明の核心は「$\mathtt{fun\; i \Rightarrow h\;(f\;i)}$」という一行に凝縮される.

  \item \textbf{定理B}(\ref{thm:rank-unique}):
        ExhaustiveTower における rank は,
        強い単射公理(Iic-塔の条件)によって一意に特徴づけられる.
        Lean の \texttt{Nat.find} と \texttt{Nat.le\_antisymm} が核心を担う.

  \item \textbf{定理C}(\ref{thm:em-iinf}):
        閉包作用素の EM代数は任意交叉で閉じており,
        完備束の下半束をなす.
        単調性と固定点条件の calc チェーンが証明を短くする.

  \item \textbf{接続定理}(\ref{thm:mul-hom-equiv}, 図~\ref{fig:filtered-hom-square}):
        FilteredRing の mul\_mem 公理は,
        BinaryTower から reindex タワーへの Hom と同値であり,
        mapOnHom の $h_g$ 条件の代数的具現化である.
\end{itemize}

\begin{remark}[OrderHom との比較]
単なる \texttt{OrderHom ι (Set α)}(単調な集合値写像)では:
\begin{itemize}
  \item rank の情報(最小レベル関数)を表現できない(定理Bの非自明性の源)
  \item reindex の関手性は「自明な前合成」に見えるが,
        NatInclusion の圏での関手的性質は定理として明示的に述べる必要がある
        (定理Aの意義)
  \item EM代数と iInf の相互作用(定理C)は
        集合の包含だけでは見えない構造
\end{itemize}
StructureTower は「単調集合族」に
\textbf{数学的な意味付け・圏論的な枠組み}を与える
型理論的構造として機能する.
\end{remark}

\end{document}
