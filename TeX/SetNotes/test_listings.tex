\documentclass[a4paper,11pt]{ltjsarticle}
\usepackage{amsmath, amssymb}
\usepackage{tcolorbox}
\tcbuselibrary{skins, breakable, listings}

\newtcblisting{leanbox}[1][]{%
  enhanced, breakable,
  colback=gray!5, colframe=gray!60!black,
  fonttitle=\bfseries\ttfamily,
  title={Lean 4 形式化},
  listing only,
  listing options={
    basicstyle=\ttfamily\small,
    breaklines=true,
    literate={ℝ}{{$\mathbb{R}$}}1 {→}{{$\to$}}1 {≤}{{$\le$}}1 {∈}{{$\in$}}1 {∃}{{$\exists$}}1 {∀}{{$\forall$}}1 {∫}{{$\int$}}1 {∘}{{$\circ$}}1
  },
  #1
}

\begin{document}
Test:
\begin{leanbox}
/-- 零点定理 (Bolzano): f(a) < 0 < f(b) ならば零点が存在。 -/
theorem bolzano {f : ℝ → ℝ} {a b : ℝ} (hab : a ≤ b)
    (hf : ContinuousOn f (Icc a b))
    (ha : f a < 0) (hb : 0 < f b) :
    ∃ c ∈ Icc a b, f c = 0
\end{leanbox}
\end{document}
