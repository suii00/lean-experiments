\documentclass[11pt,a4paper]{ltjsarticle}
\usepackage{amsmath, amssymb, amsthm}
\usepackage{tikz-cd}
\usepackage{hyperref}

\newtheorem{theorem}{定理}[section]
\newtheorem{lemma}[theorem]{補題}
\newtheorem{definition}[theorem]{定義}
\newtheorem{proposition}[theorem]{命題}
\newtheorem{corollary}[theorem]{系}
\newtheorem{remark}[theorem]{注釈}
\newtheorem{example}[theorem]{例}

\title{母なる構造の融合\\
{\large ガロア接続 $\times$ 順序理論 $\times$ 閉包作用素}}
\author{su}
\date{\today}

\begin{document}
\maketitle

\begin{center}
\small
\textit{AI assistance disclosure:}
Lean ソースコードおよび \TeX 文書は
Antigravity (Google DeepMind) で生成した。
著者による加筆・修正は行っていない。
内容の正確性は保証されず、誤りがあれば著者の責任である。
\end{center}

\begin{abstract}
Nicolas Bourbaki が提唱した三つの母なる構造(structures m\`eres)
——代数的構造・順序構造・位相的構造——に共通する
\textbf{閉包系}(syst\`eme de fermeture)を統一原理として抽出し、
ガロア接続がこれら母構造間の「橋」として閉包を誘導する構図を
Lean 4 / Mathlib 上で形式化した結果を解説する。
特に、$\sigma$-代数への応用を通じて、
三母構造が一点に会する「結節点」の存在を示す。
\end{abstract}

\tableofcontents

%% ================================================================
\section{序:母なる構造の精神}
%% ================================================================

Bourbaki は数学の全構造を三つの\textbf{母なる構造}に還元した:

\begin{center}
\begin{tabular}{lll}
\hline
母構造 & 本質 & 閉包の例 \\
\hline
代数的構造 & 演算の法則 & 部分群生成, イデアル生成 \\
順序構造   & 順序の法則 & 順序完備化, フィルター生成 \\
位相的構造 & 近さの法則 & 位相的閉包 \\
\hline
\end{tabular}
\end{center}

本稿の核心的な洞察は、この三者に共通するパターンの存在である:
\textbf{すべての母構造は閉包系を持ち、
ガロア接続がそれら母構造間で閉包を受け渡す橋として機能する}。

\begin{equation*}
\begin{tikzcd}[column sep=large, row sep=large]
\text{代数的} \arrow[rd, "\text{部分代数の束}"'] &
& \text{位相的} \arrow[ld, "\text{閉集合系}"] \\
& \text{閉包系} \arrow[u, "\text{ガロア接続}", bend left, dashed]
  \arrow[ul, "\text{ガロア接続}"', bend right, dashed]
  \arrow[d, "\text{統一原理}"] & \\
& \text{順序的} \arrow[u, "\text{完備化}"'] &
\end{tikzcd}
\end{equation*}

%% ================================================================
\section{閉包系 — 母なる構造の共通骨格}
%% ================================================================

\begin{definition}[閉包系]
集合 $\alpha$ 上の\textbf{閉包系}(closure system)とは、
$\alpha$ の冪集合 $\mathcal{P}(\alpha)$ の部分集合
$\mathcal{C} \subseteq \mathcal{P}(\alpha)$ であって、
以下の公理を満たすものである:
\begin{enumerate}
\item \textbf{全体は閉}:$\alpha \in \mathcal{C}$。
\item \textbf{交叉で閉}:$\mathcal{S} \subseteq \mathcal{C}$ が
空でないならば $\bigcap \mathcal{S} \in \mathcal{C}$。
\end{enumerate}
\end{definition}

\begin{proposition}[閉包作用素の導出]
閉包系 $\mathcal{C}$ から、閉包作用素
$\mathrm{cl} \colon \mathcal{P}(\alpha) \to \mathcal{P}(\alpha)$ が
\[
\mathrm{cl}(S) = \bigcap \{ T \in \mathcal{C} \mid S \subseteq T \}
\]
として導出される。この $\mathrm{cl}$ は以下を満たす:
\begin{itemize}
\item \textbf{膨張性}:$S \subseteq \mathrm{cl}(S)$
\item \textbf{単調性}:$S \subseteq T \implies \mathrm{cl}(S) \subseteq \mathrm{cl}(T)$
\item \textbf{冪等性}:$\mathrm{cl}(\mathrm{cl}(S)) = \mathrm{cl}(S)$
\end{itemize}
\end{proposition}

\begin{proposition}[閉 $\Leftrightarrow$ 不動点]
$S \in \mathcal{C} \iff \mathrm{cl}(S) = S$.
\end{proposition}

これらの性質は Lean ファイルの \texttt{ClosureSystem} 名前空間で形式的に証明されている。

%% ================================================================
\section{ガロア接続 — 母構造間の橋}
%% ================================================================

\begin{definition}[ガロア接続]
半順序集合 $(A, \le_A)$ と $(B, \le_B)$ の間の
\textbf{ガロア接続}(Galois connection)とは、
単調写像の対 $l \colon A \to B$, $u \colon B \to A$ であって、
\[
l(a) \le_B b \iff a \le_A u(b) \quad (\forall a \in A, b \in B)
\]
を満たすものである。$l$ を\textbf{下側随伴}(左随伴)、
$u$ を\textbf{上側随伴}(右随伴)と呼ぶ。
\end{definition}

\begin{definition}[GaloisClosureSystem]
ガロア接続 $(l, u)$ から以下が導出される:
\begin{itemize}
\item \textbf{閉包作用素}:$\mathrm{cl} := u \circ l \colon A \to A$
  (膨張的・単調・冪等)
\item \textbf{核作用素}:$\mathrm{ker} := l \circ u \colon B \to B$
  (縮小的・単調・冪等)
\end{itemize}
\end{definition}

これらが自動的に導出される構図を図式で示す:

\begin{equation*}
\begin{tikzcd}
A \arrow[r, "l", shift left] \arrow[loop left, "u \circ l = \mathrm{cl}"]
& B \arrow[l, "u", shift left] \arrow[loop right, "l \circ u = \mathrm{ker}"]
\end{tikzcd}
\end{equation*}

\subsection{随伴恒等式}

ガロア接続の最も重要な性質は、以下の随伴恒等式である:

\begin{theorem}[随伴恒等式]
\begin{align}
l \circ u \circ l &= l \\
u \circ l \circ u &= u
\end{align}
\end{theorem}

\begin{proof}
(1) $l(u(l(x))) \le l(x)$ は $u(l(x)) \le u(l(x))$ から従い、
$l(x) \le l(u(l(x)))$ は $l$ の単調性と $x \le u(l(x))$ から従う。
(2) も同様に、$u$ の単調性と $l(u(y)) \le y$ から得られる。
\end{proof}

%% ================================================================
\section{不動点理論 — 閉元と余閉元}
%% ================================================================

\begin{definition}[閉元と余閉元]
ガロア接続 $(l, u)$ に対して:
\begin{itemize}
\item \textbf{閉元}:$\{ x \in A \mid u(l(x)) = x \}$
  (閉包作用素の不動点)
\item \textbf{余閉元}:$\{ y \in B \mid l(u(y)) = y \}$
  (核作用素の不動点)
\end{itemize}
\end{definition}

\begin{theorem}[閉元と像の一致]
\begin{align}
\text{閉元の集合} &= \mathrm{Im}(u) \\
\text{余閉元の集合} &= \mathrm{Im}(l)
\end{align}
\end{theorem}

\begin{proof}
随伴恒等式 $u \circ l \circ u = u$ より、
$u(y)$ は任意の $y \in B$ に対して閉元である。
逆に $x$ が閉元ならば $x = u(l(x)) \in \mathrm{Im}(u)$。
余閉元についても同様。
\end{proof}

この定理は、$l$ が閉元から余閉元への全射を、
$u$ が余閉元から閉元への全射を与えることを意味する。
すなわち、$l$ の制限と $u$ の制限は互いに逆写像の関係にあり、
閉元と余閉元の間の\textbf{順序同型}を誘導する。

\begin{equation*}
\begin{tikzcd}
\text{閉元} \arrow[r, "l|", shift left] &
\text{余閉元} \arrow[l, "u|", shift left]
\end{tikzcd}
\quad \cong
\end{equation*}

%% ================================================================
\section{具体例:$\sigma$-代数 — 三母構造の結節点}
%% ================================================================

$\sigma$-代数は三つの母構造が一点に会する\textbf{結節点}(n\oe ud)である:

\begin{center}
\begin{tabular}{ll}
\hline
母構造 & $\sigma$-代数との関係 \\
\hline
代数的 & Bool代数としての構造($\cap, \cup, {}^c$ の演算)\\
順序的 & 包含関係で完備束($\sigma$-代数の束)\\
位相的 & Borel $\sigma$-代数=位相から生成 \\
\hline
\end{tabular}
\end{center}

\begin{example}[$\sigma$-代数のガロア接続]
集合 $\alpha$ に対して、以下のガロア接続が構成される:
\begin{align*}
l &\colon \mathcal{P}(\mathcal{P}(\alpha)) \to \mathrm{MeasurableSpace}(\alpha) \\
  &\quad S \mapsto \sigma(S) \quad (\text{$S$ から生成される $\sigma$-代数}) \\
u &\colon \mathrm{MeasurableSpace}(\alpha) \to \mathcal{P}(\mathcal{P}(\alpha)) \\
  &\quad m \mapsto \{ s \mid s \text{ は $m$-可測} \}
\end{align*}
これは Mathlib の \texttt{MeasurableSpace.giGenerateFrom} である。
\end{example}

このガロア接続における閉包作用素 $u \circ l$ は、
集合族 $S$ を $\sigma(S)$ の可測集合族に送る操作そのものである:
\[
(u \circ l)(S) = \{ t \mid t \text{ は } \sigma(S)\text{-可測} \}
\]

\begin{corollary}[$\sigma$-代数の閉包性質]
\leavevmode
\begin{enumerate}
\item \textbf{膨張性}:$S \subseteq (u \circ l)(S)$
  (元の集合族は生成された $\sigma$-代数に含まれる)
\item \textbf{冪等性}:$(u \circ l)^2 = u \circ l$
  ($\sigma$-代数の $\sigma$-代数は元の $\sigma$-代数)
\end{enumerate}
\end{corollary}

%% ================================================================
\section{統一定理 — ガロア接続から閉包系へ}
%% ================================================================

\begin{theorem}[統一定理]
$\mathrm{Set}(\alpha)$ 上のガロア接続 $(l, u)$ が与えられたとき、
その閉元の集合
$\{ s \in \mathcal{P}(\alpha) \mid u(l(s)) = s \}$
は $\alpha$ 上の閉包系を成す。
\end{theorem}

\begin{proof}
閉包系の2つの公理を検証する:
\begin{enumerate}
\item $\alpha \in \mathcal{C}$:
  膨張性より $\alpha \subseteq u(l(\alpha))$。
  一方 $u(l(\alpha)) \subseteq \alpha$ は自明。
  よって $u(l(\alpha)) = \alpha$。

\item $\mathcal{S} \subseteq \mathcal{C}$ ならば
  $\bigcap \mathcal{S} \in \mathcal{C}$:
  膨張性より $\bigcap \mathcal{S} \subseteq u(l(\bigcap \mathcal{S}))$。
  逆に、各 $t \in \mathcal{S}$ に対して
  $\bigcap \mathcal{S} \subseteq t$ かつ $t = u(l(t))$($t$ は閉元)なので、
  $u(l(\bigcap \mathcal{S})) \subseteq u(l(t)) = t$。
  よって $u(l(\bigcap \mathcal{S})) \subseteq \bigcap \mathcal{S}$。
\end{enumerate}
\end{proof}

この定理は、\textbf{どの母構造間のガロア接続も、
閉包系という共通の骨格を生む}ことを保証する。
これこそが Bourbaki の精神を宿す統一原理である。

\end{document}
