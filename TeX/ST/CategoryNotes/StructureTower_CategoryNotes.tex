% =============================================================================
% 構造塔の圏論的構造 — Lean 4 形式化と数学的解説(完全版)
% 対象ファイル:
%   StructureTower_CategoryExercises.lean     (Level 1)
%   StructureTower_CategoryExercises_L2.lean  (Level 2)
%   StructureTower_CategoryExercises_L3.lean  (Level 3)
%   StructureTower_Grounding.lean             (Level 3+)
% LuaLaTeX でコンパイルすること。
% =============================================================================
\documentclass[a4paper,11pt]{ltjsarticle}

% --- Core packages ---
\usepackage{luatexja-fontspec}
\usepackage{amsmath,amssymb,amsthm}
\usepackage{mathtools}
\usepackage{tikz-cd}
\usetikzlibrary{cd,arrows.meta,positioning,fit,backgrounds}
\usepackage{enumitem}
\usepackage{hyperref}
\usepackage{cleveref}
\usepackage{tcolorbox}
\tcbuselibrary{breakable,skins}
\usepackage{listings}
\usepackage{xcolor}
\usepackage{geometry}
\usepackage{fancyhdr}
\usepackage{float}
\usepackage{booktabs}
\usepackage{array}
\usepackage{multicol}

% --- Page geometry ---
\geometry{margin=2.5cm, top=3cm, bottom=3cm}

% --- Fonts ---
\setmainjfont{Noto Serif JP}
\setsansjfont{Noto Sans JP}

% --- Colors ---
\definecolor{leanblue}{HTML}{2563EB}
\definecolor{leanbg}{HTML}{F8FAFC}
\definecolor{leancomment}{HTML}{6B7280}
\definecolor{leankeyword}{HTML}{7C3AED}
\definecolor{leanstring}{HTML}{059669}
\definecolor{accentcolor}{HTML}{1E40AF}
\definecolor{theoremcolor}{HTML}{EFF6FF}
\definecolor{definitioncolor}{HTML}{F0FDF4}
\definecolor{remarkcolor}{HTML}{FFF7ED}
\definecolor{groundingcolor}{HTML}{FDF4FF}

% --- Hyperref ---
\hypersetup{
  colorlinks=true,
  linkcolor=accentcolor,
  urlcolor=leanblue,
  citecolor=accentcolor,
  pdftitle={構造塔の圏論的構造},
  pdfauthor={su},
}

% --- Theorem environments ---
\theoremstyle{definition}
\newtheorem{definition}{定義}[section]
\newtheorem{example}[definition]{例}

\theoremstyle{plain}
\newtheorem{theorem}[definition]{定理}
\newtheorem{lemma}[definition]{補題}
\newtheorem{proposition}[definition]{命題}
\newtheorem{corollary}[definition]{系}

\theoremstyle{remark}
\newtheorem{remark}[definition]{注意}

% --- Lean code environment ---
\lstdefinelanguage{lean4}{
  morekeywords={def,theorem,lemma,example,instance,class,structure,
    where,let,in,do,if,then,else,match,with,fun,return,
    import,open,namespace,section,end,variable,
    inductive,abbrev,noncomputable,private,protected,
    sorry,admit,have,show,suffices,calc,by,exact,
    apply,intro,intros,constructor,cases,induction,
    simp,rw,rfl,ext,ring,linarith,omega,norm_num,
    decide,trivial,assumption,contradiction,rcases,refine,
    Type,Prop,Sort,Set,true,false,
    funext,congr,simpa,
    attribute,deriving},
  sensitive=true,
  morecomment=[l]{--},
  morecomment=[n]{/-}{-/},
  morestring=[b]",
  literate=
    {α}{{\(\alpha\)}}1
    {β}{{\(\beta\)}}1
    {γ}{{\(\gamma\)}}1
    {δ}{{\(\delta\)}}1
    {ε}{{\(\varepsilon\)}}1
    {η}{{\(\eta\)}}1
    {ι}{{\(\iota\)}}1
    {κ}{{\(\kappa\)}}1
    {μ}{{\(\mu\)}}1
    {φ}{{\(\varphi\)}}1
    {→}{{\(\to\)}}1
    {←}{{\(\leftarrow\)}}1
    {↔}{{\(\leftrightarrow\)}}1
    {⟨}{{\(\langle\)}}1
    {⟩}{{\(\rangle\)}}1
    {≤}{{\(\leq\)}}1
    {≥}{{\(\geq\)}}1
    {≠}{{\(\neq\)}}1
    {∀}{{\(\forall\)}}1
    {∃}{{\(\exists\)}}1
    {∧}{{\(\wedge\)}}1
    {∨}{{\(\vee\)}}1
    {¬}{{\(\neg\)}}1
    {∈}{{\(\in\)}}1
    {∉}{{\(\notin\)}}1
    {⊆}{{\(\subseteq\)}}1
    {⊂}{{\(\subset\)}}1
    {∪}{{\(\cup\)}}1
    {∩}{{\(\cap\)}}1
    {∅}{{\(\emptyset\)}}1
    {⊤}{{\(\top\)}}1
    {⊥}{{\(\bot\)}}1
    {∘}{{\(\circ\)}}1
    {×}{{\(\times\)}}1
    {⁻¹}{{\({}^{-1}\)}}1
    {ℕ}{{\(\mathbb{N}\)}}1
    {ℤ}{{\(\mathbb{Z}\)}}1
    {ℝ}{{\(\mathbb{R}\)}}1
    {ℚ}{{\(\mathbb{Q}\)}}1
    {:=}{{:=}}2
    {▸}{{\(\blacktriangleright\)}}1,
}

\lstnewenvironment{leancode}[1][]{%
  \lstset{
    language=lean4,
    basicstyle=\ttfamily\small,
    keywordstyle=\color{leankeyword}\bfseries,
    commentstyle=\color{leancomment}\itshape,
    stringstyle=\color{leanstring},
    backgroundcolor=\color{leanbg},
    frame=single,
    rulecolor=\color{leanblue!30},
    framesep=8pt,
    xleftmargin=12pt,
    xrightmargin=12pt,
    breaklines=true,
    breakatwhitespace=true,
    showstringspaces=false,
    tabsize=2,
    captionpos=b,
    numbers=left,
    numberstyle=\tiny\color{leancomment},
    numbersep=8pt,
    aboveskip=1em,
    belowskip=1em,
    #1
  }%
}{}

% --- Tcolorbox styles ---
\newtcolorbox{proofstrategy}{
  colback=remarkcolor,
  colframe=orange!60!black,
  title={\textbf{証明戦略}},
  fonttitle=\sffamily,
  boxrule=0.5pt,
  arc=3pt,
  breakable,
}

\newtcolorbox{mathinsight}{
  colback=theoremcolor,
  colframe=accentcolor,
  title={\textbf{数学的洞察}},
  fonttitle=\sffamily,
  boxrule=0.5pt,
  arc=3pt,
  breakable,
}

\newtcolorbox{correspondence}{
  colback=definitioncolor,
  colframe=green!60!black,
  title={\textbf{対応表}},
  fonttitle=\sffamily,
  boxrule=0.5pt,
  arc=3pt,
  breakable,
}

\newtcolorbox{groundingbox}{
  colback=groundingcolor,
  colframe=violet!60!black,
  title={\textbf{具体例}},
  fonttitle=\sffamily,
  boxrule=0.5pt,
  arc=3pt,
  breakable,
}

% --- Header/Footer ---
\pagestyle{fancy}
\fancyhf{}
\renewcommand{\headrulewidth}{0.4pt}
\fancyhead[L]{\small\sffamily\nouppercase{\leftmark}}
\fancyhead[R]{\small\sffamily Lean 4 形式化}
\fancyfoot[C]{\thepage}

% --- Macros ---
\newcommand{\ST}{\mathrm{ST}}
\newcommand{\Hom}{\mathrm{Hom}}
\newcommand{\id}{\mathrm{id}}
\newcommand{\liftCl}{\mathrm{liftCl}}
\newcommand{\cl}{\mathrm{cl}}
\newcommand{\union}{\mathrm{union}}
\newcommand{\globalf}{\mathrm{global}}
\newcommand{\Kl}{\mathrm{Kl}}
\newcommand{\EM}{\mathrm{EM}}
\newcommand{\pair}{\mathrm{pair}}
\newcommand{\reindex}{\mathrm{reindex}}
\newcommand{\levelL}[1]{L(#1)}

% =============================================================================
\begin{document}

% --- Title ---
\title{%
  {\LARGE\sffamily\bfseries 構造塔の圏論的構造}\\[0.3em]
  {\large\sffamily 圏の公理・関手・積・モナド・具体例}\\[0.5em]
  {\large\sffamily Lean 4 による形式化とその数学的解説}%
}
\author{su}
\date{\today}
\maketitle

% --- AI Disclosure ---
\begin{center}
\small
\textit{AI assistance disclosure:}\\
Lean ソースコードは Claude (Anthropic) で骨格を生成し、
Codex (OpenAI) で修正した。\\
\TeX\ 文書は Claude Code (Anthropic) で生成した。\\
著者による加筆・修正は行っていない。\\
内容の正確性は保証されず、誤りがあれば著者の責任である。
\end{center}

\vspace{1em}

% --- Abstract ---
\begin{abstract}
構造塔(StructureTower)は、前順序集合 $(\iota, \leq)$ で添字付けられた
単調増大部分集合族 $\{L_i\}_{i \in \iota}$ として定義される数学的構造である。
本稿では、構造塔とそのレベル保存写像(Hom)が圏をなすことを出発点として、
4つの Lean~4 形式化ファイルにまたがる圏論的構造を体系的に解説する。

\smallskip
レベル1(基礎)では圏の公理(恒等律・結合律)、共変・反変関手(map/comap)、
忘却関手(union)と添字変換(reindex)の整合性を確認する。
レベル2(中級)では層関手と自然変換、大域切断関手(global = $\bigcap_i L_i$)、
同型射(refl/symm/trans)、直積の普遍性、
$\mathrm{const} \dashv \mathrm{global}$ という随伴の萌芽、
および map と積の交換性を扱う。
レベル3(上級)では閉包作用素 $\mathrm{cl}$ から誘導される冪等モナド
$(\liftCl_{\cl}, \eta, \mu)$ を構成し、
閉包公理(拡大性・単調性・冪等性)とモナド公理(単位律・結合律)の
正確な対応を確立する。Kleisli 射の圏と Eilenberg--Moore 代数も構成し、
後者が「全レベルが $\cl$-不動点である塔」に一致することを示す。
接地編(Level~3+)では、$\cl$ を位相的閉包・部分群生成の2例に具体化し、
同一の API が両分野で機能することを検証する。
\end{abstract}

\tableofcontents
\newpage

% =============================================================================
\section{序論}\label{sec:intro}
% =============================================================================

\subsection{動機と背景}

Nicolas Bourbaki の「構造の母」(structures m\`{e}res)の思想に触発された
構造塔(StructureTower)は、順序・代数・位相という三つの母構造を統一的に
扱う枠組みとして設計された。
本稿の中心的な問いは、この構造塔が単調写像の言い換えに留まらず、
圏論的に豊かな内部構造を持つことをいかに示すかである。

構造塔のレベル保存写像 Hom は圏を構成し、この圏の上に
関手・自然変換・極限・モナドといった圏論的装置が自然に載る。
特に閉包作用素から誘導されるモナドは、
「この枠組みでないと自然に記述できない構造」の典型例となる。

\subsection{本稿の構成}

\begin{itemize}
  \item \textbf{\S\ref{sec:core}}: 核心的定義(構造塔・Hom・補助操作)
  \item \textbf{\S\ref{sec:level1}}: Level~1 — 圏の公理と基本関手
  \item \textbf{\S\ref{sec:level2}}: Level~2 — 層関手・同型・直積・随伴の萌芽
  \item \textbf{\S\ref{sec:level3}}: Level~3 — 閉包モナド・Kleisli・EM 代数
  \item \textbf{\S\ref{sec:grounding}}: Level~3+ — 具体的閉包作用素への接地
  \item \textbf{\S\ref{sec:conclusion}}: 全体像・冪等モナドの意義・今後の展望
\end{itemize}

% =============================================================================
\section{核心的定義}\label{sec:core}
% =============================================================================

\subsection{構造塔}

\begin{definition}[構造塔]\label{def:structure-tower}
前順序集合 $(\iota, \leq)$ と型 $\alpha$ に対し、
\textbf{構造塔} $T$ とは、
写像 $L : \iota \to \mathcal{P}(\alpha)$ であって
\[
  i \leq j \implies L(i) \subseteq L(j)
\]
を満たすものである。$L(i)$ を\textbf{レベル}$i$ と呼ぶ。
\end{definition}

\begin{leancode}
@[ext]
structure StructureTower (ι α : Type*) [Preorder ι] : Type _ where
  level : ι → Set α
  monotone_level : ∀ ⦃i j : ι⦄, i ≤ j → level i ⊆ level j
\end{leancode}

外延性属性 \texttt{@[ext]} により、レベル写像が等しければ塔全体が等しい。

\subsection{射(Hom)}

\begin{definition}[射]\label{def:hom}
$T_1 : \ST(\iota, \alpha)$、$T_2 : \ST(\iota, \beta)$ に対し、
\textbf{射} $f : \Hom(T_1, T_2)$ とは、
写像 $f : \alpha \to \beta$ であって
\[
  \forall\, i \in \iota,\quad f\bigl(L_1(i)\bigr) \subseteq L_2(i)
\]
を満たすものである。
\end{definition}

\begin{leancode}
structure Hom (T₁ : StructureTower ι α) (T₂ : StructureTower ι β) where
  toFun : α → β
  preserves : ∀ i, MapsTo toFun (T₁.level i) (T₂.level i)
\end{leancode}

\begin{remark}[射の外延性と proof irrelevance]
\texttt{preserves} は命題型 (Prop) である。
Lean~4 の proof irrelevance により、
射の等しさは基底写像 \texttt{toFun} の等しさに帰着する:
\[
  f.\mathrm{toFun} = g.\mathrm{toFun} \implies f = g
\]
これが以降の多くの証明で \texttt{Hom.ext rfl} が有効である根拠である。
\end{remark}

\subsection{補助的な操作}

\begin{definition}[補助的な塔の操作]\label{def:aux-ops}
\begin{align}
  \union(T) &:= \bigcup_{i \in \iota} L(i)
    &&\text{(union: 全レベルの和集合)}\\
  \globalf(T) &:= \bigcap_{i \in \iota} L(i)
    &&\text{(global: 全レベルの共通部分)}\\
  (\mathrm{map}\; f\; T).L(i) &:= f(L(i))
    &&\text{(順像)}\\
  (\mathrm{comap}\; f\; T).L(i) &:= f^{-1}(L(i))
    &&\text{(逆像)}\\
  (\reindex\; \varphi\; T).L(i) &:= T.L(\varphi(i))
    &&\text{(添字変換、$\varphi : \kappa \to \iota$ 単調)}\\
  (\mathrm{prod}\; T_1\; T_2).L(i) &:= L_1(i) \times L_2(i)
    &&\text{(レベルごとの直積)}\\
  (\mathrm{const}\; \iota\; S).L(i) &:= S
    &&\text{(定数塔)}
\end{align}
\end{definition}

% =============================================================================
\section{Level~1: 圏の公理と基本関手}\label{sec:level1}
% =============================================================================

\subsection{構造塔の圏}

\begin{theorem}[圏の公理]\label{thm:category}
恒等射 $\id_T$ と射の合成 $g \circ f$ により、
構造塔と射は圏 $\ST(\iota)$ を形成する:
\begin{enumerate}[label=(\roman*)]
  \item \textbf{左恒等律}: $\id \circ f = f$
  \item \textbf{右恒等律}: $f \circ \id = f$
  \item \textbf{結合律}: $(h \circ g) \circ f = h \circ (g \circ f)$
\end{enumerate}
\end{theorem}

\begin{leancode}
def Hom.id (T : StructureTower ι α) : Hom T T where
  toFun := id;  preserves := by intro i x hx; exact hx

def Hom.comp (g : Hom T₂ T₃) (f : Hom T₁ T₂) : Hom T₁ T₃ where
  toFun := g.toFun ∘ f.toFun
  preserves := by intro i x hx; exact g.preserves i (f.preserves i hx)

theorem Hom.id_comp (f : Hom T₁ T₂) :
    Hom.comp (Hom.id T₂) f = f := Hom.ext rfl

theorem Hom.comp_id (f : Hom T₁ T₂) :
    Hom.comp f (Hom.id T₁) = f := Hom.ext rfl

theorem Hom.comp_assoc (h : Hom T₃ T₄) (g : Hom T₂ T₃)
    (f : Hom T₁ T₂) :
    Hom.comp (Hom.comp h g) f
    = Hom.comp h (Hom.comp g f) := Hom.ext rfl
\end{leancode}

\begin{proofstrategy}
恒等射の基底写像は $\id_\alpha$、合成の基底写像は関数合成 $g \circ f$ である。
関数合成は定義的に結合的・単位元を持つため、
三公理はいずれも \texttt{Hom.ext rfl} で閉じる。
\end{proofstrategy}

\begin{figure}[H]
\centering
\begin{tikzcd}[column sep=large, row sep=large]
  T_1 \arrow[r, "f"] \arrow[rr, bend left=35, "g \circ f"]
  \arrow[rrr, bend left=50, "h \circ g \circ f"]
  & T_2 \arrow[r, "g"] \arrow[rr, bend left=35, "h \circ g"']
  & T_3 \arrow[r, "h"]
  & T_4
\end{tikzcd}
\caption{圏 $\ST(\iota)$ における射の合成(結合律の図式)}
\end{figure}

\subsection{共変関手 map と反変関手 comap}

\begin{theorem}[関手性]\label{thm:functoriality}
map は共変関手、comap は反変関手として振る舞う:
\begin{align}
  \mathrm{map}\;\id &= \id, &
  \mathrm{map}\;g \circ \mathrm{map}\;f &= \mathrm{map}\;(g \circ f) \\
  \mathrm{comap}\;\id &= \id, &
  \mathrm{comap}\;f \circ \mathrm{comap}\;g &= \mathrm{comap}\;(g \circ f)
\end{align}
\end{theorem}

\begin{proofstrategy}
塔の外延性(\texttt{ext i x})で各点に帰着後、
$\mathrm{image\_id}$、$\mathrm{preimage\_comp}$ などの
集合の標準補題から \texttt{simp} で処理される。
\end{proofstrategy}

\begin{figure}[H]
\centering
\begin{tikzcd}[column sep=5em, row sep=3em]
  \ST(\iota, \alpha)
    \arrow[r, "\mathrm{map}\; f"]
    \arrow[dr, "\mathrm{map}\;(g \circ f)"']
  & \ST(\iota, \beta)
    \arrow[d, "\mathrm{map}\; g"] \\
  & \ST(\iota, \gamma)
\end{tikzcd}
\qquad
\begin{tikzcd}[column sep=5em, row sep=3em]
  \ST(\iota, \gamma)
    \arrow[r, "\mathrm{comap}\; g"]
    \arrow[dr, "\mathrm{comap}\;(g \circ f)"']
  & \ST(\iota, \beta)
    \arrow[d, "\mathrm{comap}\; f"] \\
  & \ST(\iota, \alpha)
\end{tikzcd}
\caption{map(左、共変)と comap(右、反変)の関手性}
\label{fig:map-comap}
\end{figure}

\subsection{忘却写像と reindex}

\begin{proposition}[union の関手性]
射 $f : \Hom(T_1, T_2)$ は union を保存する:
$f(\union(T_1)) \subseteq \union(T_2)$。
恒等と合成も union 上で整合的に振る舞い、
$\union : \ST(\iota) \to \mathbf{Set}$ は関手をなす。
\end{proposition}

\begin{proof}
$x \in \union(T_1)$ とすれば、ある $i$ が存在して $x \in L_1(i)$。
$f.\mathrm{preserves}$ により $f(x) \in L_2(i)$、
したがって $f(x) \in \union(T_2)$。
\end{proof}

\begin{proposition}[reindex の関手性]
添字変換 $\varphi : (\kappa, \leq) \to (\iota, \leq)$(単調)に対し、
$\reindex\;\varphi$ は射を引き戻す:
任意の $f : \Hom(T_1, T_2)$ に対して
$\reindex\;\varphi\;(f) : \Hom(\reindex\;\varphi\;T_1,\; \reindex\;\varphi\;T_2)$
が得られる。これは $\reindex\;\varphi$ が関手であることを意味する。
\end{proposition}

\begin{leancode}
def Hom.reindex (f : Hom T₁ T₂) (φ : κ → ι) (hφ : Monotone φ) :
    Hom (reindex φ hφ T₁) (reindex φ hφ T₂) where
  toFun := f.toFun
  preserves := by intro k x hx; exact f.preserves (φ k) hx
\end{leancode}

% =============================================================================
\section{Level~2: 層関手・同型・直積・随伴の萌芽}\label{sec:level2}
% =============================================================================

\subsection{層関手と自然変換}

\begin{definition}[層関手]\label{def:layer-functor}
添字 $i \in \iota$ を固定する。
\textbf{層関手} $\mathrm{Ev}_i : \ST(\iota) \to \mathbf{Set}$ を
\[
  \mathrm{Ev}_i(T) := T.L(i), \qquad
  \mathrm{Ev}_i(f) := f|_{L_1(i)} : L_1(i) \to L_2(i)
\]
で定義する。ここで $f|_{L_1(i)}$ は部分型への制限(\texttt{restrictLevel})である。
\end{definition}

\begin{theorem}[関手性と自然性]
$\mathrm{Ev}_i$ は関手であり、
$i \leq j$ から誘導される包含射
$\iota_{ij} : L(i) \hookrightarrow L(j)$
は自然変換 $\mathrm{Ev}_i \Rightarrow \mathrm{Ev}_j$ を構成する。
\end{theorem}

\begin{figure}[H]
\centering
\begin{tikzcd}[column sep=6em, row sep=4em]
  L_1(i) \arrow[r, "{f|_i}"] \arrow[d, hook, "\iota_{ij}"']
  & L_2(i) \arrow[d, hook, "\iota_{ij}"] \\
  L_1(j) \arrow[r, "{f|_j}"']
  & L_2(j)
\end{tikzcd}
\caption{自然性の正方形: $f|_j \circ \iota_{ij} = \iota_{ij} \circ f|_i$}
\label{fig:naturality}
\end{figure}

\begin{leancode}
theorem levelInclusion_natural (f : Hom T₁ T₂)
    {i j : ι} (hij : i ≤ j) :
    (f.restrictLevel j) ∘ (levelInclusion T₁ hij)
    = (levelInclusion T₂ hij) ∘ (f.restrictLevel i) := by
  funext ⟨x, hx⟩; rfl
\end{leancode}

\begin{proofstrategy}
部分型の元 $\langle x, h_x\rangle$ に対して両辺を展開すると、
値は $f(x)$、型付けの証明は proof irrelevance で一致するため \texttt{rfl} で閉じる。
\end{proofstrategy}

\subsection{大域切断関手}

union が「最も緩い」見方であるのに対し、global は「最も厳しい」見方を与える。

\begin{definition}[大域切断]
$\globalf(T) := \bigcap_{i \in \iota} L(i)$
\end{definition}

\begin{proposition}\label{prop:global-functor}
$\globalf \subseteq \union$($\iota$ が空でないとき)であり、
射は global を保存する。すなわち $\globalf : \ST(\iota) \to \mathbf{Set}$ は関手である。
\end{proposition}

\begin{proof}
$x \in \globalf(T_1)$ とすれば、全 $i$ に対し $x \in L_1(i)$。
$f.\mathrm{preserves}\; i$ により $f(x) \in L_2(i)$。
全 $i$ で成り立つから $f(x) \in \globalf(T_2)$。
\end{proof}

定数塔との関係:
\[
  \globalf(\mathrm{const}(\iota, S)) = S
  \quad (\iota \text{ が空でないとき})
\]

\subsection{同型射}\label{sec:iso}

\begin{definition}[同型射]
\textbf{同型} $e : T_1 \xrightarrow{\sim} T_2$ は射の対
$(e.\mathrm{hom},\; e.\mathrm{inv})$ であって
\[
  e.\mathrm{inv} \circ e.\mathrm{hom} = \id_{T_1},
  \qquad
  e.\mathrm{hom} \circ e.\mathrm{inv} = \id_{T_2}
\]
を満たすものである。
\end{definition}

\begin{theorem}[同型の基本性質]
\begin{enumerate}[label=(\roman*)]
  \item 反射律: $\mathrm{Iso.refl}(T) : T \xrightarrow{\sim} T$
  \item 対称律: $e : T_1 \xrightarrow{\sim} T_2 \implies e^{-1} : T_2 \xrightarrow{\sim} T_1$
  \item 推移律: $e_1 : T_1 \xrightarrow{\sim} T_2,\; e_2 : T_2 \xrightarrow{\sim} T_3 \implies e_2 \circ e_1 : T_1 \xrightarrow{\sim} T_3$
  \item 各レベルで全単射: $e.\mathrm{hom}|_i : L_1(i) \xrightarrow{\sim} L_2(i)$
\end{enumerate}
\end{theorem}

\begin{leancode}
def Iso.trans (e₁ : Iso T₁ T₂) (e₂ : Iso T₂ T₃) : Iso T₁ T₃ where
  hom := Hom.comp e₂.hom e₁.hom
  inv := Hom.comp e₁.inv e₂.inv
  hom_inv_id := by
    -- (e₁.inv ∘ e₂.inv) ∘ (e₂.hom ∘ e₁.hom)
    -- = e₁.inv ∘ (e₂.inv ∘ e₂.hom) ∘ e₁.hom
    -- = e₁.inv ∘ id ∘ e₁.hom = id
    ...
\end{leancode}

型の全単射(Equiv)からも自然に同型が得られる:
\texttt{Iso.ofEquiv} は、双方向のレベル保存条件があれば
$\alpha \simeq \beta$ から $T_1 \cong T_2$ を構成する。

\subsection{直積と普遍性}\label{sec:product}

\begin{definition}[直積]
$(\mathrm{prod}\; T_1\; T_2).L(i) := L_1(i) \times L_2(i)$
\end{definition}

射影 $\mathrm{fst} : \mathrm{prod}\; T_1\; T_2 \to T_1$ と
$\mathrm{snd} : \mathrm{prod}\; T_1\; T_2 \to T_2$ は自然な Hom を与える。

\begin{theorem}[直積の普遍性]\label{thm:product-universal}
任意の $f : \Hom(T, T_1)$, $g : \Hom(T, T_2)$ に対して、
射 $\pair(f, g) : \Hom(T, \mathrm{prod}\; T_1\; T_2)$ が
\textbf{一意に}存在し、
\[
  \mathrm{fst} \circ \pair(f,g) = f
  \qquad\text{かつ}\qquad
  \mathrm{snd} \circ \pair(f,g) = g
\]
を満たす。
\end{theorem}

\begin{figure}[H]
\centering
\begin{tikzcd}[column sep=large, row sep=large]
  & T
    \arrow[dl, "f"']
    \arrow[dr, "g"]
    \arrow[d, dashed, "{\exists!\;\pair(f{,}g)}"]
  & \\
  T_1
  & \mathrm{prod}(T_1, T_2)
    \arrow[l, "\mathrm{fst}"]
    \arrow[r, "\mathrm{snd}"']
  & T_2
\end{tikzcd}
\caption{直積の普遍性}
\label{fig:product-universal}
\end{figure}

\begin{leancode}
def Hom.pair (f : Hom T T₁) (g : Hom T T₂) :
    Hom T (prod T₁ T₂) where
  toFun x := (f.toFun x, g.toFun x)
  preserves := by intro i x hx
    exact ⟨f.preserves i hx, g.preserves i hx⟩

theorem Hom.pair_unique (f : Hom T T₁) (g : Hom T T₂)
    (h : Hom T (prod T₁ T₂))
    (hf : Hom.comp (fst T₁ T₂) h = f)
    (hg : Hom.comp (snd T₁ T₂) h = g) :
    h = Hom.pair f g := by
  apply Hom.ext; funext x
  exact Prod.ext
    (congr_fun (congr_arg Hom.toFun hf) x)
    (congr_fun (congr_arg Hom.toFun hg) x)
\end{leancode}

\begin{proofstrategy}
一意性の証明:
$h$ と $\pair(f,g)$ の基底写像を点 $x$ ごとに比較する。
$\mathrm{fst} \circ h = f$ から $(h(x))_1 = f(x)$、
$\mathrm{snd} \circ h = g$ から $(h(x))_2 = g(x)$ が得られ、
$\mathrm{Prod.ext}$ で $h(x) = (f(x), g(x))$ が結論される。
\end{proofstrategy}

\subsection{直積の関手性}

\begin{proposition}[prodMap の関手性]
$f : \Hom(T_1, T_2)$, $g : \Hom(S_1, S_2)$ から
$\mathrm{prodMap}(f, g) : \Hom(\mathrm{prod}(T_1, S_1), \mathrm{prod}(T_2, S_2))$
が得られ、prodMap は恒等射と合成を保存する。
さらに:
\[
  \mathrm{map}(f \times g)(\mathrm{prod}(T_1, T_2))
  = \mathrm{prod}(\mathrm{map}(f, T_1), \mathrm{map}(g, T_2))
\]
\end{proposition}

\subsection{自由構造塔と随伴の萌芽}\label{sec:adjunction}

\begin{proposition}[随伴の萌芽]\label{prop:adjunction}
定数塔と大域切断の間に、以下の自然な全単射が成り立つ:
\[
  \Hom\bigl(\mathrm{const}(\iota, S),\; T\bigr)
  \;\cong\;
  \bigl\{\,f : \alpha \to \beta \;\big|\; f(S) \subseteq \globalf(T)\,\bigr\}
\]
すなわち $\mathrm{const} \dashv \globalf$ という随伴の萌芽である。
\end{proposition}

\begin{proof}
写像:
\begin{itemize}
  \item 右辺 $\to$ 左辺: $f(S) \subseteq \globalf(T)$ と
    $\globalf(T) \subseteq L(i)$(全 $i$)から
    $f(S) \subseteq L(i)$ が従い、$f$ は Hom となる。
  \item 左辺 $\to$ 右辺: Hom $h$ の $\mathrm{preserves}$ は
    全 $i$ で $h(S) \subseteq L(i)$ を与え、
    $h(S) \subseteq \bigcap_i L(i) = \globalf(T)$。
\end{itemize}
\end{proof}

\begin{figure}[H]
\centering
\begin{tikzcd}[column sep=7em]
  \mathbf{Set}
  \arrow[r, bend left=30, "{\mathrm{const}(\iota,\text{-})}"]
  \arrow[r, phantom, "\bot" description]
  & \ST(\iota)
  \arrow[l, bend left=30, "\globalf"]
\end{tikzcd}
\caption{随伴 $\mathrm{const}(\iota, -) \dashv \globalf$}
\label{fig:adjunction}
\end{figure}

% =============================================================================
\section{Level~3: 閉包モナド}\label{sec:level3}
% =============================================================================

\subsection{閉包作用素}

\begin{definition}[閉包作用素]\label{def:closure-op}
$\mathcal{P}(\alpha)$ 上の\textbf{閉包作用素} $\cl$ とは、
$(\mathcal{P}(\alpha), \subseteq)$ 上の作用素であって以下を満たすものである:
\begin{enumerate}[label=(\roman*)]
  \item \textbf{拡大性}(extensive): $A \subseteq \cl(A)$
  \item \textbf{単調性}(monotone): $A \subseteq B \implies \cl(A) \subseteq \cl(B)$
  \item \textbf{冪等性}(idempotent): $\cl(\cl(A)) = \cl(A)$
\end{enumerate}
\end{definition}

\begin{example}
典型的な閉包作用素:
位相的閉包 $\overline{(-)}$、
部分群生成 $\langle - \rangle$、
$\sigma$-代数の生成。
\end{example}

\subsection{Levelwise 自己関手 liftCl}\label{sec:liftcl}

\begin{definition}[$\liftCl$]\label{def:liftCl}
閉包作用素 $\cl$ と構造塔 $T$ に対し、
\[
  \liftCl(\cl, T).L(i) := \cl(L(i))
\]
と定義する。$\cl$ の単調性により、$\liftCl(\cl, T)$ は再び構造塔をなす。
\end{definition}

\begin{leancode}
def liftCl (cl : ClosureOperator (Set α))
    (T : StructureTower ι α) : StructureTower ι α where
  level i := cl (T.level i)
  monotone_level := by
    intro i j hij x hx
    exact cl.monotone (T.monotone_level hij) hx
\end{leancode}

\subsection{Unit 自然変換 $\eta$}\label{sec:unit}

\begin{definition}[Unit]\label{def:unit}
拡大性 $L(i) \subseteq \cl(L(i))$ から、自然な射
$\eta_T : T \to \liftCl(\cl, T)$ が得られる:
$\eta_T = (\id_\alpha, \cl.\mathrm{le\_closure})$。
基底写像は $\id$ であり、preserves は拡大性そのものである。
\end{definition}

\begin{figure}[H]
\centering
\begin{tikzcd}[column sep=6em, row sep=4em]
  T_1 \arrow[r, "\eta_{T_1}"] \arrow[d, "\iota"']
  & \liftCl(T_1) \arrow[d, "\liftCl(\iota)"] \\
  T_2 \arrow[r, "\eta_{T_2}"']
  & \liftCl(T_2)
\end{tikzcd}
\caption{Unit $\eta$ の自然性($\iota$ は包含射)}
\label{fig:unit-natural}
\end{figure}

\subsection{Join(乗法)$\mu$}\label{sec:join}

\begin{definition}[Join]\label{def:join}
冪等性 $\cl(\cl(A)) = \cl(A)$ の $\supseteq$ 方向から、
$\mu_T : \liftCl^2(T) \to \liftCl(T)$ が得られる:
$\mu_T = (\id_\alpha, \cl.\mathrm{idempotent} \text{ の } \supseteq \text{方向})$。
\end{definition}

\begin{proposition}
$\mu_T$ と $(\eta_{\liftCl(T)} : \liftCl(T) \to \liftCl^2(T))$ は互いに逆射:
$\mu_T \circ \eta = \id$ かつ $\eta \circ \mu_T = \id$。
これは $\liftCl^2(T) \cong \liftCl(T)$ を意味する(冪等モナドの特徴)。
\end{proposition}

\subsection{モナド法則}\label{sec:monad-laws}

\begin{theorem}[閉包モナド]\label{thm:monad-laws}
三つ組 $(\liftCl_\cl, \eta, \mu)$ はモナドをなす:
\[
  \mu \circ \eta_{\cl(T)} = \id_{\cl(T)}
  \qquad
  \mu \circ \liftCl(\eta) = \id_{\cl(T)}
  \qquad
  \mu \circ \mu_{\cl(T)} = \mu \circ \liftCl(\mu)
\]
\end{theorem}

\begin{leancode}
-- 左単位律
theorem monad_left_unit (T : StructureTower ι α) :
    Hom.comp (join cl T) (unit cl (liftCl cl T))
    = Hom.id (liftCl cl T) := Hom.ext rfl

-- 右単位律
theorem monad_right_unit (T : StructureTower ι α) :
    Hom.comp (join cl T)
      (liftCl_mapId cl T (liftCl cl T)
        (fun i => cl.le_closure (T.level i)))
    = Hom.id (liftCl cl T) := Hom.ext rfl

-- 結合律
theorem monad_assoc (T : StructureTower ι α) :
    Hom.comp (join cl T) (join cl (liftCl cl T))
    = Hom.comp (join cl T) (liftCl_mapId cl ...) := Hom.ext rfl
\end{leancode}

\begin{mathinsight}
三つのモナド法則がすべて \texttt{Hom.ext rfl} で閉じる。
$\eta$, $\mu$ の基底写像が $\id$ であるため、
任意の合成の基底写像も $\id$ となり、
等式は型レベルの整合性チェックに帰着する。

\medskip
非自明な内容は $\eta$ と $\mu$ の\textbf{構成}(preserves の証明)にあり、
法則の\textbf{証明}自体は自明になる。
これは「正しく構成すれば法則は自動的に成り立つ」という
依存型理論の特質を示す好例である。
\end{mathinsight}

\begin{figure}[H]
\centering
\begin{tikzcd}[column sep=5em, row sep=4em]
  T \arrow[r, "\eta_T"] \arrow[dr, equal]
  & \cl(T) \arrow[d, "\mu"]
  &
  \cl(T) \arrow[r, "\cl(\eta_T)"] \arrow[dr, equal]
  & \cl^2(T) \arrow[d, "\mu"]
  &
  \cl^3(T) \arrow[r, "\cl(\mu)"] \arrow[d, "\mu_{\cl(T)}"']
  & \cl^2(T) \arrow[d, "\mu"]
  \\
  & \cl(T)
  &
  & \cl(T)
  &
  \cl^2(T) \arrow[r, "\mu"']
  & \cl(T)
\end{tikzcd}
\caption{モナド法則: 左単位律(左)・右単位律(中)・結合律(右)}
\label{fig:monad-laws}
\end{figure}

\begin{correspondence}
\centering
\begin{tabular}{@{}lll@{}}
\toprule
\textbf{閉包公理} & \textbf{モナド公理} & \textbf{証明の核} \\
\midrule
拡大性 $A \subseteq \cl(A)$
  & $\eta : T \to F(T)$            & \texttt{cl.le\_closure} \\
冪等性 $\cl^2 = \cl$
  & $\mu : F^2(T) \to F(T)$        & \texttt{cl.idempotent} \\
単調性 $A \subseteq B \Rightarrow \cl A \subseteq \cl B$
  & $F$ は関手                      & \texttt{cl.monotone} \\
(自明)& 左単位律 $\mu \circ \eta_F = \id$ & \texttt{Hom.ext rfl} \\
(自明)& 右単位律 $\mu \circ F\eta = \id$  & \texttt{Hom.ext rfl} \\
(自明)& 結合律 $\mu \circ F\mu = \mu \circ \mu_F$ & \texttt{Hom.ext rfl} \\
\bottomrule
\end{tabular}
\end{correspondence}

\subsection{Kleisli 射}\label{sec:kleisli}

\begin{definition}[Kleisli 射]
構造塔 $T_1, T_2$ に対し、\textbf{Kleisli 射}を
\[
  T_1 \xrightarrow{\Kl} T_2 := \Hom(T_1,\; \liftCl(\cl, T_2))
\]
で定義する。Kleisli 恒等射は $\eta_T$、Kleisli 合成は
\[
  g \circ_\Kl f := \mu_{T_3} \circ g \circ f
  \qquad (f : T_1 \xrightarrow{\Kl} T_2,\; g : T_2 \xrightarrow{\Kl} T_3)
\]
\end{definition}

\begin{figure}[H]
\centering
\begin{tikzcd}[column sep=large, row sep=large]
  T_1
    \arrow[r, "f"]
    \arrow[rrr, bend right=30, "{g \circ_\Kl f}"']
  & \cl(T_2) \arrow[r, "g"]
  & \cl(\cl(T_3)) \arrow[r, "\mu_{T_3}"]
  & \cl(T_3)
\end{tikzcd}
\caption{Kleisli 合成: $g \circ_\Kl f = \mu \circ g \circ f$}
\label{fig:kleisli}
\end{figure}

\begin{remark}
一般の Kleisli 合成には $\cl$ と $g$ の可換性(naturality)が必要である。
本形式化では $\mathrm{toFun} = \id$ の特殊ケースに限定して構成しており、
この場合に限り冪等性だけで合成が定義できる。
\end{remark}

\subsection{Eilenberg--Moore 代数}\label{sec:em}

\begin{definition}[閉元の塔(ClosedTower)]\label{def:closed-tower}
閉包作用素 $\cl$ に対し、\textbf{閉元の塔}とは
\[
  \forall\, i \in \iota,\quad \cl(L(i)) = L(i)
\]
を満たす構造塔 $T$ である。各レベルが $\cl$-不動点(閉元)になっている。
\end{definition}

\begin{leancode}
structure ClosedTower (cl : ClosureOperator (Set α))
    (ι : Type*) [Preorder ι]
    extends StructureTower ι α where
  level_closed : ∀ i, cl (level i) = level i
\end{leancode}

\begin{theorem}[EM 代数と閉元の塔の等価性]\label{thm:em-equiv}
モナド $(\liftCl, \eta, \mu)$ の Eilenberg--Moore 代数は
閉元の塔と正確に対応する:
\begin{enumerate}[label=(\roman*)]
  \item 閉元の塔 $T$ は $\liftCl$ の不動点: $\liftCl(\cl, T) = T$。
  \item 構造射 $a : \liftCl(T) \to T$($a \circ \eta = \id$, $a_{\mathrm{fun}} = \id$)
        が存在する塔は閉元の塔である。
\end{enumerate}
\end{theorem}

\begin{figure}[H]
\centering
\begin{tikzcd}[column sep=5em, row sep=4em]
  T \arrow[r, "\eta"] \arrow[dr, equal]
  & \cl(T) \arrow[d, "a"]
  &
  \cl^2(T) \arrow[r, "\cl(a)"] \arrow[d, "\mu"']
  & \cl(T) \arrow[d, "a"]
  \\
  & T
  &
  \cl(T) \arrow[r, "a"']
  & T
\end{tikzcd}
\caption{EM 代数の公理: $a \circ \eta = \id$(左)と $a \circ \mu = a \circ \cl(a)$(右)}
\label{fig:em-algebra}
\end{figure}

\begin{proofstrategy}
(i) $\liftCl(T).L(i) = \cl(L(i)) = L(i)$ は \texttt{level\_closed} から直接。

(ii) $a.\mathrm{toFun} = \id$ かつ $a.\mathrm{preserves}$ より
$\cl(L(i)) \subseteq L(i)$、
拡大性 $L(i) \subseteq \cl(L(i))$ と合わせて
$\mathrm{Set.Subset.antisymm}$ で等号を得る。
\end{proofstrategy}

\begin{proposition}[閉元の global は閉集合]
閉元の塔 $T$ の大域切断は $\cl$-閉集合:
$\cl(\globalf(T)) \subseteq \globalf(T)$。
\end{proposition}

\begin{proof}
$x \in \cl(\globalf(T))$ とする。任意の $i$ に対し
$\globalf(T) \subseteq L(i)$ より単調性から $\cl(\globalf(T)) \subseteq \cl(L(i)) = L(i)$。
これが全 $i$ で成り立つから $x \in \bigcap_i L(i) = \globalf(T)$。
\end{proof}

% =============================================================================
\section{Level~3+: 具体的閉包作用素への接地}\label{sec:grounding}
% =============================================================================

Level~3 で構築した抽象 API を、2つの具体的な数学的文脈に接地する。
同一のインターフェース($\liftCl$, $\eta$, $\mathrm{algebra}$, $\cl\_\globalf\_\mathrm{subset}$)
が位相空間と群論の両方で機能することを確認する。

\subsection{位相的閉包 $\overline{(-)}$ への接地}\label{sec:grounding-top}

位相空間 $\alpha$ における閉包 $\overline{(-)} : \mathcal{P}(\alpha) \to \mathcal{P}(\alpha)$ は
$\mathrm{ClosureOperator}$ の典型例である。

\begin{center}
\begin{tabular}{@{}ll@{}}
\toprule
$\mathrm{ClosureOperator}$ の公理 & Mathlib の補題 \\
\midrule
拡大性 & \texttt{subset\_closure} \\
単調性 & \texttt{closure\_mono} \\
冪等性 & \texttt{isClosed\_closure.closure\_eq} \\
\bottomrule
\end{tabular}
\end{center}

\begin{leancode}
noncomputable def topClosure : ClosureOperator (Set α) where
  toFun := closure
  monotone' := fun _ _ h => closure_mono h
  le_closure' := fun _ => subset_closure
  idempotent' := fun _ => isClosed_closure.closure_eq
\end{leancode}

\begin{theorem}[位相的不動点の特徴づけ]
\[
  \text{$S$ が位相的閉集合} \iff \overline{S} = S
  \iff \text{$S$ は \texttt{topClosure} の不動点}
\]
\end{theorem}

\begin{groundingbox}
\textbf{接地の帰結}:
$\mathrm{ClosedTower}(\mathrm{topClosure}, \iota)$
は「各レベルが位相的閉集合である塔」に一致する。

\medskip
具体的には:
\begin{itemize}
  \item $\liftCl(\mathrm{topClosure}, T).L(i) = \overline{L(i)}$:
    各レベルを位相的に閉じた塔
  \item $\eta_T : T \to \liftCl(T)$:
    $L(i) \subseteq \overline{L(i)}$ という自然な包含
  \item $\mathrm{algebra}(T) : \liftCl(T) \to T$:
    $\overline{L(i)} = L(i)$ なので恒等写像が使える
  \item $\cl\_\globalf\_\mathrm{subset}$:
    閉集合の族の共通部分は閉集合(の一方向)
\end{itemize}
\end{groundingbox}

\begin{leancode}
-- 閉集合の塔から ClosedTower を構成する
def closedSetTower (T : StructureTower ι α)
    (hclosed : ∀ i, IsClosed (T.level i)) :
    ClosedTower topClosure ι where
  toStructureTower := T
  level_closed := by
    intro i
    exact (hclosed i).closure_eq  -- IsClosed S → closure S = S

-- ClosedTower の global は閉集合
theorem closedTower_global_isClosed
    (T : ClosedTower (topClosure) ι) :
    IsClosed T.global :=
  (isClosed_iff_topClosure_fixed T.global).2
    (Subset.antisymm
      (ClosedTower.cl_global_subset T)
      (topClosure.le_closure T.global))
\end{leancode}

\subsection{部分群生成 $\langle - \rangle$ への接地}\label{sec:grounding-group}

群 $G$ における部分群生成 $\mathrm{Subgroup.closure}$ を
$\mathcal{P}(G) \to \mathcal{P}(G)$ に持ち上げた操作も閉包作用素をなす。

\begin{center}
\begin{tabular}{@{}ll@{}}
\toprule
$\mathrm{ClosureOperator}$ の公理 & Mathlib の補題 \\
\midrule
拡大性 & \texttt{Subgroup.subset\_closure} \\
単調性 & \texttt{Subgroup.closure\_mono} \\
冪等性 & \texttt{Subgroup.closure\_eq} \\
\bottomrule
\end{tabular}
\end{center}

\begin{leancode}
def subgroupClosure : ClosureOperator (Set G) where
  toFun := fun S => (Subgroup.closure S : Set G)
  monotone' := by
    intro S T h
    exact SetLike.coe_subset_coe.mpr (Subgroup.closure_mono h)
  le_closure' := fun S => Subgroup.subset_closure
  idempotent' := fun S =>
    congr_arg SetLike.coe (Subgroup.closure_eq (Subgroup.closure S))
\end{leancode}

\begin{theorem}[部分群の特徴づけ]
\[
  \exists H : \mathrm{Subgroup}\; G,\; (H : \mathrm{Set}\; G) = S
  \iff
  \mathrm{subgroupClosure}(S) = S
\]
\end{theorem}

\begin{groundingbox}
\textbf{接地の帰結}:
$\mathrm{ClosedTower}(\mathrm{subgroupClosure}, \iota)$
は「各レベルが部分群の台集合である塔」に一致する。

\medskip
具体的には:
\begin{itemize}
  \item 単調な部分群族 $H : \iota \to \mathrm{Subgroup}\; G$
    から $\mathrm{ClosedTower}$ が自動的に構成される
    (\texttt{subgroupTower})
  \item フィルタード群(各レベルが $1 \in L(i)$, 積で閉, 逆元で閉)
    からも $\mathrm{ClosedTower}$ が構成される
    (\texttt{filteredGroupTower})
  \item $\cl\_\globalf\_\mathrm{subset}$:
    部分群の族の共通部分は部分群(の一方向)
\end{itemize}
\end{groundingbox}

\begin{leancode}
-- 単調な部分群族から ClosedTower を構成する
def subgroupTower (H : ι → Subgroup G)
    (hmono : ∀ ⦃i j : ι⦄, i ≤ j → H i ≤ H j) :
    ClosedTower subgroupClosure ι where
  level := fun i => (H i : Set G)
  monotone_level := by intro i j hij x hx; exact hmono hij hx
  level_closed := by
    intro i
    exact (isSubgroupCarrier_iff_fixed _).1 ⟨H i, rfl⟩

-- 部分群塔の global は部分群の台集合
theorem closedTower_global_isSubgroup
    (T : ClosedTower subgroupClosure ι) :
    ∃ H : Subgroup G, (H : Set G) = T.global :=
  (isSubgroupCarrier_iff_fixed T.global).2
    (Subset.antisymm
      (ClosedTower.cl_global_subset T)
      (subgroupClosure.le_closure T.global))
\end{leancode}

\subsection{統合: 同一 API の実証}\label{sec:grounding-synthesis}

Level~3 の抽象 API が、位相と代数の両方で同じ形で機能する。

\begin{correspondence}
\centering
\begin{tabular}{@{}p{3.2cm}p{4cm}p{4cm}@{}}
\toprule
\textbf{抽象 API} & \textbf{位相的解釈} & \textbf{代数的解釈} \\
\midrule
$\cl : \mathrm{ClosureOperator}$
  & $\overline{(-)}$(位相的閉包)
  & $\langle - \rangle$(生成部分群)\\[4pt]
$\cl(S) = S$(不動点)
  & $S$ が閉集合
  & $S = \uparrow H$ for some $H$\\[4pt]
$\mathrm{ClosedTower}$
  & 閉集合の単調族
  & 部分群の単調族\\[4pt]
$\liftCl(T).L(i)$
  & $\overline{L(i)}$
  & $\langle L(i) \rangle$\\[4pt]
$\eta : T \to \liftCl(T)$
  & $L(i) \subseteq \overline{L(i)}$
  & $L(i) \subseteq \langle L(i) \rangle$\\[4pt]
$\mathrm{algebra} : \liftCl(T) \to T$
  & $\overline{L(i)} = L(i)$ なので $\id$
  & $\langle L(i) \rangle = L(i)$ なので $\id$\\[4pt]
$\cl\_\globalf\_\mathrm{subset}$
  & $\overline{\bigcap_i L(i)} \subseteq \bigcap_i L(i)$
  & $\langle \bigcap_i L(i) \rangle \subseteq \bigcap_i L(i)$\\
\bottomrule
\end{tabular}
\end{correspondence}

\begin{figure}[H]
\centering
\begin{tikzcd}[column sep=4em, row sep=3em]
  |[draw,rounded corners]|{\text{ClosureOperator} \; \cl}
    \arrow[r, "\text{topClosure}"]
    \arrow[d, "\text{subgroupClosure}"']
  & |[draw,rounded corners]|{\text{閉集合の塔}}
  \\
  |[draw,rounded corners]|{\text{部分群の塔}}
  &
  |[draw,rounded corners,dashed]|{\text{ClosedTower}}
    \arrow[ul, dashed]
    \arrow[l, dashed]
\end{tikzcd}
\caption{同一の \texttt{ClosedTower} API を共有する2つの接地}
\label{fig:grounding}
\end{figure}

% =============================================================================
\section{全体像と展望}\label{sec:conclusion}
% =============================================================================

\subsection{三層の圏論的構造}

本稿で構築した構造を整理すると以下の階層が浮かび上がる。

\begin{figure}[H]
\centering
\begin{tikzcd}[row sep=4em, column sep=tiny]
  & |[draw,fill=violet!10,rounded corners]|{\textsf{Level~3: モナド}}
    \arrow[dl] \arrow[dr] & \\
  |[draw,fill=violet!10,rounded corners]|{%
    \parbox{3.2cm}{\centering\textsf{Kleisli 圏}\\[-0.2em]
    \small $T_1 \xrightarrow{\Kl} T_2$}}
  & &
  |[draw,fill=violet!10,rounded corners]|{%
    \parbox{3.2cm}{\centering\textsf{EM 代数}\\[-0.2em]
    \small 閉元の塔}} \\[1em]
  & |[draw,fill=blue!10,rounded corners]|{\textsf{Level~2: 関手・極限}}
    \arrow[dl] \arrow[d] \arrow[dr] & \\
  |[draw,fill=blue!10,rounded corners]|{%
    \parbox{2.5cm}{\centering\textsf{層関手}\\[-0.2em]
    \small $\mathrm{Ev}_i$}}
  & |[draw,fill=blue!10,rounded corners]|{%
    \parbox{2.5cm}{\centering\textsf{直積}\\[-0.2em]
    \small 普遍性}}
  & |[draw,fill=blue!10,rounded corners]|{%
    \parbox{3cm}{\centering\textsf{随伴の萌芽}\\[-0.2em]
    \small $\mathrm{const} \dashv \globalf$}} \\[1em]
  & |[draw,fill=green!10,rounded corners]|{\textsf{Level~1: 圏の公理}}
    \arrow[dl] \arrow[d] \arrow[dr] & \\
  |[draw,fill=green!10,rounded corners]|{%
    \parbox{2cm}{\centering\textsf{map}\\[-0.2em]
    \small 共変}}
  & |[draw,fill=green!10,rounded corners]|{%
    \parbox{2cm}{\centering\textsf{comap}\\[-0.2em]
    \small 反変}}
  & |[draw,fill=green!10,rounded corners]|{%
    \parbox{2.5cm}{\centering\textsf{reindex}\\[-0.2em]
    \small 添字変換}}
\end{tikzcd}
\caption{構造塔の圏論的構造の階層}
\label{fig:hierarchy}
\end{figure}

\subsection{冪等モナドの意義}

閉包作用素から誘導されるモナドが\textbf{冪等モナド}である事実は、
次の一連の帰結をもたらす:

\begin{enumerate}[label=(\roman*)]
  \item $\mu$ が $\eta$ の逆射となり $F^2(T) \cong F(T)$
  \item モナド法則が型レベルの整合性に帰着(\texttt{Hom.ext rfl})
  \item EM 代数が「不動点」という明快な特徴づけを得る
  \item Kleisli 圏が比較的単純な構造を持つ
\end{enumerate}

特に (iii) により、「閉集合の塔」「部分群の塔」という具体的な数学的対象が、
圏論的には EM 代数として統一的に扱われることが示される。

\subsection{今後の方向性}

\begin{description}
  \item[\textbf{$\sigma$-代数への接地}]
    $\mathrm{MeasurableSpace}$ への第3の接地。
    $\sigma$-代数生成 $\sigma(\cdot)$ は閉包作用素であり、
    同一の API が可測論にも機能することを確認する。

  \item[\textbf{Mathlib Category との接続}]
    正式な \texttt{Category} インスタンスの定義と
    \texttt{CategoryTheory.Monad} への登録。
    本形式化は圏論の公理を手動で検証したが、
    Mathlib の圏論フレームワークへの統合が次のステップとなる。

  \item[\textbf{$\mathrm{toFun} \neq \id$ の Kleisli 合成}]
    一般の Kleisli 合成には $\cl$ と $g$ の naturality が必要である。
    この条件を形式化し、どのクラスの閉包作用素で一般的な Kleisli 圏が
    得られるかを探求する。

  \item[\textbf{Hom の 2-圏的構造}]
    Hom 集合の間の順序構造(塔の包含による半順序)から
    enriched 圏・2-圏的構造への発展。
    自然変換の間の修正(modification)に相当する構造が得られると期待される。

  \item[\textbf{comap $\dashv$ map の随伴}]
    \texttt{CategoryTheory.Adjunction} を用いた
    $\mathrm{comap}\; f \dashv \mathrm{map}\; f$ の形式化。
    これはガロア接続の圏論的定式化に対応する。
\end{description}

% =============================================================================
\end{document}
