% =============================================================================
% Structure Tower 発展演習 Level 5
% I-adic フィルトレーションの形式化:Canonical Example の構成
% LuaLaTeX + ltjsarticle
% =============================================================================
\documentclass[a4paper,11pt]{ltjsarticle}

% --- Core packages ---
\usepackage{luatexja-fontspec}
\usepackage{amsmath,amssymb,amsthm}
\usepackage{mathtools}
\usepackage{tikz-cd}
\usepackage{tikz}
\usetikzlibrary{arrows.meta,positioning,calc,shapes.geometric,decorations.pathmorphing}
\usepackage{enumitem}
\usepackage{hyperref}
\usepackage{tcolorbox}
\tcbuselibrary{skins,breakable}
\usepackage{listings}
\usepackage{xcolor}
\usepackage{geometry}
\usepackage{fancyhdr}
\usepackage{float}
\usepackage{booktabs}
\usepackage{array}

% --- Page geometry ---
\geometry{margin=2.5cm, top=3cm, bottom=3cm}

% --- Fonts ---
\setmainjfont{Harano Aji Mincho}[
  BoldFont = Harano Aji Mincho,
  BoldFeatures = {FakeBold=2},
]
\setsansjfont{Harano Aji Gothic}
\setmonofont[Scale=0.85]{DejaVu Sans Mono}

% --- Colors ---
\definecolor{leanblue}{HTML}{2563EB}
\definecolor{leanbg}{HTML}{F8FAFC}
\definecolor{leancomment}{HTML}{6B7280}
\definecolor{leankeyword}{HTML}{7C3AED}
\definecolor{leanstring}{HTML}{059669}
\definecolor{accentcolor}{HTML}{1E40AF}
\definecolor{theoremcolor}{HTML}{EFF6FF}
\definecolor{definitioncolor}{HTML}{F0FDF4}
\definecolor{remarkcolor}{HTML}{FFF7ED}
\definecolor{warningcolor}{HTML}{FEF3C7}
\definecolor{iadiccolor}{HTML}{FDF4FF}

% --- Hyperref ---
\hypersetup{
  colorlinks=true,
  linkcolor=accentcolor,
  urlcolor=leanblue,
  citecolor=accentcolor,
  pdftitle={Structure Tower 発展演習 Level 5},
  pdfauthor={su},
}

% --- Theorem environments ---
\theoremstyle{definition}
\newtheorem{definition}{定義}[section]
\newtheorem{example}[definition]{例}
\newtheorem{construction}[definition]{構成}

\theoremstyle{plain}
\newtheorem{theorem}[definition]{定理}
\newtheorem{lemma}[definition]{補題}
\newtheorem{proposition}[definition]{命題}
\newtheorem{corollary}[definition]{系}

\theoremstyle{remark}
\newtheorem{remark}[definition]{注意}

% --- Lean code listing environment ---
\lstdefinelanguage{lean4}{
  morekeywords={def,theorem,lemma,example,instance,class,structure,
    where,let,in,do,if,then,else,match,with,fun,return,
    import,open,namespace,section,end,variable,
    inductive,abbrev,noncomputable,private,protected,
    sorry,have,show,suffices,calc,by,exact,
    apply,intro,intros,constructor,cases,induction,
    simp,rw,rfl,ext,ring,linarith,omega,
    decide,trivial,assumption,contradiction,push_neg,
    Type,Prop,Sort,Set,true,false,
    attribute,deriving,refine,change,congr,simpa,
    funext,positivity,norm_num,rcases,obtain,classical},
  sensitive=true,
  morecomment=[l]{--},
  morecomment=[n]{/-}{-/},
  morestring=[b]",
  literate=
    {alpha}{{\(\alpha\)}}1
    {beta}{{\(\beta\)}}1
    {gamma}{{\(\gamma\)}}1
    {iota}{{\(\iota\)}}1
    {->}{{\(\to\)}}2
    {<-}{{\(\leftarrow\)}}2
    {<->}{{\(\leftrightarrow\)}}3
    {<=}{{\(\leq\)}}2
    {>=}{{\(\geq\)}}2
    {/=}{{\(\neq\)}}2
    {forall}{{\(\forall\)}}6
    {exists}{{\(\exists\)}}6
    {/\\}{{\(\wedge\)}}2
    {\\/}{{\(\vee\)}}2
    {~}{{\(\neg\)}}1
    {elem}{{\(\in\)}}4
    {subset}{{\(\subseteq\)}}6
    {cap}{{\(\cap\)}}3
    {cup}{{\(\cup\)}}3
    {empty}{{\(\emptyset\)}}5
    {bot_}{{\(\bot\)}}4
    {top_}{{\(\top\)}}4,
}

\lstnewenvironment{leancode}[1][]{%
  \lstset{
    language=lean4,
    basicstyle=\ttfamily\small,
    keywordstyle=\color{leankeyword}\bfseries,
    commentstyle=\color{leancomment}\itshape,
    stringstyle=\color{leanstring},
    backgroundcolor=\color{leanbg},
    frame=single,
    rulecolor=\color{leanblue!30},
    framesep=8pt,
    xleftmargin=12pt,
    xrightmargin=12pt,
    breaklines=true,
    breakatwhitespace=true,
    showstringspaces=false,
    tabsize=2,
    captionpos=b,
    numbers=left,
    numberstyle=\tiny\color{leancomment},
    numbersep=8pt,
    aboveskip=1em,
    belowskip=1em,
    #1
  }%
}{}

% --- Tcolorbox styles ---
\newtcolorbox{proofstrategy}{
  colback=remarkcolor,
  colframe=orange!60!black,
  title={\textbf{証明戦略}},
  fonttitle=\sffamily,
  boxrule=0.5pt,
  arc=3pt,
}

\newtcolorbox{mathinsight}{
  colback=theoremcolor,
  colframe=accentcolor,
  title={\textbf{数学的洞察}},
  fonttitle=\sffamily,
  boxrule=0.5pt,
  arc=3pt,
}

\newtcolorbox{iadicsummary}{
  colback=iadiccolor,
  colframe=violet!60!black,
  title={\textbf{I-adic Tower の全体像}},
  fonttitle=\sffamily,
  boxrule=0.5pt,
  arc=3pt,
}

% --- Header / Footer ---
\pagestyle{fancy}
\fancyhf{}
\renewcommand{\headrulewidth}{0.4pt}
\fancyhead[L]{\small\sffamily\nouppercase{\leftmark}}
\fancyhead[R]{\small\sffamily Structure Tower Level 5}
\fancyfoot[C]{\thepage}

% --- Math macros ---
\newcommand{\N}{\mathbb{N}}
\newcommand{\Nop}{\mathbb{N}^{\mathrm{op}}}
\newcommand{\R}{\mathbb{R}}
\newcommand{\cl}{\mathrm{cl}}
\newcommand{\iinf}{\widehat{I}}
\newcommand{\glb}{\mathrm{glb}}
\newcommand{\levelop}{\mathrm{level}}
\newcommand{\Hom}{\mathrm{Hom}}
\newcommand{\liftCl}{\mathrm{liftCl}}
\newcommand{\idealspan}[1]{\langle #1 \rangle}
\newcommand{\ST}[2]{\mathrm{ST}(#1,\,#2)}
\DeclareMathOperator{\iInf}{\bigcap}

% =============================================================================
\begin{document}
% =============================================================================

% --- Title ---
\title{%
  {\LARGE\sffamily\bfseries Structure Tower 発展演習 Level 5}\\[0.5em]
  {\large\sffamily I-adic フィルトレーションの形式化:\\[0.2em]
  L1--L4 の全構造が収束する Canonical Example}\\[0.3em]
  {\normalsize\sffamily Lean 4 / Mathlib4 による形式化とその数学的解説}%
}
\author{su}
\date{\today}
\maketitle

% --- AI Disclosure ---
\begin{center}
\small
\textit{AI assistance disclosure:}\\
Lean ソースコードは Claude (Anthropic) で骨格を生成し、
Codex (OpenAI) で修正した。\\
\TeX{} 文書は Claude Code (Anthropic) で生成した。\\
著者による加筆・修正は行っていない。\\
内容の正確性は保証されず、誤りがあれば著者の責任である。
\end{center}

\vspace{1em}

% --- Abstract ---
\begin{abstract}
本稿は Lean 4 / Mathlib4 による Structure Tower 発展演習(Level 5)の形式化を数学的に解説する。
Level 1--4 で構築した StructureTower・ClosedTower・ClosureOperator の理論を基盤として、
可換環論の中核的構成である \textbf{$I$-adic フィルトレーション}を一つの標準例(canonical example)として統合する。
具体的には以下を形式化する:
(1)イデアル冪族 $I^0 \supseteq I^1 \supseteq I^2 \supseteq \cdots$ を $\Nop$ 添字の StructureTower として構成する(双対化パターン);
(2)イデアル生成 $S \mapsto \sqrt[\mathrm{ideal}]{S} = \langle S \rangle_R$ が ClosureOperator をなし,各レベル $I^n$ はその不動点となることを示す;
(3)環準同型 $\varphi \colon R \to S$ が条件 $\varphi(I) \subseteq J$ を満たすとき,$\varphi$ が塔の射 $\mathrm{idealPowTower}(I) \to \mathrm{idealPowTower}(J)$ を誘導することを確認する;
(4)分離条件 $\bigcap_n I^n = 0$ と Krull の交叉定理を StructureTower の語彙で定式化する。
これにより,Level 1--4 で展開した一般理論が一つの代数的例に合流することを実証する。
\end{abstract}

\tableofcontents
\newpage

% =============================================================================
\section{はじめに:一般理論を一点に束ねる}
% =============================================================================

\subsection{Level 1--4 の振り返り}

Structure Tower の形式化プロジェクトは,以下の段階的発展をたどってきた。

\begin{itemize}
  \item \textbf{Level 1--2}:基本定義(\texttt{StructureTower ι α},単調族 \texttt{Hom},\texttt{union},\texttt{glb})と順序論的例(区間族,上集合族など)。
  \item \textbf{Level 3}:ClosureOperator による \texttt{liftCl} と \texttt{ClosedTower} の理論。閉包モナドの代数的側面(algebra Hom,Kleisli 合成)。
  \item \textbf{Level 4}:三分野の接地—位相(\texttt{topClosure}),部分群(\texttt{subgroupClosure}),$\sigma$-代数—を ClosedTower として統一的に記述。閉包演算子の比較順序と rank 一意性。
  \item \textbf{Level 5(本稿)}:$I$-adic フィルトレーションを StructureTower の枠組みに組み込み,L1--L4 の全構造が一点に収束することを示す。
\end{itemize}

\subsection{動機:canonical example の必要性}

一般的な数学理論において,抽象的な定義・定理の価値は,具体的な「標準例」によって初めて明確になる。
Structure Tower の理論も同様であり,Level 5 の目標は次の問いに答えることである:

\begin{center}
  \textit{「StructureTower・ClosedTower・Hom のすべての条件を同時に満たす最も自然な例は何か?」}
\end{center}

答えは可換環論の基本的構成,\textbf{$I$-adic フィルトレーション}である。

\begin{iadicsummary}
$I$-adic Tower \texttt{idealPowTower(I)} は同時に以下を満たす:
\begin{itemize}[nosep]
  \item \checkmark \ \textbf{StructureTower} $\Nop$ $R$($\Nop$ 添字の単調族)
  \item \checkmark \ \textbf{乗法互換性}($x \in I^m,\, y \in I^n \Rightarrow xy \in I^{m+n}$)
  \item \checkmark \ \textbf{ClosedTower}(各レベル $I^n$ は \texttt{idealClosure} の不動点)
  \item \checkmark \ \textbf{比較射}($I \subseteq J$ が塔の射を誘導)
  \item \checkmark \ \textbf{分離条件}($\bigcap_n I^n = 0$ のとき \texttt{glb} = $\{0\}$)
\end{itemize}
\end{iadicsummary}

% =============================================================================
\section{準備:StructureTower フレームワークの復習}
% =============================================================================

\subsection{基本定義}

\begin{definition}[StructureTower]
  前順序集合 $(\iota, \leq)$ と型 $\alpha$ に対し,\textbf{StructureTower} $(\iota, \alpha)$ とは
  写像 $\levelop \colon \iota \to \mathcal{P}(\alpha)$ であって,単調性
  \[
    i \leq j \;\Longrightarrow\; \levelop(i) \subseteq \levelop(j)
  \]
  を満たすものである。
\end{definition}

\begin{definition}[glb, union]
  StructureTower $T$ に対し,\textbf{glb} および \textbf{union} を
  \[
    \glb(T) \;:=\; \bigcap_{i \in \iota} \levelop_i, \qquad
    \mathrm{union}(T) \;:=\; \bigcup_{i \in \iota} \levelop_i
  \]
  と定める。
\end{definition}

\begin{definition}[Hom]
  二つの StructureTower $T_1 \colon \iota \to \mathcal{P}(\alpha)$ と $T_2 \colon \iota \to \mathcal{P}(\beta)$ の間の\textbf{射(Hom)}とは,関数 $f \colon \alpha \to \beta$ であって,任意の $i \in \iota$ に対して
  \[
    f\bigl(\levelop^{T_1}(i)\bigr) \;\subseteq\; \levelop^{T_2}(i)
  \]
  を満たすものである。
\end{definition}

\begin{definition}[ClosedTower]
  ClosureOperator $\cl$ と StructureTower $T$ に対し,$T$ が \textbf{$\cl$-ClosedTower} であるとは,任意の $i$ に対して
  \[
    \cl\bigl(\levelop(i)\bigr) \;=\; \levelop(i)
  \]
  が成立することである。つまり,各レベルが $\cl$ の不動点(閉集合)である。
\end{definition}

図~\ref{fig:closed-tower} に ClosedTower の構造を示す。

\begin{figure}[H]
  \centering
  \begin{tikzcd}[row sep=2em, column sep=3em]
    \mathcal{P}(\alpha)
      \arrow[r, bend left=20, "\cl"]
      \arrow[loop left, "\cl \circ \cl = \cl"]
    & \mathcal{P}(\alpha) \\
    \levelop(i)
      \arrow[r, phantom, "\ni x"]
      \arrow[u, hook]
    & \cl\bigl(\levelop(i)\bigr) = \levelop(i)
      \arrow[u, hook]
      \arrow[l, phantom, "\ni x"]
  \end{tikzcd}
  \caption{ClosedTower の条件:各レベル $\levelop(i)$ は $\cl$ の不動点}
  \label{fig:closed-tower}
\end{figure}

% =============================================================================
\section{\texorpdfstring{$\S$L5-1: $I$-adic Tower の基盤}{§L5-1: I-adic Tower の基盤}}
% =============================================================================

\subsection{\texorpdfstring{$\Nop$ 添字と双対化パターン}{ℕᵒᵈ 添字と双対化パターン}}

可換環 $R$ のイデアル $I$ に対する \textbf{$I$-adic フィルトレーション}は,
\[
  R = I^0 \;\supseteq\; I^1 = I \;\supseteq\; I^2 \;\supseteq\; I^3 \;\supseteq\; \cdots
\]
という\textbf{減少的}なイデアルの族である。一方,StructureTower の単調性条件は\textbf{増加的}($i \leq j \Rightarrow \levelop(i) \subseteq \levelop(j)$)を要求する。

この矛盾を解消するための標準的手法が「\textbf{双対化}」である。
$\N$ の順序を逆転した前順序 $\Nop$(順序双対)を添字として用いると:
\[
  a \leq b \;\text{ in }\;\Nop \;\iff\; b \leq a \;\text{ in }\;\N
\]
したがって $\Nop$ では $\cdots \leq 3 \leq 2 \leq 1 \leq 0$ という順序になる(図~\ref{fig:nop-dual})。

\begin{figure}[H]
  \centering
  \begin{tikzcd}[row sep=1.5em, column sep=2.5em]
    \text{in } \N: &
    0 \ar[r, "\leq"] & 1 \ar[r, "\leq"] & 2 \ar[r, "\leq"] & 3 \ar[r, "\leq"] & \cdots \\
    \text{in } \Nop: &
    0 \ar[r, leftarrow, "\leq"] & 1 \ar[r, leftarrow, "\leq"] & 2 \ar[r, leftarrow, "\leq"] & 3 \ar[r, leftarrow, "\leq"] & \cdots \\
    \levelop: &
    R \ar[r, hookleftarrow, "\supseteq"] & I \ar[r, hookleftarrow, "\supseteq"] & I^2 \ar[r, hookleftarrow, "\supseteq"] & I^3 \ar[r, hookleftarrow, "\supseteq"] & \cdots
  \end{tikzcd}
  \caption{$\Nop$ 添字による双対化:増加方向と減少族の整合}
  \label{fig:nop-dual}
\end{figure}

$\Nop$ の添字 $n$ に対して $\levelop(n) := I^n$(正確には $I^{\mathrm{ofDual}(n)}$)と定めると,
$i \leq j$ in $\Nop$ $\Rightarrow$ $\mathrm{ofDual}(j) \leq \mathrm{ofDual}(i)$ in $\N$ $\Rightarrow$ $I^{\mathrm{ofDual}(i)} \supseteq I^{\mathrm{ofDual}(j)}$,
つまり $\levelop(i) \subseteq \levelop(j)$ となり,単調性が成立する。

\subsection{\texorpdfstring{$\mathrm{idealPowTower}$ の構成}{idealPowTower の構成}}

\begin{construction}[$\mathrm{idealPowTower}$]
  可換環 $R$ のイデアル $I$ に対し,$\Nop$ 添字の StructureTower を
  \[
    \mathrm{idealPowTower}(I) \;:=\; \bigl(\Nop \to \mathcal{P}(R),\quad n \mapsto \bigl|I^{\mathrm{ofDual}(n)}\bigr|\bigr)
  \]
  と定める。ここで $|J|$ はイデアル $J$ の台集合(carrier)を表す。
\end{construction}

\begin{theorem}[単調性]
  $\mathrm{idealPowTower}(I)$ は well-defined な StructureTower,すなわち
  \[
    i \leq j \;\text{ in }\;\Nop \;\Longrightarrow\; \levelop(i) \subseteq \levelop(j).
  \]
\end{theorem}

\begin{proof}
  $i \leq j$ in $\Nop$ とすると,$\mathrm{ofDual}(j) \leq \mathrm{ofDual}(i)$ in $\N$。
  Mathlib の補題 \texttt{Ideal.pow\_le\_pow\_right}:$m \leq n \Rightarrow I^n \leq I^m$(大きい冪は小さいイデアル)を適用すると,
  $I^{\mathrm{ofDual}(i)} \geq I^{\mathrm{ofDual}(j)}$(イデアルとして),
  したがって台集合として $\levelop(i) = |I^{\mathrm{ofDual}(i)}| \subseteq |I^{\mathrm{ofDual}(j)}| = \levelop(j)$。
\end{proof}

対応する Lean 定義を以下に示す。

\begin{leancode}[caption={idealPowTower の Lean 定義(L5-1a)}]
def idealPowTower (I : Ideal R) : StructureTower N_op R where
  level n := (I ^ OrderDual.ofDual n : Ideal R)  -- carrier set
  monotone_level := by
    intro i j hij x hx
    exact (Ideal.pow_le_pow_right (I := I)
      (m := OrderDual.ofDual j) (n := OrderDual.ofDual i)
      (OrderDual.ofDual_le_ofDual.mpr hij)) hx
-- Note: i <= j in N_op  <->  ofDual j <= ofDual i in N
--       <->  I^(ofDual i) >= I^(ofDual j) as ideals
--       <->  level i  subset  level j  as sets
\end{leancode}

\begin{proposition}[最大レベル]
  $\mathrm{idealPowTower}(I)$ の「最大レベル」($\Nop$ の最大元 $0 = \mathrm{toDual}(0)$ に対応)は全体集合である:
  \[
    \levelop\bigl(\mathrm{toDual}(0)\bigr) = |I^0| = |R| = R.
  \]
\end{proposition}

\begin{proof}
  $I^0 = \top_R$(環の単元イデアル全体)であるから,その台集合は $R$ 全体。
\end{proof}

\subsection{乗法互換性}

$I$-adic フィルトレーションは乗法に関して以下の重要な互換性を持つ。

\begin{theorem}[乗法互換性:L5-1c]
  $x \in I^m$ かつ $y \in I^n$ ならば $xy \in I^{m+n}$。
\end{theorem}

\begin{proof}
  $x \in I^m,\, y \in I^n$ より $xy \in I^m \cdot I^n$。
  Mathlib の \texttt{Ideal.pow\_add}:$I^m \cdot I^n = I^{m+n}$ を用いると,$xy \in I^{m+n}$。

  Lean での証明:\texttt{simpa [pow\_add] using Ideal.mul\_mem\_mul hx hy}
\end{proof}

\begin{remark}
  この条件は,フィルタリングされた環(filtered ring)の公理 $F_m \cdot F_n \subseteq F_{m+n}$ そのものであり,
  $\mathrm{idealPowTower}(I)$ が FilteredRing の構造を持つことを意味する。
\end{remark}

\subsection{イデアル包含と比較射}

\begin{theorem}[NatInclusion:L5-1d]
  $I \subseteq J$(イデアルとして)ならば,任意の $n \in \Nop$ に対して
  \[
    \levelop^I(n) \;\subseteq\; \levelop^J(n), \quad \text{すなわち}\quad I^n \subseteq J^n.
  \]
  これを \texttt{NatInclusion}(levelwise な包含)と呼ぶ。
\end{theorem}

\begin{proof}
  \texttt{Ideal.pow\_right\_mono}:$I \leq J \Rightarrow I^n \leq J^n$ の直接適用。
\end{proof}

\begin{theorem}[比較射:L5-1e]
  $I \subseteq J$ のとき,恒等関数 $\mathrm{id}_R$ は塔の射 $\mathrm{Hom}(\mathrm{idealPowTower}(I),\, \mathrm{idealPowTower}(J))$ を与える。
\end{theorem}

図~\ref{fig:comparison-hom} に比較射の構造を示す。

\begin{figure}[H]
  \centering
  \begin{tikzcd}[row sep=2.5em, column sep=4em]
    \mathrm{idealPowTower}(I) \arrow[r, "\mathrm{id}_R", "\text{(NatInclusion)}"']
    & \mathrm{idealPowTower}(J) \\
    I^n \arrow[r, hook, "\subseteq"] \arrow[u, hook]
    & J^n \arrow[u, hook]
  \end{tikzcd}
  \caption{$I \subseteq J$ による比較射:各レベルで $I^n \subseteq J^n$}
  \label{fig:comparison-hom}
\end{figure}

% =============================================================================
\section{\texorpdfstring{$\S$L5-2: idealClosure と ClosedTower}{§L5-2: idealClosure と ClosedTower}}
% =============================================================================

\subsection{ClosureOperator としての idealClosure}

\begin{definition}[idealClosure]
  可換環 $R$ に対し,$\mathcal{P}(R)$ 上の\textbf{イデアル生成閉包}を
  \[
    \mathrm{idealClosure}(S) \;:=\; \bigl|\,\langle S \rangle_R\,\bigr| \;=\; |\mathrm{Ideal.span}(S)|
  \]
  と定める。ここで $\langle S \rangle_R$ は集合 $S$ が生成するイデアルである。
\end{definition}

\begin{theorem}[ClosureOperator の公理:L5-2a]
  $\mathrm{idealClosure}$ は $\mathcal{P}(R)$ 上の ClosureOperator である,すなわち:
  \begin{enumerate}[label=(\roman*)]
    \item \textbf{拡大性}:$S \subseteq \mathrm{idealClosure}(S)$(\texttt{Ideal.subset\_span})
    \item \textbf{単調性}:$S \subseteq T \Rightarrow \mathrm{idealClosure}(S) \subseteq \mathrm{idealClosure}(T)$(\texttt{Ideal.span\_mono})
    \item \textbf{冪等性}:$\mathrm{idealClosure}(\mathrm{idealClosure}(S)) = \mathrm{idealClosure}(S)$(\texttt{Ideal.span\_eq})
  \end{enumerate}
\end{theorem}

\begin{proof}
  各公理について:
  \begin{enumerate}[label=(\roman*)]
    \item $\mathrm{Ideal.subset\_span}$ が $S \subseteq |\langle S \rangle|$ を与える。
    \item $\mathrm{Ideal.span\_mono}$:$S \subseteq T$ ならば $\langle S \rangle \leq \langle T \rangle$,したがって台集合も包含。
    \item $\langle S \rangle$ はイデアルであるから $\langle |\langle S \rangle| \rangle = \langle S \rangle$(\texttt{Ideal.span\_eq}),よって台集合が一致。
  \end{enumerate}
\end{proof}

図~\ref{fig:closure-axioms} に三公理の関係を示す。

\begin{figure}[H]
  \centering
  \begin{tikzcd}[row sep=2em, column sep=3em]
    S
      \arrow[r, hook, "\text{拡大性}"]
      \arrow[dr, hook, bend right=10, "\text{拡大性}"']
    & \langle S \rangle
      \arrow[r, "\mathrm{span}", bend left=20]
      \arrow[loop right, "\mathrm{id}\ (\text{冪等性})"]
    & \langle\langle S \rangle\rangle = \langle S \rangle \\
    T \supseteq S
      \arrow[r, hook, "\text{単調性}"']
    & \langle T \rangle \supseteq \langle S \rangle
  \end{tikzcd}
  \caption{ClosureOperator の三公理:idealClosure の場合}
  \label{fig:closure-axioms}
\end{figure}

\subsection{イデアルは不動点}

\begin{theorem}[不動点:L5-2b]
  $I$ が $R$ のイデアルならば,$I$ の台集合は $\mathrm{idealClosure}$ の不動点:
  \[
    \mathrm{idealClosure}(|I|) \;=\; |I|, \quad \text{すなわち}\quad |\langle I \rangle| = |I|.
  \]
\end{theorem}

\begin{proof}
  $I$ はすでにイデアルであるから $\langle I \rangle = I$(\texttt{Ideal.span\_eq})。
  したがって台集合として $|\langle I \rangle| = |I|$。
\end{proof}

\subsection{idealPowTower は ClosedTower}

上記の不動点定理を使うと,$I$-adic Tower が ClosedTower であることが直ちに従う。

\begin{theorem}[ClosedTower:L5-2c]
  $\mathrm{idealPowTower}(I)$ は $\mathrm{idealClosure}$ に関する ClosedTower である:
  \[
    \forall n \in \Nop,\quad \mathrm{idealClosure}\bigl(\levelop(n)\bigr) = \levelop(n).
  \]
\end{theorem}

\begin{proof}
  $\levelop(n) = |I^{\mathrm{ofDual}(n)}|$ はイデアル $I^{\mathrm{ofDual}(n)}$ の台集合であるから,
  補題(不動点)を適用すると $\mathrm{idealClosure}(\levelop(n)) = \levelop(n)$。
\end{proof}

\begin{leancode}[caption={idealPowTower\_closedTower の Lean 定義(L5-2c)}]
def idealPowTower_closedTower (I : Ideal R) :
    ClosedTower (idealClosure (R := R)) N_op where
  toStructureTower := idealPowTower I
  level_closed := by
    intro n
    -- level n = carrier of I^(ofDual n), which is an ideal
    -- => idealClosure fixes it (Ideal.span_eq)
    exact idealClosure_fixed_of_ideal (I ^ OrderDual.ofDual n)
\end{leancode}

\subsection{glb のイデアル性}

\begin{theorem}[glb はイデアル:L5-2e]
  $\mathrm{idealPowTower}(I)$ の glb は,イデアルの無限交叉 $\displaystyle\bigcap_{n \geq 0} I^n$ の台集合と一致する:
  \[
    \glb\bigl(\mathrm{idealPowTower}(I)\bigr)
    \;=\; \left|\;\bigcap_{n:\N} I^n\;\right|
    \;=\; \bigcap_{n \geq 0} |I^n|.
  \]
  特に,この集合はイデアル $\displaystyle\bigcap_{n:\N} I^n$ の台集合として環論的意味を持つ。
\end{theorem}

\begin{proof}
  Mathlib の \texttt{Submodule.coe\_iInf}(サブモジュールの iInf と台集合の iInter の整合)を使って示す:
  \[
    \glb(T) = \bigcap_{n:\Nop} \levelop(n) = \bigcap_{n:\Nop} |I^{\mathrm{ofDual}(n)}| = \bigcap_{m:\N} |I^m| = \left|\bigcap_{m:\N} I^m\right|.
  \]
\end{proof}

さらに,ClosedTower の一般定理から以下の系が得られる。

\begin{corollary}[glb の閉性:L5-2d]
  \[
    \mathrm{idealClosure}\bigl(\glb(\mathrm{idealPowTower}(I))\bigr) \;\subseteq\; \glb(\mathrm{idealPowTower}(I)).
  \]
  すなわち $\glb = \displaystyle\bigcap_n I^n$ はイデアル生成に関して閉じている。
\end{corollary}

\begin{proof}
  ClosedTower の一般定理 \texttt{ClosedTower.cl\_glb\_subset} の直接適用。
  この定理は「ClosedTower ならば $\cl(\glb(T)) \subseteq \glb(T)$」を述べる。
\end{proof}

% =============================================================================
\section{\texorpdfstring{$\S$L5-3: 環準同型と塔の射}{§L5-3: 環準同型と塔の射}}
% =============================================================================

\subsection{環準同型による誘導}

環準同型 $\varphi \colon R \to S$ が $\varphi(I) \subseteq J$ を満たすとき,
$\varphi$ は各レベルで $I^n$ を $J^n$ に写す。これを形式化する。

\begin{lemma}[冪の保存:L5-3a]
  環準同型 $\varphi \colon R \to S$ と $\varphi(I) \subseteq J$ を仮定する。
  このとき任意の $n \in \N$ に対して
  \[
    \varphi(I^n) \;\subseteq\; J^n.
  \]
\end{lemma}

\begin{proof}
  Mathlib の \texttt{Ideal.map\_pow}:$\varphi(I^n) = (\varphi(I))^n$ を使う。
  仮定 $\varphi(I) \subseteq J$ より $(\varphi(I))^n \subseteq J^n$(\texttt{Ideal.pow\_right\_mono})。

  Lean での証明:
  \begin{center}
    \texttt{rw [Ideal.map\_pow]; exact Ideal.pow\_right\_mono hIJ n}
  \end{center}
\end{proof}

\begin{theorem}[誘導された塔の射:L5-3b]
  環準同型 $\varphi \colon R \to S$ と $\varphi(I) \subseteq J$ に対し,$\varphi$ は
  \[
    \varphi \colon \mathrm{idealPowTower}(I) \;\to\; \mathrm{idealPowTower}(J)
  \]
  という StructureTower の射を誘導する。
\end{theorem}

\begin{proof}
  各 $n \in \Nop$ と $x \in \levelop^I(n) = |I^{\mathrm{ofDual}(n)}|$ に対して,
  $\varphi(x) \in \varphi(I^{\mathrm{ofDual}(n)}) \subseteq J^{\mathrm{ofDual}(n)} = \levelop^J(n)$ を示せばよい。
  これは補題(冪の保存)の直接的帰結。
\end{proof}

図~\ref{fig:ring-hom-tower} に環準同型と塔の射の関係を示す。

\begin{figure}[H]
  \centering
  \begin{tikzcd}[row sep=2.5em, column sep=5em]
    R \arrow[r, "\varphi"] \arrow[d, "\varphi(I) \subseteq J"'] & S \\
    \mathrm{idealPowTower}(I) \arrow[r, "\varphi", "\text{Hom}"']
    & \mathrm{idealPowTower}(J) \\
    I^n \arrow[r, hook, "\varphi(I^n) \subseteq J^n"'] \arrow[u, hook]
    & J^n \arrow[u, hook]
  \end{tikzcd}
  \caption{環準同型 $\varphi \colon R \to S$ が $\varphi(I) \subseteq J$ のとき塔の射を誘導する}
  \label{fig:ring-hom-tower}
\end{figure}

\subsection{恒等射と合成の整合性}

\begin{proposition}[恒等射:L5-3c]
  恒等準同型 $\mathrm{id}_R \colon R \to R$ は $\mathrm{idealPowTower}(I) \to \mathrm{idealPowTower}(I)$ の自明な塔の射を与える。
\end{proposition}

\begin{theorem}[合成の整合性:L5-3d]
  $\varphi \colon R \to S$,$\psi \colon S \to T$ とし,$\varphi(I) \subseteq J$,$\psi(J) \subseteq K$ とする。このとき
  \[
    \mathrm{Hom}(\psi,J,K) \circ \mathrm{Hom}(\varphi,I,J)
    \;=\;
    \mathrm{Hom}(\psi \circ \varphi,\, I,\, K)
  \]
  が \textbf{toFun} のレベルで成立する。
\end{theorem}

\begin{proof}
  両辺の \texttt{toFun} はいずれも $\psi \circ \varphi$ であるから,
  Lean では \texttt{Hom.ext rfl} で証明できる。
\end{proof}

図~\ref{fig:tower-comp} に合成の交換図式を示す。

\begin{figure}[H]
  \centering
  \begin{tikzcd}[row sep=2em, column sep=4em]
    \mathrm{idealPowTower}(I)
      \arrow[r, "\varphi"]
      \arrow[rr, bend right=25, "\psi \circ \varphi"']
    & \mathrm{idealPowTower}(J)
      \arrow[r, "\psi"]
    & \mathrm{idealPowTower}(K) \\
    I^n \arrow[r, hook] & J^n \arrow[r, hook] & K^n
  \end{tikzcd}
  \caption{塔の射の合成:$(\psi \circ \varphi)$ は合成射に等しい}
  \label{fig:tower-comp}
\end{figure}

% =============================================================================
\section{\texorpdfstring{$\S$L5-4: 分離条件と Krull の交叉定理}{§L5-4: 分離条件と Krull の交叉定理}}
% =============================================================================

\subsection{分離条件の定義と同値}

\begin{definition}[分離条件:L5-4b]
  イデアル $I$ の $I$-adic フィルトレーションが\textbf{分離的(separated)}であるとは,
  \[
    \mathrm{IsSeparated}(I) \;\;:\iff\;\; \bigcap_{n:\N} I^n \;=\; \bot \;\;\text{(零イデアル)}
  \]
  が成立することである。台集合の言語では:
  \[
    \bigcap_{n:\N} I^n = 0 \;\iff\; \bigcap_{n \geq 0} I^n = \{0\} \;\iff\; \glb\bigl(\mathrm{idealPowTower}(I)\bigr) = \{0\}.
  \]
\end{definition}

\begin{theorem}[分離条件の同値:L5-4b]
  \[
    \mathrm{IsSeparated}(I) \;\iff\; \glb\bigl(\mathrm{idealPowTower}(I)\bigr) = \{0\}.
  \]
\end{theorem}

\begin{proof}
  glb の定義と iInf の台集合が iInter に一致すること(\texttt{Submodule.coe\_iInf})を組み合わせることで,
  $\glb(\mathrm{idealPowTower}(I)) = |\bigcap_n I^n|$ が分かる。
  $\bigcap_n I^n = \bot$ iff $|\bigcap_n I^n| = \{0\}$(\texttt{Submodule.bot\_coe})であるから主張を得る。
\end{proof}

\subsection{「脱出定理」:非零元の有限段階脱出}

\begin{theorem}[脱出定理:L5-4c]
  $\mathrm{IsSeparated}(I)$ のとき,$x \neq 0$ ならば有限段階で $I$-adic 塔から「脱出」する:
  \[
    x \neq 0 \;\Longrightarrow\; \exists\, n \in \N,\; x \notin I^n.
  \]
\end{theorem}

\begin{proof}
  分離条件より $\glb = \{0\}$,すなわち $\bigcap_n I^n = \{0\}$。
  $x \neq 0$ より $x \notin \{0\} = \bigcap_n I^n$,
  したがって $\exists\, n,\; x \notin I^n$。
\end{proof}

\begin{mathinsight}
  この「脱出定理」は Level 4 の \texttt{EscapeExercises} における
  \texttt{SeparatedFilteredAddGroup.exists\_not\_mem\_of\_ne\_zero} と同じパターンである。
  StructureTower の語彙で統一的に述べると:
  \[
    \glb(T) = \{e\} \;\Longrightarrow\; \forall x \neq e,\; \exists\, i,\; x \notin \levelop(i).
  \]
\end{mathinsight}

\subsection{Krull の交叉定理}

\begin{theorem}[Krull の交叉定理:L5-4e]
  $R$ が Noetherian 可換環,$I$ がそのイデアルのとき,
  \[
    \bigcap_{n:\N} I^n \;\leq\; I \cdot \bigcap_{n:\N} I^n.
  \]
  StructureTower の語彙では:「glb は $I$ 倍で不変(のようなもの)」,
  すなわち $\glb(\mathrm{idealPowTower}(I)) \subseteq I \cdot \glb(\mathrm{idealPowTower}(I))$。
\end{theorem}

\begin{proof}[証明の概略]
  Mathlib の \texttt{Ideal.mem\_iInf\_smul\_pow\_eq\_bot\_iff} を使う。
  $x \in \bigcap_n I^n$ のとき,Noetherian 条件と Nakayama の補題の変種(Krull の交叉定理の本質)により,
  ある $r \in I$ が存在して $(1 - r) \cdot x = 0$,したがって $x = r \cdot x \in I \cdot (\bigcap_n I^n)$。
\end{proof}

\begin{corollary}[Jacobson radical 条件]
  $I$ が Jacobson radical $\mathrm{Jac}(R)$ に含まれるとき(特に $R$ が局所環で $I = \mathfrak{m}$ が極大イデアル),
  \[
    \bigcap_{n:\N} I^n = 0
  \]
  が成立する,すなわち $\mathrm{IsSeparated}(I)$。
\end{corollary}

図~\ref{fig:krull} に Krull の交叉定理の構造を示す。

\begin{figure}[H]
  \centering
  \begin{tikzcd}[row sep=2em, column sep=4em]
    R \supseteq I \supseteq I^2 \supseteq \cdots \\
    \displaystyle\bigcap_n I^n
      \arrow[u, hook, "\text{各 }I^n \text{ に含まれる}"]
      \arrow[r, hook, "\subseteq"]
    & I \cdot \displaystyle\bigcap_n I^n
      \arrow[ul, hook, "\text{Krull: }\leq I\cdot(-)"']
  \end{tikzcd}
  \caption{Krull の交叉定理:$\bigcap_n I^n$ は $I$ 倍しても変わらない}
  \label{fig:krull}
\end{figure}

% =============================================================================
\section{統合:4分野を貫く StructureTower}
% =============================================================================

Level 1--5 の全体を通じて,StructureTower の枠組みは以下の4分野を統一的に記述することが確認された。

\begin{table}[H]
  \centering
  \renewcommand{\arraystretch}{1.4}
  \begin{tabular}{lllll}
    \toprule
    \textbf{分野} & \textbf{添字} $\iota$ & \textbf{ClosureOp} & \textbf{分離条件} & \textbf{Grounding(L4)} \\
    \midrule
    位相(Topology)   & 開基族       & \texttt{topClosure}        & $\bigcap_n \overline{U_n} = \{x\}$ & L4-2 \\
    群論(Group)      & 正規部分群列  & \texttt{subgroupClosure}   & $\bigcap_n G_n = \{e\}$            & L4-3 \\
    可測論(Measur.)  & $\sigma$-代数 & ---                        & ---                                & L4-4($\sigma$-代数) \\
    可換環(Ring)     & $\Nop$(冪)  & \texttt{idealClosure}      & $\bigcap_n I^n = \{0\}$            & \textbf{L5(本稿)} \\
    \bottomrule
  \end{tabular}
  \caption{4分野における StructureTower の統一記述}
  \label{tab:unification}
\end{table}

\begin{mathinsight}
  4分野に共通するパターンは:
  \begin{enumerate}
    \item 各レベル $\levelop(n)$ は対応する閉包演算子の\textbf{不動点}(closed set, subgroup, ideal)。
    \item glb $= \bigcap_n \levelop(n)$ は同じ閉包演算子のもとで閉じている($\cl(\glb) \subseteq \glb$)。
    \item 分離条件(glb $= \{e\}$)が成立するとき,非自明な元は有限段階で「塔から脱出」する。
  \end{enumerate}
  これが StructureTower 理論の中核的な三段論法である。
\end{mathinsight}

図~\ref{fig:four-domains} に4分野の統合図式を示す。

\begin{figure}[H]
  \centering
  \begin{tikzpicture}[
    node distance = 2.5cm,
    fieldbox/.style = {rectangle, rounded corners, draw=accentcolor, fill=theoremcolor,
                     minimum width=3.5cm, minimum height=1.2cm, align=center, font=\small},
    center/.style = {ellipse, draw=violet!80!black, fill=iadiccolor,
                     minimum width=4cm, minimum height=1.5cm, align=center, font=\sffamily\bfseries},
    arrow/.style  = {-{Stealth[scale=1.2]}, thick, accentcolor}
  ]
    \node[center]                           (st)  {StructureTower\\ClosedTower};
    \node[fieldbox, above left  = of st]      (top) {位相\\topClosure};
    \node[fieldbox, above right = of st]      (grp) {群論\\subgroupClosure};
    \node[fieldbox, below left  = of st]      (msr) {可測論\\$\sigma$-代数};
    \node[fieldbox, below right = of st]      (rng) {\textbf{可換環}\\idealClosure};
    \draw[arrow] (top) -- (st) node[midway, above left,  font=\tiny] {ClosedTower};
    \draw[arrow] (grp) -- (st) node[midway, above right, font=\tiny] {ClosedTower};
    \draw[arrow] (msr) -- (st) node[midway, below left,  font=\tiny] {MeasurableTower};
    \draw[arrow] (rng) -- (st) node[midway, below right, font=\tiny] {ClosedTower (L5)};
  \end{tikzpicture}
  \caption{4分野の統合:StructureTower・ClosedTower が共通フレームワークを提供する}
  \label{fig:four-domains}
\end{figure}

% =============================================================================
\section{おわりに:Level 5 の成果と展望}
% =============================================================================

\subsection{Level 5 で確立したこと}

以下の表に,本稿(Level 5)で形式化した命題の一覧を示す。

\begin{table}[H]
  \centering
  \renewcommand{\arraystretch}{1.3}
  \begin{tabular}{lll}
    \toprule
    \textbf{Lean 識別子} & \textbf{内容} & \textbf{節} \\
    \midrule
    \texttt{idealPowTower}               & I-adic tower の StructureTower 構成   & \S3.2 \\
    \texttt{idealPowTower\_top\_level}   & 最大レベル $= R$                       & \S3.2 \\
    \texttt{idealPow\_mul\_mem}          & 乗法互換性 $xy \in I^{m+n}$            & \S3.3 \\
    \texttt{idealPowTower\_natInclusion} & $I \subseteq J \Rightarrow$ NatInclusion & \S3.4 \\
    \texttt{idealPowTower\_comparison}   & 比較射(Hom)の構成                   & \S3.4 \\
    \texttt{idealClosure}                & イデアル生成 ClosureOperator           & \S4.1 \\
    \texttt{idealClosure\_fixed\_of\_ideal} & イデアルは不動点                    & \S4.2 \\
    \texttt{idealPowTower\_closedTower}  & ClosedTower としての形式化             & \S4.3 \\
    \texttt{idealPow\_glb\_closed}    & glb の閉性                          & \S4.4 \\
    \texttt{idealPow\_glb\_is\_ideal} & glb はイデアル                      & \S4.4 \\
    \texttt{ringHom\_idealPow\_le}       & 環準同型と冪の保存                     & \S5.1 \\
    \texttt{ringHom\_towerHom}           & 誘導された塔の射                       & \S5.1 \\
    \texttt{ringHom\_towerHom\_comp}     & 合成の整合性                           & \S5.2 \\
    \texttt{IsSeparated}                 & 分離条件の定義                         & \S6.1 \\
    \texttt{isSeparated\_iff\_glb\_eq} & 分離条件 $\iff$ glb $= \{0\}$      & \S6.1 \\
    \texttt{escape\_of\_isSeparated}     & 非零元の脱出定理                       & \S6.2 \\
    \texttt{krull\_intersection\_statement} & Krull の交叉定理(Noetherian 環)   & \S6.3 \\
    \bottomrule
  \end{tabular}
  \caption{Level 5 で形式化した命題の一覧}
\end{table}

\subsection{次のステップの展望}

Level 5 の完成により,Structure Tower プロジェクトの第一フェーズ(L1--L5)が完結した。
第二フェーズとして以下が考えられる。

\begin{enumerate}
  \item \textbf{$I$-adic 完備化}:Cauchy 列による完備化 $\hat{R}$ の構成と StructureTower との対応。
  \item \textbf{Rees 代数}:$\bigoplus_n I^n t^n$ を StructureTower の直和として記述し,次数環との接続。
  \item \textbf{Kleisli 合成の一般化}:$\varphi \colon R \to S$ が一般の Kleisli 射である場合の naturality 条件の確立。
  \item \textbf{Mathlib \texttt{CategoryTheory.Monad} との正式接続}:ClosureOperator を Monad として定式化し,\texttt{CategoryTheory.Monad.Algebra} との整合性を示す。
  \item \textbf{2-圏的構造}:Hom 間の順序から enriched category(豊穣圏)へ,Structure Tower の圏論的精密化。
\end{enumerate}

% =============================================================================
\end{document}
