% =============================================================================
% Structure Tower 発展演習 Level 6
% I-adic 完備化と StructureTower:Cauchy 列・null 列・普遍性・ClosedTower
% LuaLaTeX + ltjsarticle
% =============================================================================
\documentclass[a4paper,11pt]{ltjsarticle}

% --- Core packages ---
\usepackage{luatexja-fontspec}
\usepackage{amsmath,amssymb,amsthm}
\usepackage{mathtools}
\usepackage{tikz-cd}
\usepackage{tikz}
\usetikzlibrary{arrows.meta,positioning,calc,shapes.geometric,decorations.pathmorphing,cd}
\usepackage{enumitem}
\usepackage{hyperref}
\usepackage{tcolorbox}
\tcbuselibrary{skins,breakable}
\usepackage{listings}
\usepackage{xcolor}
\usepackage{geometry}
\usepackage{fancyhdr}
\usepackage{float}
\usepackage{booktabs}
\usepackage{array}
\usepackage{caption}

% --- Page geometry ---
\geometry{margin=2.5cm, top=3cm, bottom=3cm}

% --- Fonts ---
\setmainjfont{Harano Aji Mincho}[
  BoldFont = Harano Aji Mincho,
  BoldFeatures = {FakeBold=2},
]
\setsansjfont{Harano Aji Gothic}
\setmonofont[Scale=0.85]{DejaVu Sans Mono}

% --- Colors ---
\definecolor{leanblue}{HTML}{2563EB}
\definecolor{leanbg}{HTML}{F8FAFC}
\definecolor{leancomment}{HTML}{6B7280}
\definecolor{leankeyword}{HTML}{7C3AED}
\definecolor{leanstring}{HTML}{059669}
\definecolor{accentcolor}{HTML}{1E40AF}
\definecolor{theoremcolor}{HTML}{EFF6FF}
\definecolor{definitioncolor}{HTML}{F0FDF4}
\definecolor{remarkcolor}{HTML}{FFF7ED}
\definecolor{warningcolor}{HTML}{FEF3C7}
\definecolor{completioncolor}{HTML}{F0F9FF}
\definecolor{cauchycolor}{HTML}{FDF4FF}

% --- Hyperref ---
\hypersetup{
  colorlinks=true,
  linkcolor=accentcolor,
  urlcolor=leanblue,
  citecolor=accentcolor,
  pdftitle={Structure Tower 発展演習 Level 6},
  pdfauthor={su},
}

% --- Theorem environments ---
\theoremstyle{definition}
\newtheorem{definition}{定義}[section]
\newtheorem{example}[definition]{例}
\newtheorem{construction}[definition]{構成}

\theoremstyle{plain}
\newtheorem{theorem}[definition]{定理}
\newtheorem{lemma}[definition]{補題}
\newtheorem{proposition}[definition]{命題}
\newtheorem{corollary}[definition]{系}

\theoremstyle{remark}
\newtheorem{remark}[definition]{注意}

% --- Lean code listing environment ---
\lstdefinelanguage{lean4}{
  morekeywords={def,theorem,lemma,example,instance,class,structure,
    where,let,in,do,if,then,else,match,with,fun,return,
    import,open,namespace,section,end,variable,
    inductive,abbrev,noncomputable,private,protected,
    sorry,admit,have,show,suffices,calc,by,exact,
    apply,intro,intros,constructor,cases,induction,
    simp,rw,rfl,ext,ring,linarith,omega,norm_num,
    decide,trivial,assumption,contradiction,
    Type,Prop,Sort,Set,true,false,
    attribute,deriving},
  sensitive=true,
  morecomment=[l]{--},
  morecomment=[n]{/-}{-/},
  morestring=[b]",
  literate=
    {->}{{\(\to\)}}2
    {<-}{{\(\leftarrow\)}}2
    {<->}{{\(\leftrightarrow\)}}3
    {<=}{{\(\leq\)}}2
    {>=}{{\(\geq\)}}2
    {:=}{{:=}}2
    {=>}{{\(\Rightarrow\)}}2,
}

\lstnewenvironment{leancode}[1][]{%
  \lstset{
    language=lean4,
    basicstyle=\ttfamily\footnotesize,
    keywordstyle=\color{leankeyword}\bfseries,
    commentstyle=\color{leancomment}\itshape,
    stringstyle=\color{leanstring},
    backgroundcolor=\color{leanbg},
    frame=single,
    rulecolor=\color{leanblue!30},
    framesep=8pt,
    xleftmargin=12pt,
    xrightmargin=12pt,
    breaklines=true,
    breakatwhitespace=true,
    showstringspaces=false,
    tabsize=2,
    captionpos=b,
    numbers=left,
    numberstyle=\tiny\color{leancomment},
    numbersep=8pt,
    aboveskip=1em,
    belowskip=1em,
    inputencoding=utf8,
    extendedchars=true,
    #1
  }%
}{}

% --- Tcolorbox styles ---
\newtcolorbox{proofstrategy}{
  colback=remarkcolor,
  colframe=orange!60!black,
  title={\textbf{証明戦略}},
  fonttitle=\sffamily,
  boxrule=0.5pt,
  arc=3pt,
}

\newtcolorbox{mathinsight}{
  colback=theoremcolor,
  colframe=accentcolor,
  title={\textbf{数学的洞察}},
  fonttitle=\sffamily,
  boxrule=0.5pt,
  arc=3pt,
}

\newtcolorbox{completionbox}{
  colback=completioncolor,
  colframe=leanblue,
  title={\textbf{完備化の概念}},
  fonttitle=\sffamily,
  boxrule=0.5pt,
  arc=3pt,
}

\newtcolorbox{levelguide}{
  colback=cauchycolor,
  colframe=leankeyword!70!black,
  title={\textbf{演習レベル指針}},
  fonttitle=\sffamily,
  boxrule=0.5pt,
  arc=3pt,
}

% --- Header/Footer ---
\pagestyle{fancy}
\fancyhf{}
\renewcommand{\headrulewidth}{0.4pt}
\fancyhead[L]{\small\sffamily\nouppercase{\leftmark}}
\fancyhead[R]{\small\sffamily Structure Tower Level 6}
\fancyfoot[C]{\thepage}

% --- Math macros ---
\newcommand{\Rhat}{\widehat{R}}
\newcommand{\Ihat}{\widehat{I}}
\newcommand{\NatD}{\mathbb{N}^{\mathrm{op}}}

% =============================================================================
\begin{document}

% --- Title ---
\title{%
  {\LARGE\sffamily\bfseries Structure Tower 発展演習 Level 6}\\[0.5em]
  {\large\sffamily $I$-adic 完備化と StructureTower:\\
  Cauchy 列・null 列・普遍性・ClosedTower}\\[0.3em]
  {\normalsize\sffamily Lean 4 / Mathlib4 による形式化とその数学的解説}%
}
\author{su}
\date{\today}
\maketitle

% --- AI Disclosure ---
\begin{center}
\small
\textit{AI assistance disclosure:}\\
Lean ソースコードは Claude (Anthropic) で骨格を生成し、
Codex (OpenAI) で修正した。\\
\TeX\ 文書は Claude Code (Anthropic) で生成した。\\
著者による加筆・修正は行っていない。\\
内容の正確性は保証されず、誤りがあれば著者の責任である。
\end{center}

\vspace{1em}

% --- Abstract ---
\begin{abstract}
本稿は Lean 4 / Mathlib4 を用いた Structure Tower 形式化シリーズの Level 6 として,
$I$-adic 完備化を \texttt{StructureTower} の言語で記述した演習ファイルの数学的解説である.
Level 1--5 で構築した \texttt{idealPowTower},\texttt{ClosedTower},\texttt{ringHom\_towerHom}
などの枠組みを全て統合し,
\begin{enumerate}
  \item \textbf{Cauchy 列の塔的定義}:$({\mathbb{N}} \to R)$ 上の \texttt{StructureTower} として Cauchy 列の「速さ」を階層化する;
  \item \textbf{null 列と I-adic Setoid}:null 列の代数的閉性と,分離条件下での定数 null 列の特徴付け;
  \item \textbf{完備化の普遍性}:\texttt{AdicCompletion} による $\widehat{R}$ の構成と,埋め込み $\iota\colon R \to \widehat{R}$ が誘導する塔準同型;
  \item \textbf{完備塔と ClosedTower}:$\widehat{R}$ 上の冪塔が \texttt{ClosedTower} かつ自動的に分離的であること;
\end{enumerate}
を段階的に論じる.Bourbaki の母構造論的な視点——順序構造・代数構造・位相構造の三位一体——が一つの形式的構成に収束する過程を追う.
\end{abstract}

\tableofcontents
\newpage

% =============================================================================
\section{導入:Bourbaki の完備化とその塔的解釈}
% =============================================================================

\subsection{Level 6 の動機}

Level 5 では,可換環 $R$ とそのイデアル $I$ に対して
\[
  \texttt{idealPowTower}(I)\colon \texttt{StructureTower}\; \mathbb{N}^{\mathrm{op}}\; R,
  \quad \texttt{level}(n) = I^n
\]
という canonical な例を構成し,L1--L4 の全構造がこの一例に合流することを確認した.

Level 6 の目標はその先にある:\textbf{$I$-adic 完備化}を \texttt{StructureTower} の言語で記述することである.
具体的には,
\[
  \text{Cauchy 列} \;\to\; \text{null 列} \;\to\; \text{商環} \;\to\; \text{完備化の普遍性} \;\to\; \text{ClosedTower}
\]
という可換環論の中核パイプラインを段階的に形式化する.

\subsection{Bourbaki の精神}

$I$-adic 完備化は Bourbaki の母構造の精神に忠実な例である:

\begin{center}
\begin{tikzpicture}[
  box/.style={draw, rounded corners=4pt, fill=theoremcolor, minimum width=3.5cm, minimum height=1cm, align=center, font=\small},
  arrow/.style={-Stealth, thick, leanblue},
]
  \node[box, fill=definitioncolor] (ord) at (0,0) {順序構造\\$\mathbb{N}^{\mathrm{op}}$ 添字の塔};
  \node[box, fill=remarkcolor] (alg) at (5,0) {代数構造\\環の Cauchy 列};
  \node[box, fill=completioncolor] (top) at (2.5,-2.5) {位相構造\\$I$-adic 位相の完備性};
  \node[box, fill=cauchycolor, minimum width=4cm] (comp) at (2.5,-5) {\textbf{$I$-adic 完備化}\\$\widehat{R} = \varprojlim R/I^n$};
  \draw[arrow] (ord) -- (comp);
  \draw[arrow] (alg) -- (comp);
  \draw[arrow] (top) -- (comp);
  \node at (2.5, -3.8) {\footnotesize 三つの構造が統合される};
\end{tikzpicture}
\end{center}

\begin{mathinsight}
Level 6 の核心的洞察:Cauchy 列の「速さ」を \texttt{StructureTower} のレベルとして捉える.
\[
  \texttt{level}(k) = \bigl\{x\colon \mathbb{N} \to R \;\big|\; \forall m,n,\; x_m - x_n \in I^{\min(m,n)+k}\bigr\}
\]
$k$ が大きいほど条件が厳しい(「速い」Cauchy 列).
$\mathbb{N}^{\mathrm{op}}$ で添字化すると減少族 $\to$ 増加族 となり,
これは \texttt{idealPowTower} と同じ $\mathbb{N}^{\mathrm{op}}$ パターンである.
\end{mathinsight}

\subsection{Level 6 の全体構成}

\begin{levelguide}
難易度マーク:
\begin{itemize}[leftmargin=2em]
  \item[$\bullet$] \textbf{🟢 緑}:直接的な定義展開・定義の組み合わせ
  \item[$\bullet$] \textbf{🟡 黄}:補題の選択と結合が必要
  \item[$\bullet$] \textbf{🔴 赤}:複数の概念を統合した本質的な推論
\end{itemize}
\end{levelguide}

\begin{center}
\begin{tikzcd}[column sep=large, row sep=large]
  \texttt{cauchySeqTower}(I) \ar[r, "\text{§L6-1}"]
  & \texttt{IsIAdicNull} \ar[r, "\text{§L6-2}"]
  & \texttt{iadicSetoid} \ar[d, "\text{商環}"] \\
  \texttt{ClosedTower}(\hat{R}) \ar[u, "\text{§L6-4}"']
  & \texttt{Hom(towerHom)} \ar[l, "\text{§L6-4}"']
  & \widehat{R} \ar[l, "\iota\text{: §L6-3}"]
\end{tikzcd}
\captionof{figure}{Level 6 の概念間の関係}
\end{center}

% =============================================================================
\section{\S L6-1:Cauchy 列の塔的定義}
% =============================================================================

\subsection{$I$-adic Cauchy 列}

\subsubsection{古典的定義と塔的定義の比較}

$I$-adic Cauchy 列の古典的定義は次のとおりである:

\begin{definition}[$I$-adic Cauchy 列(古典型)]
列 $x\colon \mathbb{N} \to R$ が \textbf{$I$-adic Cauchy} であるとは,
\[
  \forall k \in \mathbb{N},\; \exists N \in \mathbb{N},\; \forall m,n \ge N,\; x_m - x_n \in I^k
\]
が成り立つことをいう.
\end{definition}

一方,Lean の形式化では「一様型」を採用する:

\begin{definition}[$I$-adic Cauchy 列(一様型,Lean 版)]
列 $x\colon \mathbb{N} \to R$ が \textbf{$I$-adic Cauchy(一様型)} であるとは,
\[
  \forall m,n \in \mathbb{N},\; x_m - x_n \in I^{\min(m,n)}
\]
が成り立つことをいう(\texttt{IsIAdicCauchy} に対応).
\end{definition}

\begin{remark}
一様型は「$m$ 番目と $n$ 番目の差は $I^{\min(m,n)}$ に入る」という条件であり,
古典型の $k=0$ 相当に対応する.
これは \texttt{cauchySeqTower} のレベル $0$ に対応する.
\end{remark}

\subsubsection{Cauchy 列の塔構成}

\begin{construction}[\texttt{cauchySeqTower}]\label{constr:cauchyseq}
$R$ を可換環,$I \subseteq R$ をイデアルとする.
関数空間 $(\mathbb{N} \to R)$ 上の構造塔
\[
  \texttt{cauchySeqTower}(I)\colon \texttt{StructureTower}\; \mathbb{N}^{\mathrm{op}}\; (\mathbb{N} \to R)
\]
を次で定義する:
\[
  \texttt{level}(k) = \bigl\{x\colon \mathbb{N} \to R \;\big|\; \forall m,n \in \mathbb{N},\; x_m - x_n \in I^{\min(m,n) + k}\bigr\}
\]
ここで $k \in \mathbb{N}^{\mathrm{op}}$ は「速さ」パラメータである.
\end{construction}

\begin{figure}[H]
\centering
\begin{tikzpicture}[
  level/.style={draw, rounded corners=3pt, minimum width=6cm, minimum height=0.7cm,
                fill=cauchycolor, font=\small, align=center},
  arrow/.style={-Stealth, gray!70, thick},
]
  \node[level, fill=leanbg!80] (k0) at (0, 0) {level $0$: $x_m - x_n \in I^{\min(m,n)}$(一般 Cauchy 列)};
  \node[level] (k1) at (0, -1.4) {level $1$: $x_m - x_n \in I^{\min(m,n)+1}$(より速い)};
  \node[level] (k2) at (0, -2.8) {level $2$: $x_m - x_n \in I^{\min(m,n)+2}$(さらに速い)};
  \node[font=\large] (dots) at (0, -3.9) {$\vdots$};
  \node[level, fill=remarkcolor] (kinf) at (0, -5.1) {level $k$: $x_m - x_n \in I^{\min(m,n)+k}$($k$ が大きいほど厳しい)};
  \draw[arrow] (k0.south) -- (k1.north) node[midway, right, font=\small] {$\subseteq$};
  \draw[arrow] (k1.south) -- (k2.north) node[midway, right, font=\small] {$\subseteq$};
\end{tikzpicture}
\caption{Cauchy 列の塔:$k$ が大きいほど「速い」Cauchy 列のみを含む($\mathbb{N}^{\mathrm{op}}$ 添字で包含関係が増加的)}
\end{figure}

\begin{proposition}[\texttt{cauchySeqTower} の単調性]\label{prop:cauchy-mono}
$i \le j$ in $\mathbb{N}^{\mathrm{op}}$ ならば $\texttt{level}(i) \subseteq \texttt{level}(j)$.
\end{proposition}

\begin{proof}
$i \le j$ in $\mathbb{N}^{\mathrm{op}}$ とは $j^* \le i^*$ in $\mathbb{N}$(ただし $*$ は \texttt{ofDual})を意味する.
$x \in \texttt{level}(i)$ とする,すなわち任意の $m,n$ に対して $x_m - x_n \in I^{\min(m,n)+i^*}$.
$j^* \le i^*$ より $\min(m,n)+j^* \le \min(m,n)+i^*$,
ゆえに $I^{\min(m,n)+i^*} \subseteq I^{\min(m,n)+j^*}$(イデアルの冪の単調性).
よって $x_m - x_n \in I^{\min(m,n)+j^*}$,すなわち $x \in \texttt{level}(j)$.
\end{proof}

\subsubsection{代数的閉性}

Cauchy 列の塔には自然な代数構造が入る:

\begin{proposition}[定数列の帰属]
任意の $r \in R$ および $k \in \mathbb{N}^{\mathrm{op}}$ に対して,
定数列 $\overline{r} = (r, r, r, \ldots)$ は $\texttt{level}(k)$ に属する.
\end{proposition}

\begin{proof}
$\overline{r}_m - \overline{r}_n = r - r = 0 \in I^{\min(m,n)+k}$ より明らか($0$ は任意のイデアルの元).
\end{proof}

\begin{proposition}[加法的閉性]
$x, y \in \texttt{level}(k)$ ならば $x + y \in \texttt{level}(k)$.
\end{proposition}

\begin{proof}
$(x+y)_m - (x+y)_n = (x_m - x_n) + (y_m - y_n)$.
$x_m - x_n,\, y_m - y_n \in I^{\min(m,n)+k}$ なので,
\texttt{Ideal.add\_mem} より $(x_m - x_n) + (y_m - y_n) \in I^{\min(m,n)+k}$.
\end{proof}

\begin{proposition}[スカラー倍の閉性]
$r \in R$,$x \in \texttt{level}(k)$ ならば $(n \mapsto r \cdot x_n) \in \texttt{level}(k)$.
\end{proposition}

\begin{proof}
$r \cdot x_m - r \cdot x_n = r \cdot (x_m - x_n)$.
$x_m - x_n \in I^{\min(m,n)+k}$ なので,
$I^k$ は $R$-加群であるから $r \cdot (x_m - x_n) \in I^{\min(m,n)+k}$.
\end{proof}

\begin{remark}
これら三つの命題は,\texttt{cauchySeqTower} が \textbf{フィルトレーション $R$-加群}の構造を持つことを示している.
Level 5 で確認した乗法互換性(\texttt{cauchySeqTower\_mul\_mem})の加法・スカラー版に対応する.
\end{remark}

\begin{leancode}[caption={L6-1: Cauchy 列の塔定義}]
-- L6-1b: Cauchy 列の塔
def cauchySeqTower : StructureTower Nod (N -> R) where
  level k := {x : N -> R | forall m n, x m - x n in I ^ (min m n + ofDual k)}
  monotone_level := by
    intro i j hij x hx m n
    exact Ideal.pow_le_pow_right
      (Nat.add_le_add_left (OrderDual.ofDual_le_ofDual.mpr hij) _) (hx m n)

-- L6-1c: level 0 と IsIAdicCauchy の一致
theorem cauchySeqTower_level_zero :
    (cauchySeqTower I).level (OrderDual.toDual 0) =
      {x : N -> R | IsIAdicCauchy I x} := by
  ext x; simp [cauchySeqTower, IsIAdicCauchy]
\end{leancode}

% =============================================================================
\section{\S L6-2:null 列と同値関係}
% =============================================================================

\subsection{$I$-adic null 列}

\begin{definition}[$I$-adic null 列]
列 $x\colon \mathbb{N} \to R$ が \textbf{$I$-adic null} であるとは,
\[
  \forall k \in \mathbb{N},\; \exists N \in \mathbb{N},\; \forall n \ge N,\; x_n \in I^k
\]
が成り立つことをいう.すなわち,$x_n$ が $I$-adic 位相で $0$ に収束する.
\end{definition}

\begin{remark}
null 列と \texttt{cauchySeqTower} の \texttt{global} の対比:
\begin{align*}
  \texttt{global}(T) &= \bigcap_{k} \texttt{level}(k) = \{x \mid \forall k,\, \forall m,n,\, x_m - x_n \in I^k\} \\
  \text{null 列} &= \{x \mid \forall k,\, \exists N,\, \forall n \ge N,\, x_n \in I^k\}
\end{align*}
null 列は「すべてのレベルに \emph{eventually} 属する列」,
global は「すべてのレベルに \emph{常に} 属する列」である.
\end{remark}

\subsection{null 列の代数的閉性}

\begin{proposition}
次が成り立つ:
\begin{enumerate}
  \item 零列 $0$ は null 列である.
  \item $x, y$ が null ならば $x + y$ も null である.
  \item $x$ が null ならば $-x$ も null である.
\end{enumerate}
\end{proposition}

\begin{proof}
\begin{enumerate}
  \item $0_n = 0 \in I^k$ for all $n, k$.$N = 0$ で OK.
  \item 各 $k$ に対し $x, y$ のそれぞれから $N_x, N_y$ を取り,$N = \max(N_x, N_y)$ とおく.
    $n \ge N$ ならば $(x+y)_n = x_n + y_n \in I^k + I^k \subseteq I^k$(\texttt{Ideal.add\_mem}).
  \item $I^k$ はイデアルなので $-a \in I^k \iff a \in I^k$.
\end{enumerate}
\end{proof}

\subsection{$I$-adic Setoid}

\begin{construction}[$I$-adic Setoid]
$({\mathbb{N}} \to R)$ 上の同値関係を
\[
  x \sim y \;\iff\; x - y \text{ が null 列}
\]
と定義する.これは確かに同値関係である:
\begin{center}
\begin{tikzpicture}[
  node/.style={draw, rounded corners=3pt, fill=theoremcolor, font=\small, align=center, minimum width=3cm, minimum height=0.9cm},
  arrow/.style={-Stealth, thick},
]
  \node[node] (refl) at (0,0) {\textbf{反射性}\\$x - x = 0$ は null};
  \node[node] (symm) at (5,0) {\textbf{対称性}\\$y - x = -(x-y)$ は null};
  \node[node] (trans) at (2.5,-2) {\textbf{推移性}\\$x - z = (x-y)+(y-z)$ は null};
  \draw[arrow] (refl) -- (trans);
  \draw[arrow] (symm) -- (trans);
\end{tikzpicture}
\end{center}
\end{construction}

\begin{leancode}[caption={L6-2e: I-adic Setoid の構成}]
def iadicSetoid : Setoid (N -> R) where
  r x y := IsIAdicNull I (x - y)
  iseqv := {
    refl := fun x => by simp [isIAdicNull_zero I]
    symm := fun {x y} hxy => by
      have : y - x = -(x - y) := by ring
      rw [this]; exact isIAdicNull_neg I hxy
    trans := fun {x y z} hxy hyz => by
      have : x - z = (x - y) + (y - z) := by ring
      rw [this]; exact isIAdicNull_add I hxy hyz
  }
\end{leancode}

\subsection{分離条件と null 定数列}

\begin{theorem}[分離条件と null 定数列の対応]\label{thm:separated-null}
$I$ が \textbf{分離的}($\bigcap_{n} I^n = 0$)であるとき,
定数列 $(r, r, r, \ldots)$ が null 列であることと $r = 0$ は同値である:
\[
  \texttt{IsSeparated}(I) \implies
  \bigl[\text{const}(r) \text{ は null}\bigr] \iff r = 0.
\]
\end{theorem}

\begin{proof}
($\Rightarrow$) $r \neq 0$ と仮定する.分離条件より Level 5 の脱出定理(\texttt{escape\_of\_isSeparated})が使え,
$r \notin I^n$ となる $n$ が存在する.しかし const$(r)$ が null ならば,そのような $n$ に対して
$\exists N,\, r \in I^n$ となり矛盾する.

($\Leftarrow$) $r = 0$ ならば,定数列はゼロ列だから null である.
\end{proof}

\begin{mathinsight}
定理 \ref{thm:separated-null} は「\textbf{分離的} $\iff$ $I$-adic 位相が $T_1$ $\iff$ 定数 null 列は零列のみ」という等価性の形式化である.
これは後の $\iota\colon R \to \widehat{R}$ の単射性の根拠となる.
\end{mathinsight}

% =============================================================================
\section{\S L6-3:完備化の普遍性}
% =============================================================================

\subsection{Mathlib の $I$-adic 完備化}

Lean / Mathlib では,$I$-adic 完備化は次で与えられる:
\[
  \widehat{R} := \texttt{AdicCompletion}\, I\, R = \varprojlim_{n} R/I^n,
  \quad
  \iota := \texttt{algebraMap}\; R\; \widehat{R}\colon R \to \widehat{R}.
\]

\begin{construction}[完備化塔 \texttt{completionPowTower}]
$\widehat{R}$ 上のイデアル $\hat{I} := \iota(I) \cdot \widehat{R}$ に対して,
\[
  \texttt{completionPowTower}(I) := \texttt{idealPowTower}(\hat{I})\colon
  \texttt{StructureTower}\; \mathbb{N}^{\mathrm{op}}\; \widehat{R}
\]
を Level 5 の \texttt{idealPowTower} パターンをそのまま $\widehat{R}$ に適用して構成する.
\end{construction}

\subsection{埋め込み $\iota$ が誘導する塔準同型}

\begin{theorem}[塔準同型の誘導]\label{thm:tower-hom}
埋め込み $\iota\colon R \to \widehat{R}$ は,
\texttt{StructureTower} の準同型
\[
  \texttt{completion\_towerHom}(I)\colon \texttt{idealPowTower}(I) \to \texttt{completionPowTower}(I)
\]
を誘導する.
\end{theorem}

\begin{proof}
Level 5 で確立した \texttt{ringHom\_towerHom} の canonical な適用である:
$\phi = \iota$,$J = \hat{I}$,条件 $\hat{I} \le \hat{I}$(恒等的)より直ちに従う.
\end{proof}

\begin{figure}[H]
\centering
\begin{tikzcd}[column sep=5em, row sep=4em]
  \texttt{idealPowTower}(I) \ar[r, "\texttt{completion\_towerHom}", "\iota"'] \ar[dr, "\phi"', bend right=20]
  & \texttt{completionPowTower}(I) \ar[d, "\exists!\, \bar{\phi}", dashed] \\
  & \texttt{idealPowTower}(J) \quad (S \text{ が完備})
\end{tikzcd}
\caption{完備化の普遍性:任意の塔準同型 $\phi$ は $\iota$ を経由して一意分解する}
\label{fig:univ-prop}
\end{figure}

\begin{proposition}[合成の互換性]\label{prop:comp-compat}
$\phi\colon S \to R$ が環準同型,$J \subseteq S$ がイデアルで $\phi(J) \subseteq I$ を満たすとき,
\[
  \texttt{completion\_towerHom}(I) \circ \texttt{ringHom\_towerHom}(\phi, J, I)
  = \texttt{ringHom\_towerHom}(\iota \circ \phi,\, J,\, \hat{I})
\]
が成り立つ.
\end{proposition}

\begin{proof}
両辺の \texttt{toFun} を比べると,どちらも $\iota \circ \phi$ に等しい.
\texttt{Hom.ext rfl} で終わる.
\end{proof}

\begin{figure}[H]
\centering
\begin{tikzcd}[column sep=4em, row sep=3.5em]
  \texttt{idealPowTower}(J)
    \ar[r, "\texttt{ringHom\_towerHom}(\phi)"]
    \ar[rr, bend left=30, "\texttt{ringHom\_towerHom}(\iota \circ \phi)"]
  & \texttt{idealPowTower}(I)
    \ar[r, "\texttt{completion\_towerHom}"]
  & \texttt{completionPowTower}(I)
\end{tikzcd}
\caption{合成の可換性:命題 \ref{prop:comp-compat} の図式的表現}
\end{figure}

\subsection{分離条件下での単射性}

\begin{theorem}[$\iota$ の単射性]\label{thm:injective}
$I$ が分離的であるならば,$\iota\colon R \to \widehat{R}$ は単射である:
\[
  \texttt{IsSeparated}(I) \implies \texttt{Function.Injective}(\iota).
\]
\end{theorem}

\begin{proof}
\texttt{isHausdorff\_of\_isSeparated} により $R$ は $I$-adic Hausdorff となる.
Mathlib の \texttt{AdicCompletion.of\_inj} を使うと,
\texttt{AdicCompletion.of} $I\, R$ が単射であることが分かる.
\texttt{completionMap} と \texttt{AdicCompletion.of} は同じ埋め込みであるから,結論を得る.
\end{proof}

\begin{mathinsight}
分離条件 $\bigcap_n I^n = 0$ は「$I$-adic フィルトレーションが $R$ の元を十分区別できる」ことを意味し,
それが $\iota$ の単射性を保証する.
図式的に:
\begin{center}
\begin{tikzcd}
  R \ar[rr, "\iota", hook] \ar[dr, "\text{商}"']
  & & \widehat{R} \\
  & R/(I^n) \ar[ur, "\varprojlim"']
\end{tikzcd}
\end{center}
\end{mathinsight}

\begin{leancode}[caption={L6-3b: 完備化塔準同型と単射性}]
-- L6-3b: iota が tower hom を誘導する(ringHom_towerHom の直接適用)
noncomputable def completion_towerHom :
    Hom (idealPowTower I) (completionPowTower I) :=
  ringHom_towerHom (completionMap I) I (completionIdeal I) le_rfl

-- L6-3e: 分離条件下での iota の単射性
theorem completion_towerHom_injective_of_separated
    (hI : IsSeparated I) :
    Function.Injective (completionMap I) := by
  let _ : IsHausdorff I R := isHausdorff_of_isSeparated I hI
  intro x y hxy
  have hxy' : AdicCompletion.of I R x = AdicCompletion.of I R y := by
    simpa only [completionMap, completionRing] using hxy
  exact (AdicCompletion.of_inj (I := I) (M := R)).mp hxy'
\end{leancode}

% =============================================================================
\section{\S L6-4:完備塔と ClosedTower}
% =============================================================================

\subsection{完備化塔の ClosedTower 構造}

Level 5 では \texttt{idealPowTower}$(I)$ が \texttt{idealClosure}($= $ イデアルの Span 作用素)に関する ClosedTower であることを示した.
Level 6 では,この結果を $\widehat{R}$ に「コピー」する:

\begin{theorem}[完備化塔は ClosedTower]
\[
  \texttt{completionPowTower}(I) = \texttt{idealPowTower}(\hat{I})
\]
は,$\widehat{R}$ 上の \texttt{idealClosure} に関する ClosedTower である.
\end{theorem}

\begin{proof}
Level 5 の \texttt{idealPowTower\_closedTower} を $\hat{I}$ に適用するだけである:
各 $n$ に対して $\hat{I}^n$ はイデアルなので,その \texttt{Ideal.span} は自分自身に等しい.
\end{proof}

\subsection{完備化の分離性}

\begin{theorem}[完備化は自動的に分離的]\label{thm:completion-sep}
\[
  \texttt{IsSeparated}(\hat{I}),
  \quad\text{すなわち}\quad
  \bigcap_{n \in \mathbb{N}} \hat{I}^n = 0 \;\text{ in }\; \widehat{R}.
\]
\end{theorem}

\begin{proof}
Mathlib は $\widehat{R}$ に対して $\hat{I}$-adic Hausdorff 性を付与している
(\texttt{IsHausdorff.map\_algebraMap\_iff} と \texttt{inferInstance}).
これより \texttt{isSeparated\_of\_isHausdorff} を適用して分離性を得る.
\end{proof}

\begin{corollary}[完備化の global は $\{0\}$]
\[
  \texttt{global}(\texttt{completionPowTower}(I)) = \{0\} \subseteq \widehat{R}.
\]
\end{corollary}

\begin{proof}
定理 \ref{thm:completion-sep} と Level 5 の \texttt{isSeparated\_iff\_global\_eq} より直ちに従う.
\end{proof}

\subsection{完備化版・脱出定理}

\begin{theorem}[脱出定理(完備化版)]\label{thm:escape}
$x \in \widehat{R}$ が $x \neq 0$ ならば,
\[
  \exists n \in \mathbb{N},\quad x \notin \hat{I}^n.
\]
すなわち,零でない元は有限段で $\hat{I}$-adic tower から脱出する.
\end{theorem}

\begin{proof}
定理 \ref{thm:completion-sep} と Level 5 の \texttt{escape\_of\_isSeparated} の直接適用:
\texttt{escape\_of\_isSeparated} $(\hat{I})$ (\texttt{completionPowTower\_isSeparated} $I$) $hx$.
\end{proof}

\subsection{global の閉性}

\begin{theorem}[global の閉性]\label{thm:global-closed}
\[
  \texttt{idealClosure}(\texttt{global}(\texttt{completionPowTower}(I)))
  \subseteq \texttt{global}(\texttt{completionPowTower}(I)).
\]
\end{theorem}

\begin{proof}
\texttt{completionPowTower} が ClosedTower であることから,
Level 3 の \texttt{ClosedTower.cl\_global\_subset} を直接適用する.
\end{proof}

\begin{remark}
定理 \ref{thm:global-closed} と系(global $= \{0\}$)を合わせると,
\texttt{idealClosure} $\{0\} \subseteq \{0\}$,
すなわち $\{0\}$ が \texttt{idealClosure} の不動点であることが確認できる(自明だが枠組みの整合性の検証).
\end{remark}

\begin{figure}[H]
\centering
\begin{tikzcd}[column sep=5em, row sep=3.5em]
  R \ar[r, "\iota", hook] \ar[d, "\texttt{idealPowTower}(I)"']
  & \widehat{R} \ar[d, "\texttt{completionPowTower}(I)"] \\
  \texttt{StructureTower}\; \mathbb{N}^{\mathrm{op}}\; R
    \ar[r, "\texttt{completion\_towerHom}"]
    \ar[d, "\text{ClosedTower (L5)}"']
  & \texttt{StructureTower}\; \mathbb{N}^{\mathrm{op}}\; \widehat{R}
    \ar[d, "\text{ClosedTower (L6)}"] \\
  \texttt{global} = \{0\} \text{ iff Separated}
    \ar[r, "\text{自動的に成立}"]
  & \texttt{global} = \{0\} \text{ (常に)}
\end{tikzcd}
\caption{L5 と L6 の対応:$R$ の構造が $\widehat{R}$ に「自動改善」される}
\label{fig:l5-l6-compare}
\end{figure}

\begin{leancode}[caption={L6-4: 完備塔の ClosedTower と分離性}]
-- L6-4a: 完備化塔は ClosedTower
noncomputable def completionPowTower_closedTower :
    ClosedTower (idealClosure (R := completionRing I)) Nod :=
  idealPowTower_closedTower (completionIdeal I)

-- L6-4b: 完備化は自動的に分離的
theorem completionPowTower_isSeparated :
    IsSeparated (completionIdeal I) := by
  let _ : IsHausdorff (completionIdeal I) (completionRing I) :=
    (IsHausdorff.map_algebraMap_iff (I := I) (S := completionRing I)).2
      (inferInstance : IsHausdorff I (completionRing I))
  exact isSeparated_of_isHausdorff (completionIdeal I)

-- L6-4e: global の閉性
theorem completionPow_global_closed :
    idealClosure (R := completionRing I) (completionPowTower I).global
      subseteq (completionPowTower I).global :=
  (completionPowTower_closedTower I).cl_global_subset
\end{leancode}

% =============================================================================
\section{Level 6 の全体像:L1--L6 の合流}
% =============================================================================

\subsection{各レベルの対応表}

\begin{center}
\begin{tabular}{lll}
\toprule
\textbf{概念} & \textbf{L5 での確認} & \textbf{L6 での発展} \\
\midrule
\texttt{StructureTower} & \texttt{idealPowTower}$(R)$ & Cauchy 塔$(\mathbb{N} \to R)$, 完備化塔$(\widehat{R})$ \\
乗法互換性 & \texttt{mul\_mem} & 加法・スカラー閉性 \\
\texttt{ClosedTower} & L5-2c & 完備化版(L6-4a) \\
\texttt{cl\_global\_subset} & L5-2d & 完備化版(L6-4e) \\
環準同型 $\to$ Hom & \texttt{ringHom\_towerHom} & $\iota$ による塔準同型(L6-3b) \\
分離条件 & \texttt{IsSeparated} 検証 & $\widehat{R}$ では自動成立(L6-4b) \\
脱出定理 & \texttt{escape\_of\_isSeparated} & 完備化版(L6-4d) \\
\texttt{global} $= \{0\}$ & 仮定として要求 & 完備化で常に成立(L6-4c) \\
\bottomrule
\end{tabular}
\end{center}

\subsection{完備化の「最良性」}

\begin{mathinsight}
Bourbaki の完備化の普遍性:$\widehat{R}$ は
\begin{enumerate}
  \item \textbf{StructureTower}:$\hat{I}$-adic フィルトレーションを持つ;
  \item \textbf{ClosedTower}:各 $\hat{I}^n$ が \texttt{idealClosure} の不動点;
  \item \textbf{分離的}:$\bigcap_n \hat{I}^n = 0$(自動的に成立);
  \item \textbf{完備的}:Cauchy 列が収束する;
\end{enumerate}
を同時に満たす $R$ の「最小拡大」である.
Level 1--5 で蓄積した全条件が,完備化において初めて一斉に実現される.
\end{mathinsight}

\begin{figure}[H]
\centering
\begin{tikzpicture}[
  node/.style={draw, rounded corners=5pt, minimum width=4.5cm, minimum height=1cm, align=center, font=\small},
  arrow/.style={-Stealth, thick, leanblue},
  label/.style={font=\footnotesize, midway, above},
]
  \node[node, fill=leanbg] (L1) at (0, 8) {L1: \texttt{StructureTower}\\基本定義・圏構造};
  \node[node, fill=leanbg] (L2) at (0, 6.3) {L2: \texttt{Hom}・合成・自然性};
  \node[node, fill=leanbg] (L3) at (0, 4.6) {L3: \texttt{ClosedTower}\\閉包作用素};
  \node[node, fill=leanbg] (L4) at (0, 2.9) {L4: \texttt{NatInclusion}\\自然な包含関係};
  \node[node, fill=warningcolor] (L5) at (0, 1.2) {L5: \texttt{idealPowTower}\\$I$-adic canonical 例};
  \node[node, fill=cauchycolor, minimum width=5.5cm] (L6) at (0, -0.8)
    {L6: $I$-adic 完備化\\Cauchy 列・普遍性・ClosedTower};
  \draw[arrow] (L1) -- (L2);
  \draw[arrow] (L2) -- (L3);
  \draw[arrow] (L3) -- (L4);
  \draw[arrow] (L4) -- (L5);
  \draw[arrow] (L5) -- (L6) node[label] {};
  \node[font=\footnotesize, right] at (0.75, 0.2) {全構造が合流};
\end{tikzpicture}
\caption{Level 1--6 の積み上げ構造}
\end{figure}

\subsection{次のステップ候補(Level 7 以降)}

\begin{enumerate}
  \item \textbf{Cauchy 列の収束と完備性}:
    \texttt{cauchySeqTower} の元が $\widehat{R}$ 上で収束することを示す.
    逆極限 $\varprojlim R/I^n$ との同型を構成し,
    \texttt{StructureTower} としての同値性を確認する.

  \item \textbf{Rees 代数と次数付き構造}:
    $\bigoplus_n I^n t^n$ を \texttt{StructureTower} の直和として記述する.
    次数環 (graded ring) と \texttt{StructureTower} の「次数付き射」の関係を形式化する.

  \item \textbf{Mathlib \texttt{CategoryTheory.Monad} との正式接続}:
    \texttt{idealClosure} の \texttt{ClosedTower} が
    \texttt{CategoryTheory.Monad.Algebra} と同型であることを形式証明する.
    Level 3 のモナド法則の圏論版への橋渡し.
\end{enumerate}

% =============================================================================
\section*{まとめ}
% =============================================================================
\addcontentsline{toc}{section}{まとめ}

本稿では,Structure Tower 演習 Level 6 として $I$-adic 完備化を \texttt{StructureTower} の言語で記述した Lean 4 ファイルを数学的に解説した.

\begin{itemize}
  \item \S L6-1 では Cauchy 列の「速さ」を階層化した \texttt{cauchySeqTower} を構成し,加法・スカラー倍の閉性を確認した.
  \item \S L6-2 では null 列の代数的閉性から $I$-adic Setoid を構成し,分離条件下で定数 null 列が零列のみであることを示した.
  \item \S L6-3 では Mathlib の \texttt{AdicCompletion} を利用して完備化塔を構成し,埋め込み $\iota$ が塔準同型を誘導することを確認した(Level 5 の \texttt{ringHom\_towerHom} の canonical 適用).
  \item \S L6-4 では完備化塔が \texttt{ClosedTower} であり,自動的に分離的かつ global $= \{0\}$ を満たすことを示した.
\end{itemize}

Bourbaki の母構造論的な視点——順序・代数・位相の三位一体——が,\texttt{StructureTower} という単一の枠組みに統合される過程を,Lean 4 の形式的証明として明示できた.

% =============================================================================
\end{document}
