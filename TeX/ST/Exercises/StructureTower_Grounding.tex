\documentclass[11pt,a4paper,lualatex,ja=standard]{bxjsarticle}
\usepackage{amsmath,amssymb,amsthm}
\usepackage{mathtools}
\usepackage{tikz-cd}
\usepackage{hyperref}
\usepackage{xcolor}
\usepackage{graphicx}

\theoremstyle{definition}
\newtheorem{theorem}{定理}[section]
\newtheorem{definition}[theorem]{定義}
\newtheorem{lemma}[theorem]{補題}
\newtheorem{example}[theorem]{例}
\newtheorem{remark}[theorem]{注意}

\title{構造の塔:具体例の接地(位相と代数への応用)}
\author{su}
\date{\today}

\begin{document}

\maketitle

\begin{center}
\small
\textit{AI assistance disclosure:}
Lean ソースコードは Claude (Anthropic) で骨格を生成し、
Codex (OpenAI) で修正した。
\TeX 文書は Gemini 3.1Pro / Antigravity (Google DeepMind) で生成した。
著者による加筆・修正は行っていない。
内容の正確性は保証されず、誤りがあれば著者の責任である。
\end{center}
\begin{abstract}
本稿では、抽象的に構築された閉包作用素(Closure Operator)理論を位相空間論と群論の2つの分野に適用し、「構造の塔(Structure Tower)」の枠組みが具体的にどのように接地(grounding)されるかを解説する。抽象的なAPIが異なる数学的分野でも同一のインターフェースとして機能することを確認する。
\end{abstract}

\section{序論}
抽象的な閉包作用素 $cl$ は、集合 $S$ に対して $S \subseteq cl(S)$ (拡大性)、$S \subseteq T \implies cl(S) \subseteq cl(T)$ (単調性)、および $cl(cl(S)) = cl(S)$ (冪等性)を満たす作用素として定義される。
これを用いて、各レベル $T_i$ が $cl(T_i) = T_i$ を満たすような構造の塔を \textbf{ClosedTower} と呼ぶ。
本稿では、この $cl$ を以下の2つの具体的な例に適用し、理論の汎用性を実証する:
\begin{itemize}
    \item \textbf{位相空間論}:$cl$ を位相的閉包(topological closure)に対応させる。
    \item \textbf{群論}:$cl$ を部分群生成(subgroup generation)に対応させる。
\end{itemize}

\section{位相的閉包への接地}
位相空間 $\alpha$ における閉包作用素を $\overline{(\cdot)}$ と書く。これは \texttt{topClosure} として定義され、抽象的な閉包作用素の3つの公理を満たす。

\begin{theorem}
位相空間 $\alpha$ において、部分集合 $S \subseteq \alpha$ が閉集合であることは、$S = \overline{S}$ となることと同値である。
\end{theorem}

これより、各レベル $T_i$ が位相的に閉集合であるような「閉集合の塔」は、まさに \texttt{topClosure} に関する ClosedTower に他ならない。

\begin{example}
定数閉集合塔(Constant Closed Tower)は、ある一つの閉集合 $S$ をすべてのレベルに配置した塔であり、これは自明に ClosedTower となる。
\end{example}

\section{部分群生成への接地}
次に、群 $G$ における部分集合 $S \subseteq G$ が生成する部分群 $\langle S \rangle$ を考える。この生成作用素は、写像 $S \mapsto \langle S \rangle$ として、集合 $G$ 上の閉包作用素 \texttt{subgroupClosure} を定める。

\begin{theorem}
群 $G$ の部分集合 $S \subseteq G$ に対して、ある部分群 $H \le G$ が存在して $S = H$ となること($S$ が部分群の台集合であること)は、$S = \langle S \rangle$ となることと同値である。
\end{theorem}

これにより、各レベル $T_i$ が部分群をなす「部分群の塔」は、\texttt{subgroupClosure} に関する ClosedTower として特徴付けられる。

\section{抽象APIの統合と可視化}
抽象レベルで構築した塔のモナド的API(\texttt{liftCl}, \texttt{unit}, \texttt{algebra} など)が、位相論的および代数的な枠組みの両方でそのまま適用できる。

具体的には、塔 $T$ の各レベルに閉包作用素を適用する操作 $\mathsf{liftCl}(T)$ を考える。
\begin{align*}
    (\mathsf{liftCl}_{\text{top}}(T))_i &= \overline{T_i} \quad \text{(レベルごとの閉包)}\\
    (\mathsf{liftCl}_{\text{sub}}(T))_i &= \langle T_i \rangle \quad \text{(レベルごとの生成部分群)}
\end{align*}

このとき、部分集合から閉包への包含写像 \texttt{unit} と、既に閉である場合の恒等写像 \texttt{algebra} は次のような可換図式で表される:

\begin{figure}[h]
    \centering
    \begin{tikzcd}
        T \arrow[r, "\mathsf{unit}"] \arrow[dr, "\mathrm{id}"'] & \mathsf{liftCl}(T) \arrow[d, "\mathsf{algebra}"] \\
         & T
    \end{tikzcd}
    \caption{ClosedTowerにおけるunitとalgebraの隨伴関係}
\end{figure}

なお、この $\mathsf{algebra}$ 射は、$T$ が ClosedTower(各レベルが閉集合、または部分群)である場合にのみ存在し、このとき $\mathsf{algebra} \circ \mathsf{unit} = \mathrm{id}$ が成り立つ。

さらに、大域的な元の共通部分について、以下の定理が両方の文脈で同一の証明構造から導かれる:
\begin{theorem}
$T$ が ClosedTower のとき、$T$ の全体共通部分(global section)も閉である。
すなわち、
\[
    \overline{\bigcap_i T_i} \subseteq \bigcap_i T_i \quad (\text{位相的閉包})
\]
\[
    \left\langle \bigcap_i T_i \right\rangle \subseteq \bigcap_i T_i \quad (\text{部分群生成})
\]
\end{theorem}
この事実は、異なる数学的ドメインにおける事象が、単一の抽象APIを通して統一的に扱えることを如実に示している。

\end{document}
